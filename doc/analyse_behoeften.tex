Bij dit spel is het de bedoeling dat deelnemers tijdens de duur van het spel (dat loopt over enkele weken/maanden) virtueel aandelen, opties, trackers, fondsen, enz. kunnen verhandelen om zo op het einde van het spel een zo hoog mogelijk rendement neer te zetten. Hiervoor krijgt iedereen aan het begin van het spel eenzelfde budget toegewezen. 

Voorafgaand aan de start van het spel is een inschrijvingsperiode. Hierbij spelers kunnen zich registeren met behulp van hun eID. 

Het spel start op een vastgestelde dag en tijdstip: 1 maart 2010 om 8u 's morgens. Elke speler krijgt een startbudget van €100.000.
\todo{Waarom fixed registratieperiode?}
Vanaf dat moment hebben de gebruikers toegang tot hun portefolio en kunnen effecten worden verhandeld.

De gebruiker kan effecten aankopen van verschillende bronnen:
\begin{itemize}
\item{aandelen op de Continumarkt van Brussel}
\item{aandelen op de Eurolist van Parijs}
\item{aandelen op de "Lokaal" van Amsterdam}
\item{trackers op Euronext Amsterdam}
\end{itemize}

Het spel eindigt op 31 mei 2010, waarna een eindklassement wordt opgemaakt. De koersen op de website worden niet meer ge\"updatet, en er kunnen ook geen transacties meer gedaan worden. Gebruikers kunnen wel nog steeds inloggen op de website om statistieken te raadplegen.
