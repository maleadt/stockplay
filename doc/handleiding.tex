%
% Installatiehandleiding
%

\section{Installatiehandleiding}

\subsection{Initialisatie van de database}

\begin{enumerate}
\item{Installeer Oracle Database 10g Express Edition}
\item{Cre\"er de tablespace ``StockPlay'':
\begin{verbatim}
CREATE TABLESPACE STOCKPLAY 
DATAFILE 'stockplay.dat' 
SIZE 50M autoextend on
\end{verbatim}
}
\item{Maak een gebruiker ''stockplay'' aan met paswoord 'chocolademousse' en stel de default tablespace in op ``stockplay''}
\item{Log in met de zonet aangemaakte gebruiker, en voer het SQL-script ``installatiescript.sql'' uit}
\end{enumerate}

Om vervolgens de database op te vullen met een set aan historische data (wat niet verplicht is, maar bijvoorbeeld de performantie van de kunstmatige intelligentie verhoogt, alsook het uitzicht van de grafieken verbeterd), kan de historische scraper gebriukt worden. Hiervoer moet echter eerst eenmaal de reguliere scraper uitgevoerd worden, zodat de benodigde informatie duidende op aandelen, indexen en beurzen zich in de database bevindt. Vervolgens kan men het script ``stockplay-historic'' gebruiken om de database aan de hand van de ingevulde informatie wat betreft aandelen etc, koersen van de Euronext site op te halen die tot 2 jaar terug in de tijd gaan.


\subsection{Backend}

\paragraph{Automatisch} Gebruik het installatieprogramma.

\paragraph{Manueel}
In geval van een manuele installatie (voor wanneer de installer niet bruikbaar is, bijvoorbeeld op een Linux-based systeem), moeten de volgende stappen doorlopen worden:
\begin{enumerate}
\item Installatie van Tomcat 6.x
\item Toevoegen van relevante libraries aan het Tomcat classpath
\begin{itemize_compact}
\item log4j-1.2.16.jar
\item xmlrpc-client-3.1.3.jar
\item xmlrpc-common-3.1.3.jar
\item xmlrpc-server-3.1.3.jar
\item commons-dbcp-1.4.jar
\item commons-logging-1.1.jar
\item commons-pool-1.5.4.jar
\item cache4j\_0.4.jar
\item ojdbc14.jar
\item ws-commons-util-1.0.2.jar
\end{itemize_compact}
\item Registreren van de backend als Tomcat webapplicatie
\begin{itemize}
\item Activeer en surf naar de Tomcat webapp manager
\item Upload stockplay\_backend.war
\end{itemize}
\end{enumerate}

\subsection{Webserver}

\begin{enumerate}
\item{Installeer de IIS role met ASP.NET 3.5 op een Windows Server besturingsysteem}
\item{Maak een nieuwe site aan en plaats er de bestanden in}
\item{Eventueel dienen de DNS records van de domeinnaam nog aangepast te worden om naar de nieuwe installatie te verwijzen}
\end{enumerate}

\subsection{Administratie desktopclient}

\begin{enumerate}
\item Installeer de \makeurl{e-ID middleware}{http://eid.belgium.be/nl/Hoe\_installeer\_je\_de\_eID/Quick\_Install/} (minimum versie 3.5)
\item{Voer het installatiebestand AdministrationSetup.exe uit}
\end{enumerate}

\subsection{Perl componenten}

Voor de installatie van alle in Perl geschreven componenten (de AI, de scraper, de historische scraper, en enkele andere hulpscripts) wordt er voorzien in een CPAN compatibel pakket ``StockPlay-0.01.tar.gz''. Deze kan ge\"installeerd worden met een CPAN-compatibele package manager naar keuze (zoals CPAN.pm of ActiveState's PPM), maar een manuele installatie blijft ook altijd mogelijk:
\begin{itemize}
\item Pak het archief ``StockPlay-0.01.tar.gz'' uit, en verander de huidige directory naar waar het archief uitgepakt is.
\item Configureer de build-environment door ``perl Build.PL'' uit te voeren. Hiervoor is de module \emph{Module::Builder} nodig.
\item Test of de meegeleverde code correct op het systeem werkt door ``./Build test'' uit te voeren.
\item Installeer de modules via ``./Build install''. Hiervoor zijn administrator-rechten nodig als de doeldirectory een systeempad is. Om dit te vermijden kan gebruik gemaakt worden van een environment geconfigureerd met \emph{local::lib}.
\end{itemize}

Hierna zijn de relevante scripts (``stockplay-scraper'', ``stockplay-ai'', ``stockplay-benchmark'', etc) beschikbaar op het systeem. Om te configureren op welke server en met welke credentials het framework connecteerd, moet het bestand ``config'' die zich bevindt in de subdirectory ``.stockplay'' van de gebruiker zijn thuisfolder, correct gewijzigd worden. Zo kan men daar elke module configuratieparameters meegeven (zie daarvoor de codedocumentatie). Een voorbeeld van dergelijke directives:
\begin{verbatim}
[factory]
server=http://be04.kapti.com:6800/backend/public
username=foo
password=bar
\end{verbatim}

Hiervoor moet natuurlijk eerst een voldoende geprivilegieerd gebruiker aangemaakt worden.


\subsection{Transactiemanager}

\begin{enumerate}
\item{Installeer het script zodat de .jar automatisch opgeroepen wordt na een serverboot}
\item{De transactiemanager blijft draaien en zal periodiek zijn order controleren en eventueel uitvoeren}
\end{enumerate}

\subsection{Puntenmanager}
Deze handleiding gaat ervan uit dat de puntenmanager ge\"installeerd wordt op een Linux besturingssysteem.
\begin{enumerate}
\item{Kopieer de map van de puntenmanager inclusief lib-map en config.txt bestand naar de gewenste directory. Als voorbeeld nemen we de home map van root.}
\item{Maak een leeg tekstbestand aan met als naam puntenmanager.cron en plaats daarin volgende Cronjob: '0 0 * * 1-5 java -jar /home/root/PointsManager/PointManager.jar /home/Dieter/PointsManager/config.txt'. Deze cronjob zal iedere weekdag de puntenberekening uitvoeren}
\item{Open een terminal en ga naar de directory waar dit tekstbestand is aangemaakt.  Voer dan volgend command uit: 'crontab puntenmanager.cron'}
\item{Voer vervolgens 'crontab -l' uit om te verifi\"eren dat de cronjob correct is toegevoegd.}
\end{enumerate}

\subsection{Mobiele client}
\textbf{Opmerking}: Omdat het onmogelijk is om bij de vastgestelde testopstelling van dit project een URL bereikbaar te maken vanaf het internet, is het niet mogelijk om de Java ME-client uit te testen op een echte GSM. Er is immers geen haalbare oplossing om een verbinding op te zetten tussen de server waar de backend van dit project op draait, en de GSM zelf. Daarom kan de applicatie enkel worden opgeleverd in een emulator. Er is voor gekozen om gebruik te maken van de Sony Ericsson Java ME-emulator, omdat de standaard Sun Java-emulator enkel toe laat om de applicatie vanuit netbeans te starten.

\begin{enumerate}
\item{Installeer de Sony Ericsson Java SDK (te vinden in de zipfile 99962-semc\_java\_me\_cldc\_sdk\_2\_5\_0\_6.zip)}
\item{Open de emulator via Start $\Rightarrow$ Programs $\Rightarrow$ Sony Ericsson $\Rightarrow$ Java ME SDK for CLDC $\Rightarrow$ Run MIDP Application}
\item{Er verschijnt nu een dialoogvenster waarin je de Java ME-jad kan selecteren. Navigeer naar de juiste map en kies ...}
\todo aanvullen
\end{enumerate}


%
% Gebruikershandleiding
%

\section{Gebruikershandleiding}

\subsection{Administratie desktopclient}

De desktopapplicatie bevindt zich op de computer ``Nernst''.
Op deze machine kan je inloggen met de gebruikersnaam ``Administrator'' en het bekende wachtwoord ``e=mc**2''.
De desktopapplicatie bevindt zich in de map ''VOP Groep 7'' op het bureaublad en kan je opstarten door middel van de snelkoppeling ``StockPlay administratie''.

\subsubsection{Algemene zaken}
\label{sec:handl:admin:algemeen}

\paragraph{Verifi\"eren connectiviteit}
\begin{wrapfigure}{r}{0.5\textwidth}
	\centering
		\includegraphics[width=0.48\textwidth]{images/handleiding/administratie/verbindingsfout}
	\caption{Een verbindingsfout}
	\label{fig:handl:admin:verbindingsfout}
\end{wrapfigure}

Indien bij het opstarten van de applicatie de StockPlay servers niet toegankelijk blijken te zijn, wordt de gebruiker hiervan ge\"informeerd zoals te zien is in figuur \ref{fig:handl:admin:verbindingsfout}. Controleer of de PC waarop de applicatie gedraaid wordt, wel internetconnectiviteit heeft. Indien dit het geval is, moeten er contact opgenomen worden met de ondersteunende dienst om de backend en client-configuratie te controleren.

\paragraph{Inloggen}

Mogelijke inloggegevens zijn "Administrator'' als gebruikersnaam en "chocolademousse" als wachtwoord. Vul die gegevens in en druk op ``Login'', zoals te zien is in figuur \ref{fig:handl:admin:login}. Indien verkeerde credentials ingevoerd worden, krijgt de gebruiker hier melding van zoals te zien is in figuur \ref{fig:handl:admin:login-failure}.
Je kan ook zelf een account aanmaken, zie daarvoor \ref{sec:handl:admin:gebruikers}. Als deze de ``Administrator'' rechten bezit, dan kan je hiermee ook inloggen.

\begin{figure}
	\centering
	\subfloat[Het aanmeldformulier]{		\includegraphics[width=0.5\textwidth]{images/handleiding/administratie/login}
	\label{fig:handl:admin:login}}
	\subfloat[Een login probleem]{		\includegraphics[width=0.45\textwidth]{images/handleiding/administratie/login-failure}
	\label{fig:handl:admin:login-failure}}
\end{figure}

Het standaard openingsscherm bevat een overzicht van de status van de verschillende componenten, zoals te zien is in figuur \ref{fig:handl:admin:status-overzicht}.
Alsook bevat deze enkele belangrijke statistieken over de backend:
\begin{enumerate_compact}
\item Het aantal ingelogde gebruikers
\item Hoe lang de backend reeds draait
\item Hoeveel requests de backend reeds verwerkt heeft
\end{enumerate_compact}

\begin{figure}[h!]
	\centering
		\includegraphics[width=\textwidth]{images/handleiding/administratie/status-overzicht}
	\caption{Overzicht van de componenten.}
		\label{fig:handl:admin:status-overzicht}
\end{figure}

\subsubsection{Beheer van effecten}
\label{sec:handl:admin:effecten}

Als je op ``Effectenbeheer'' in de linkerkolom klikt verschijnt een overzicht van alle effecten, zoals te zien in figuur \ref{sec:handl:admin:effecten} (het inladen van de effecten duurt een kan even duren).
In het uitschuifmenu bevinden zich enkele filters die je kan activeren door ze aan te klikken.

\begin{figure}[h!]
	\centering
		\includegraphics[width=\textwidth]{images/handleiding/administratie/effecten-overzicht}
	\caption{Overzichtsscherm voor het effectenbeheer.}
		\label{fig:handl:admin:effecten-overzicht}
\end{figure}

Op elk effect kan je de volgende acties uitvoeren:

\begin{itemize_compact}
\item{Naam wijzigen (in de tabel klikken)}
\item{Type wijzigen (in de tabel klikken)}
\item{Zichtbaarheid wijzigen (in de tabel klikken, of de effecten selecteren en de knoppen onderaan gebruiken)}
\item{Geschorst-zijn wijzigen (in de tabel klikken, of de effecten selecteren en de knoppen onderaan gebruiken)}
\end{itemize_compact}

Om de wijzigingen permanent te maken klik je vervolgens op de knop ``Opslaan''.

\subsubsection{Beheer van gebruikers}
\label{sec:handl:admin:gebruikers}

Als je ``Gebruikersbeheer'' in het navigatiemenu aanklikt, verschijnt het gebruikersbeheer zoals in figuur \ref{fig:handl:admin:gebruikers-overzicht}.
Links verschijnt er ook terug een aantal filters die je kan activeren.

\begin{figure}[h!]
	\centering
		\includegraphics[width=\textwidth]{images/handleiding/administratie/gebruikers-overzicht}
	\caption{Het gebruikersbeheer.}
		\label{fig:handl:admin:gebruikers-overzicht}
\end{figure}

Opmerking: In tegenstelling tot de wijzigingen die je uitgevoerd hebt bij het effectenbeheer zijn alle acties hier direct definitief!!

De mogelijk acties zijn:
\begin{itemize_compact}
\item{Aanmaken van een nieuwe gebruiker met ``Registreer gebruiker''.}
\item{Bewerken van een bestaande gebruiker met ``Bewerk gebruiker''.}
\end{itemize_compact}

\paragraph{Gebruiker toevoegen}

Na het klikken op de knop ``Registreer gebruiker'' verschijnt een scherm zoals in figuur \ref{fig:handl:admin:gebruikers-nieuw}. Hierin moet je de velden ''nickname'', ``wachtwoord'', ''achternaam'', ''voornaam'' en ''e-mailadres'' invullen. 
De velden startbedrag, cash en punten worden bij het registeren van de gebruiker in de backend automatisch voor je ingevuld.

\begin{figure}[h!]
	\centering
		\includegraphics[scale=0.75]{images/handleiding/administratie/gebruikers-nieuw}
	\caption{Een nieuwe gebruiker aanmaken.}
		\label{fig:handl:admin:gebruikers-nieuw}
\end{figure}

\paragraph{Gebruiker bewerken}

Bij het bewerken van een gebruiker verschijnt een gelijkaardig venster als dat van ``Registreer gebruiker'' (figuur \ref{fig:handl:admin:gebruikers-edit})
Dit venster verschilt hierin in de volgende punten:

\begin{itemize}
\item{Om het wachtwoord te veranderen, klik je op de knop ``Verander paswoord'', vervolgens verschijnt het wachtwoord veld}
\item{De hoeveelheid cash en punten kan worden gewijzigd. De applicatie vereist echter wel dat dit gelogged wordt, en daarom verschijnt er een nieuw scherm waar je een reden dient op te geven..}
\end{itemize}

\begin{figure}[h!]
	\centering
		\includegraphics[scale=0.75]{images/handleiding/administratie/gebruikers-edit}
	\caption{Details voor een gebruiker bekijken.}
		\label{fig:handl:admin:gebruikers-edit}
\end{figure}

Het wijzigen van de cashpositie wordt gedaan in een apart venster, waarin extra details zoals de ``reden van wijziging'' kunnen ingevuld worden. Zie daarvoor figuur \ref{fig:handl:admin:gebruikers-edit-cash}.

\begin{figure}[h!]
	\centering
		\includegraphics[scale=0.75]{images/handleiding/administratie/gebruikers-edit-cash}
	\caption{Een wijziging in de cashpositie van een gebruiker vastleggen.}
			\label{fig:handl:admin:gebruikers-edit-cash}
\end{figure}

\subsection{Handleiding webapplicatie}

De website is te bereiken op het interne schoolnetwerk op het internetadres ``http://nernst/''.
Buiten het schoolnetwerk kan men een versie van de site bekijken op ``http://www.kapti.com''.

De welkomstpagina verschijnt waarin een korte omschrijving van het spel gegeven wordt. Het navigatiemenu links op de pagina en het informatieblok onderaan worden op elke pagina weergegeven. In dat informatieblok kan je informatie vinden over het pad van de huidige pagina die je bekijkt. De aangemelde gebruiker staat ook in het informatieblok. Als je naar deze overzichtspagina terug wil keren kan je het item 'Home' aanklikken in het navigatiemenu.

Klik je in het navigatiemenu de optie Ranking aan, dan krijg je een overzicht te zien van de tien spelers die op dit moment de hoogste score behaald hebben.
Deze pagina is op het moment van schrijven nog niet afgewerkt.

Je kan de pagina ``Help'' aanklikken om een nog wat extra algemene informatie over het spel en ook de status van het project te bekijken.

De pagina's ``Portfolio'', ``Orders'' en ``Transaction History'' werken enkel maar als een gebruiker aangemeld is.
Om te kunnen aanmelden dien je eerst over een lidmaatschap te beschikken. Een lidmaatschap kan je verkrijgen door in het informatieblok op de verwijzing 'Register' te klikken. Je kan vervolgens een gebruikersnaam kiezen en een wachtwoord opgeven. Na het invullen van de velden klik je ``Registreer'' aan.
Na het doorlopen van deze procedure zal je ook een email ontvangen hebben. Die kan je bewaren ter geheugensteuntje voor je gebruikersnaam.
Nu heb je een persoonlijk lidmaatschap. Om het te activeren kan je inloggen met de gekozen gebruikersnaam en bijhorende wachtwoord.

\subsubsection{Overzicht van effecten - opvragen van details}

Onder ``Securities Overview'' krijg je een overzicht te zien van de effecten die op dit moment in het StockPlay systeem aanwezig zijn. Je kan in het navigatiemenu ook rechtstreeks een beurs aanklikken. Zo krijg je al direct een eerste filtering op de lijst.
Op dit moment gebeurd het ophalen nog wat traag. Dit komt omdat door de abstractielagen nogal wat verbindingen tussen verschillende lagen en servers opgezet moeten worden.
De beurzen onder ``Securities Overview'' zijn die die op dit moment in het spel aanwezig zijn en waarop gescraped wordt.

Klik je op de naam van een effect, dan krijg je detailinformatie te zien over een effect.
Je vind er de huidige prijs, verandering, de beurs waar deze op noteert, de ISIN code, het symbool, de openingsprijs, de maximumprijs van die dag en de laagste prijs van de laagste verhandelingsdag.

De grafiek kent vele mogelijkheden.
Deze is opgebouwd uit twee delen. Het bovenste deel bevat een weergave van het prijsverloop van het gekozen effect. Onderaan zijn bargrafieken te zien van de volumes. Het bovendeel toont standaard het verloop van de voorbije 7 dagen. Hierop kunnen een aantal handelingen verricht worden.
Als je de muisaanwijzer boven dit deel gaat hangen verschijnt het menu.

Met de selectie knop wijzig je de modus van de muis. Standaard staat deze op de verplaats modus.
In deze modus kan je dubbel klikken op een locatie om daarop in te zoomen. Of je kan meer detail krijgen door met de muisaanwijzer het deel aan te wijzen waar je meer detail op wil en het muiswieltje naar boven te verplaatsen. Verplaats het wieltje naar onderen om terug uit te zoomen.
Je kan ook een deel van de grafiek aanklikken en verslepen. Het tekengebied wordt pas opnieuw getekent nadat de linker muisknop losgelaten is.
In de selectie modus is het nog steeds mogelijk om meer of minder detail te verkrijgen door dubbel te klikken en het scrolwieltje te gebruiken.
Alleen ga je in deze modus bij het slepen niet de grafiek verplaatsen, maar een detail venster aanduiden dat je wil bekijken.

Gebruik de knop ``vorige'' om het vorige beeld terug op te roepen. Deze knop kan je oneindig blijven gebruiken tot je terug in de standaardweergave terechtkomt.

Met de knoppen ``kleiner'' en ``groter'' is het opnieuw mogelijk om meer of minder detail op te vragen.

Klik op 'Toevoegen' om een referentie toe te voegen. Op het huidige moment is dit nog een vaste referentie, maar in de toekomst zal het mogelijk zijn deze zelf te kiezen. Een extra lijn wordt bijgetekend. Deze kan je weghalen door in de legende de x aan te klikken rechts van de referentie die je wenst te verwijderen.
Het zal in de toekomst ook mogelijk zijn meerdere referenties toe te voegen.

De ``Herstel'' actie gaat de standaardweergave van de grafiek herstellen. Referenties en weergave geschiedenissen worden terug weggehaald.

Alle datapunten in de grafiek worden dynamisch opgehaald via AJAX technologie na het wijzigen van de view. Het kan dus even duren voor de nieuwe wijzigen zichtbaar worden. Bij het uitzoomen worden minder punten en dus minder detail opgehaald. Het zal nooit mogelijk zijn een tijdsdeel te bekijken waar geen data voor handen is voor ofwel de hoofdgrafiek ofwel voor de referentielijnen. Het maximum uitgezoomde level toont dus alle data die momenteel in het systeem zit.

'Portfolio', 'Orders' en 'Transaction History' bieden overzichten voor de huidige ingelogde gebruiker.

\subsubsection{Aankopen van effecten}

Klik links op ``Securities Overview'' of een van de beurzen die aanwezig is in het spel om een lijst te krijgen van effecten. Klik op de ``Buy'' knop van het gewenste aandeel. Deze knop is ook te vinden op een detailpagina van een effect.

Je kan een prijs een een aantal aandelen opgeven die je wenst te kopen. Als aan de voorwaarden vervolgens voldaan wordt, dan zal de transactionmanager deze op bepaalde tijdstippen omzetten naar een geldige transactie. Hierbij zal het order dan uitgevoerd worden en worden de effecten aan het portfolio toegevoegd.

\subsection{Mobiele client}

\subsubsection{Algemeen}
Bij het opstarten van de applicatie verschijnt er eerst een spashscreen, dat na enkele seconden wordt vervolgd door het loginscherm. Gebruik om in te loggen een gebruikersaccount dat werd aangemaakt via de website of de administratieclient.
\begin{figure}[h!]
	\centering
		\includegraphics[scale=0.75]{images/handleiding/mobile/Login}
	\caption{Het aanmeld-venster.}
	\label{fig:handl:mobile:login}
\end{figure}

Tijdens de werking van het programma, wordt er voor elke request die naar de backend wordt gestuurd een waarschuwingsvenster getoond. Dit is eigen aan de emulator, en op een echte telefoon wordt na een aantal keren de mogelijkheid automatisch de mogelijkheid getoond om altijd een connectie toe te laten.
\begin{figure}[h!]
	\centering
		\includegraphics[scale=0.75]{images/handleiding/mobile/ConnectionWarning}
	\caption{Een waarschuwingsvenster wat betreft de verbinding.}
	\label{fig:handl:mobile:warning}
\end{figure}

Nadat er succesvol kon worden ingelogd, verschijnt een menu met de volgende mogelijkheden:
\begin{itemize_compact}
	\item{View Portfolio}
	\item{View Orders}
	\item{Create order}
\end{itemize_compact}
\begin{figure}[h!]
	\centering
		\includegraphics[scale=0.75]{images/handleiding/mobile/Mainmenu}
	\caption{Het hoofdmenu.}
	\label{fig:handl:mobile:menu}
\end{figure}

\subsubsection{View Portfolio}
\begin{figure}[h!]
	\centering
		\includegraphics[scale=0.75]{images/handleiding/mobile/Portfolio}
	\caption{De portfolio bekijken.}
	\label{fig:handl:mobile:portfolio}
\end{figure}
In dit scherm wordt een eenvoudig overzicht getoond met alle aandelen die zich in het portefolio van de gebruiker bevinden. Van elk effect die de speler in bezit heeft wordt de naam, de huidige waarde, en de relatieve  koersverandering van die dag.

\subsubsection{View Orders}
\begin{figure}[h!]
	\centering
		\includegraphics[scale=0.75]{images/handleiding/mobile/Orders}
	\caption{De orders bekijken.}
	\label{fig:handl:mobile:orders}
\end{figure}
Bij 'View orders' krijg je een overzicht van alle orders die momenteel hangende zijn, alsook van de meest recente afgewerkte orders. Van elk order is het effect, de prijs, de hoeveelheid en de status vermeld.
Onderaan dit venster verschijnt er bij 'Cancel an order' een overzicht van lopende orders, door op een item uit deze lijst te klikken wordt het geannuleerd.

\subsubsection{Create order}
\begin{figure}[h!]
	\centering
		\includegraphics[scale=0.75]{images/handleiding/mobile/CreateOrder}
	\caption{Een order aanmaken.}
	\label{fig:handl:mobile:createorder}
\end{figure}
In dit venster is het mogelijk om enkele eenvoudige orders aan te maken. Daarvoor zijn de naam van het effect, de hoeveelheid, de prijs en het type order (kopen en verkopen, hetzij direct of met een opgegeven limiet) vereiste velden. Nadat alle velden zijn invuld moet er op de linker functietoets worden gedrukt om het order aan te maken.

