%
% Configuratie
%

% Preambule met standaardinstellingen
\documentclass[a4paper,oneside]{report}

% Noot: zorg ervoor dat Nederlandse woordsplitsing geactiveerd is.
\usepackage[english,dutch]{babel}

% Noot: je kan het graphicxpakket een optie dvips of pdftex doorgeven
% in dat geval moet je ze ook aan iiiscriptie doorgeven, dus bijvoorbeeld
% \usepackage[dvips]{graphicx}
% \usepackage[dvips]{iiiscriptie}
\usepackage{graphicx}
\usepackage{iiiscriptie}

% Nuttig pakket voor URL's
\usepackage{url}

% Genereer een index
% gebruik \index{tekst} om een index toe te voegen
\usepackage{makeidx}
\makeindex

% Extra functies
% Verkleinde margin entry
\setlength{\marginparwidth}{1.2in}
\let\oldmarginpar\marginpar
\renewcommand\marginpar[1] {\-\oldmarginpar[\raggedleft\footnotesize #1]%
{\raggedright\footnotesize #1}}

% Een TODO-entry
\newcommand{\todo}[1] {
	\addcontentsline{tdo}{todo}{\protect{#1}}
	\marginpar{#1}
}

% Een lijst van TODO-entries
\makeatletter
\newcommand \listoftodos {
	\section*{Todo list} \@starttoc{tdo}
}
\newcommand\l@todo[2] {
	\par\noindent \textit{#2}, \parbox{10cm}{#1}\par
} \makeatother

% Float defini\"eren voor codefragmenten
\usepackage{float}
\floatstyle{ruled}
\newfloat{code}{thp}{lop}
\floatname{code}{Codefragment}

% Hyperlink maken en URL in footnote tonen
\usepackage{hyperref}
\newcommand{\makeurl}[2]{\href{#2}{#1} \footnote{#2}}

% Functiedefinitie voor protocolstudie
\newcommand{\function}[5] {
	\subsubsection{#1}
	\begin{tabular}{|r p{10cm}|}
	\hline
	\textsc{Gebruik} |		& #2 \\
	\textsc{Parameters} |		& #3 \\
	\textsc{Output} |		& #4 \\
	\textsc{Autorisatie} |		& #5 \\
	\hline
	\end{tabular}
}

% Functiedefinities voor logboek handling
\usepackage{ifthen}
\newcommand{\lbdate}{}
\newcommand{\lbsetdate}[1]{
  \gdef\lbdate{#1}
}
\newcommand{\lbentry}[5] {
	\ifthenelse{\equal{\lbdate}{#1}}
	{
	}
	{
		\ifthenelse{\equal{\lbdate}{}}{}{
			& & & \\ \hline % TODO: workaround, zou niet nodig moeten zijn
			\end{tabular}
		}
		\subsection{#1}
		\begin{tabular}{|r r r p{10cm}|}
		\hline
		\textsc{Begin} & \textsc{Einde} & \textsc{Duur} & \textsc{Beschrijving} \\
		\hline
	}
	\lbsetdate{#1}
	#2 & #3 & #4 & #5 \\
}
\newcommand{\lbstop}[1] {
	& & & \\ \hline % TODO: consistentie fix, zie hierboven
	\end{tabular}
	\lbsetdate{}
	% totaal aantal uren
}

% Compacte enumeraties
\newenvironment{enumerate_compact}{
\begin{enumerate}
  \setlength{\itemsep}{1pt}
  \setlength{\parskip}{0pt}
  \setlength{\parsep}{0pt}
}{\end{enumerate}}
\newenvironment{itemize_compact}{
\begin{itemize}
  \setlength{\itemsep}{1pt}
  \setlength{\parskip}{0pt}
  \setlength{\parsep}{0pt}
}{\end{itemize}}

% Compacte environment voor use-cass
\newenvironment{compact}{\setlength{\parskip}{0pt}}{}

%stijlen voor listings
%
\lstdefinestyle{SQL}{
  breaklines=true,
  language=SQL,
  basicstyle=\normalsize,
  keywordstyle=\ttfamily\color{OliveGreen},
  identifierstyle=\ttfamily\color{CadetBlue}\bfseries,
  commentstyle=\color{Brown},
  stringstyle=\ttfamily,
  showstringspaces=true
}


\definecolor{myid}{rgb}{0.1,0.1,0.1}
\lstdefinestyle{Java}{language=java,
basicstyle=\ttfamily\normalsize,
numbers=left,stepnumber=1,numberstyle=\small\ttfamily,
numbersep=5pt,frame=tlbr,extendedchars=true,
commentstyle=\color{OliveGreen}\ttfamily,
%% stringstyle=\color{red}\ttfamily,
stringstyle=\ttfamily\color{Magenta},
keywordstyle=\ttfamily\color{Violet}\bfseries,
ndkeywordstyle=\ttfamily\color{Yellow}\bfseries,
identifierstyle=\ttfamily\color{myid},
% sensitive=false,
basicstyle=\scriptsize,
}



%
% Titelpagina
%

% Invullen velden
\departement{Departement Toegepaste Ingenieurswetenschappen}
\deptadres{Schoonmeersstraat 52 - 9000 Gent}
\studiejaar{3e Bachelor Informatica}
\soortrapport{
Analyse voor het vakoverschrijdend project Informatica
}
\title{Verslag 'StockPlay'}
\author{
Tim BESARD\\
Dieter DEFORCE\\
Laurens VAN ACKER\\
Thijs WALCARIUS
}

% Pagina maken
\begin{document}
\maketitle
\pagenumbering{roman}
\tableofcontents
\pagenumbering{arabic}


%
% Inhoud
%

% Inleiding
\chapter*{Inleiding}
StockPlay is een spel waarmee beleggen op de beurs wordt gesimuleerd. De deelnemers kunnen aandelen, opties, trackers, fondsen, enz. verhandelen (later "effecten" genoemd). De bedoeling van het spel is om een zo goed mogelijke prestatie neer te zetten: zowel de continue prestaties, alsook het rendement op het einde van het spel zijn belangrijk.

% Analyse
\part{Analyse}
%Preambule met standaardinstellingen
\documentclass[a4paper,oneside]{report}
%Noot: zorg ervoor dat Nederlandse woordsplitsing geactiveerd is.
\usepackage[dutch]{babel}
% Noot: je kan het graphicxpakket een optie dvips of pdftex doorgeven
% in dat geval oet je ze ook aan iiiscriptie doorgeven, dus bijvoorbeeld
% \usepackage[dvips]{graphicx}
% \usepackage[dvips]{iiiscriptie}
\usepackage{graphicx}
\usepackage{iiiscriptie}
%Nuttig pakket voor URL's
\usepackage{url}
%extra functies door Tim
% Verkleinde margin entry
\setlength{\marginparwidth}{1.2in}
\let\oldmarginpar\marginpar
\renewcommand\marginpar[1] {\-\oldmarginpar[\raggedleft\footnotesize #1]%
{\raggedright\footnotesize #1}}

% Een TODO-entry
\newcommand{\todo}[1] {
	\addcontentsline{tdo}{todo}{\protect{#1}}
	\marginpar{#1}
}

% Een lijst van TODO-entries
\makeatletter
\newcommand \listoftodos {
	\section*{Todo list} \@starttoc{tdo}
}
\newcommand\l@todo[2] {
	\par\noindent \textit{#2}, \parbox{10cm}{#1}\par
} \makeatother

% Float defini\"eren voor codefragmenten
\usepackage{float}
\floatstyle{ruled}
\newfloat{code}{thp}{lop}
\floatname{code}{Codefragment}

% Hyperlink maken en URL in footnote tonen
\usepackage{hyperref}
\newcommand{\makeurl}[2]{\href{#2}{#1} \footnote{#2}}

% Functiedefinitie voor protocolstudie
\newcommand{\function}[5] {
	\subsubsection{#1}
	\begin{tabular}{|r p{10cm}|}
	\hline
	\textsc{Gebruik} |		& #2 \\
	\textsc{Parameters} |		& #3 \\
	\textsc{Output} |		& #4 \\
	\textsc{Autorisatie} |		& #5 \\
	\hline
	\end{tabular}
}

% Functiedefinities voor logboek handling
\usepackage{ifthen}
\newcommand{\lbdate}{}
\newcommand{\lbsetdate}[1]{
  \gdef\lbdate{#1}
}
\newcommand{\lbentry}[5] {
	\ifthenelse{\equal{\lbdate}{#1}}
	{
	}
	{
		\ifthenelse{\equal{\lbdate}{}}{}{
			& & & \\ \hline % TODO: workaround, zou niet nodig moeten zijn
			\end{tabular}
		}
		\subsection{#1}
		\begin{tabular}{|r r r p{10cm}|}
		\hline
		\textsc{Begin} & \textsc{Einde} & \textsc{Duur} & \textsc{Beschrijving} \\
		\hline
	}
	\lbsetdate{#1}
	#2 & #3 & #4 & #5 \\
}
\newcommand{\lbstop}[1] {
	& & & \\ \hline % TODO: consistentie fix, zie hierboven
	\end{tabular}
	\lbsetdate{}
	% totaal aantal uren
}

% Compacte enumeraties
\newenvironment{enumerate_compact}{
\begin{enumerate}
  \setlength{\itemsep}{1pt}
  \setlength{\parskip}{0pt}
  \setlength{\parsep}{0pt}
}{\end{enumerate}}
\newenvironment{itemize_compact}{
\begin{itemize}
  \setlength{\itemsep}{1pt}
  \setlength{\parskip}{0pt}
  \setlength{\parsep}{0pt}
}{\end{itemize}}

% Compacte environment voor use-cass
\newenvironment{compact}{\setlength{\parskip}{0pt}}{}

%stijlen voor listings
%
\lstdefinestyle{SQL}{
  breaklines=true,
  language=SQL,
  basicstyle=\normalsize,
  keywordstyle=\ttfamily\color{OliveGreen},
  identifierstyle=\ttfamily\color{CadetBlue}\bfseries,
  commentstyle=\color{Brown},
  stringstyle=\ttfamily,
  showstringspaces=true
}


\definecolor{myid}{rgb}{0.1,0.1,0.1}
\lstdefinestyle{Java}{language=java,
basicstyle=\ttfamily\normalsize,
numbers=left,stepnumber=1,numberstyle=\small\ttfamily,
numbersep=5pt,frame=tlbr,extendedchars=true,
commentstyle=\color{OliveGreen}\ttfamily,
%% stringstyle=\color{red}\ttfamily,
stringstyle=\ttfamily\color{Magenta},
keywordstyle=\ttfamily\color{Violet}\bfseries,
ndkeywordstyle=\ttfamily\color{Yellow}\bfseries,
identifierstyle=\ttfamily\color{myid},
% sensitive=false,
basicstyle=\scriptsize,
}

%
%Invullen velden voor titelpagina.
%
\departement{Departement Toegepaste Ingenieurswetenschappen}
\deptadres{Schoonmeersstraat 52 - 9000 Gent}
\studiejaar{3e Bachelor Informatica}
\soortrapport{
Analyse voor het vakoverschrijdend project Informatica
}
\title{Analyse 'Stockplay'}
\author{
Tim BESARD\\
Dieter DEFORCE\\
Laurens VAN ACKER\\
Thijs WALCARIUS
}
\begin{document}
\maketitle
\pagenumbering{roman}
\tableofcontents
\pagenumbering{arabic}
% Behoeften-analyse
\chapter{Behoefteanalyse}
Bij dit spel is het de bedoeling dat deelnemers tijdens de duur van het spel (dat loopt over enkele weken/maanden) virtueel aandelen, opties, trackers, fondsen, enz. kunnen verhandelen om zo op het einde van het spel een zo hoog mogelijk rendement neer te zetten. Hiervoor krijgt iedereen aan het begin van het spel eenzelfde budget toegewezen. 

Voorafgaand aan de start van het spel is een inschrijvingsperiode. Hierbij spelers kunnen zich registeren met behulp van hun eID. 

Het spel start op een vastgestelde dag en tijdstip: 1 maart 2010 om 8u 's morgens. Elke speler krijgt een startbudget van €100.000.
\todo{Waarom fixed registratieperiode?}
Vanaf dat moment hebben de gebruikers toegang tot hun portefolio en kunnen effecten worden verhandeld.

De gebruiker kan effecten aankopen van verschillende bronnen:
\begin{itemize}
\item{aandelen op de Continumarkt van Brussel}
\item{aandelen op de Eurolist van Parijs}
\item{aandelen op de "Lokaal" van Amsterdam}
\item{trackers op Euronext Amsterdam}
\end{itemize}

Het spel eindigt op 31 mei 2010, waarna een eindklassement wordt opgemaakt. De koersen op de website worden niet meer ge\"updatet, en er kunnen ook geen transacties meer gedaan worden. Gebruikers kunnen wel nog steeds inloggen op de website om statistieken te raadplegen.


% Functionele analyse
\chapter{Functionele analyse}
\subsection{Webapplicatie}

Deze interface wordt gebruik om deel te nemen aan het spel. Een gebruiker surft hierbij naar de server die de interface host, en krijgt direct de spelomgeving te zien, dit zonder eerst aan bepaalde softwarevereisten (zoals een Java runtime) te moeten voldaan hebben. Een deel van de functionaliteit is beperkt tot geregistreerde gebruikers, maar hier meer over later.

Eerst en vooral is er de algemene overzichtspagina. Die geeft de gebruiker een zicht over:
\begin{itemize}
\item{de huidige waarde van de aangekochte effecten en de meer/minwaarde op de aankoopprijs}
\item{cashpositie}
\item{totaal van cashpositie + huidige waarde van de effecten}
\item{huidig rendement}
\item{grafiek met een overzicht van hun rendement tov het gemiddeld rendement, beste rendement, enz}
\item{algemeen klassement, tussenklassementen}
\end{itemize}

Vervolgens kan de gebruiker zijn portefolio bekijken, en daar volgende informatie uit halen:
\begin{itemize}
\item{overzicht van het portefolio van de gebruiker}
\item{effecten momenteel in bezit, aankoopprijs (per stuk en in totaal), huidige koers, rendement en winst/verlies op het aandeel}
\end{itemize}

Er is ook een pagina die een zicht biedt op alle effecten aanwezig in het spel. Om het overzicht te behouden voorziet dat overzicht in verschillende filters:
\begin{itemize}
\item{Per beurs}
\item{Per type (aandeel, tracker, ...)}
\item{Per index}
\item{Per naam}
\item{Aandelen die als favoriet zijn gekenmerkt door de gebruiker}
\item{Per prijs}
\item{Per volume}
\end{itemize}

Per aandeel kan vervolgens doorgeklikt worden naar een overzichtspagina, die volgende informatie biedt:
\begin{itemize}
\item{grafiek met koers}
\item{hoog/laag van de dag}
\item{huidige koers}
\item{openingskoers}
\item{verschil}
\item{omzet}
\item{mogelijkheid om te kopen/te verkopen}
\end{itemize}

De gebruiker kan ook een overzicht bovenhalen waarop zijn transactiegeschiedenis zichtbaar is. Daarbij krijgt hij per transactie het volgende te zien:
\begin{itemize}
\item{het tijdstip}
\item{het effect}
\item{het type transactie}
\item{het aantal}
\item{de kostprijs voor de transactie}
\item{de winst/verlies door transactie}
\item{totaal van uw portefeuille}
\end{itemize}
Ook hier kan men steeds doorklikken naar een detailpagina in kwestie, en ook het overzicht behouden met behulp van volgende filters:
\begin{itemize}
\item{enkel aankopen/verkopen}
\item{enkel met winst/verlies}
\end{itemize}

Er zijn ook verschillende klassementen aanwezig:
\begin{itemize}
\item{Meest aangekochte aandelen}
\item{Meest verkochte aandelen}
\item{Spelers top}
\end{itemize}

De interface voorziet ook in een pagina om aandelen te kopen of verkopen. Hiervoor zijn verschillende mogelijkheden:
\begin{itemize}
\item{de prijs die je biedt voor het aandeel. Pas als het aandeel die koers bereikt wordt het aandeel effectief aangekocht}
\item{de hoeveelheid aandelen die je wenst aan te kopen}
\item{de max. geldigheidsduur van dit order (1 uur/dag/week/maand)}
\end{itemize}
Tijdens de aankoop krijgt de gebruiker een overzicht van de totale kostprijs en de verschillende taksen die erop staan.


\subsection{Desktopapplicatie}

Deze applicatie voorziet in het beheer van het hele systeem. De beheerder logt daarvoor in met behulp van zijn eID.

De desktopapplicatie is opgesplitst in drie grote componenten.
Enerzijds is er het overzicht van de gebruikers. Per gebruiker zijn de volgende beheersopdrachten mogelijk: 
\begin{itemize}
\item{wijzigen van}
\item{verwijderen van gebruiker}
\item{aanpassen van portefolio: kopen/verkopen van effecten}
\item{de hoeveelheid cash van de gebruiker aanpassen}
\end{itemize}

Er is ook een overzicht van de effecten voorzien, die de volgende functionaliteit biedt:
\begin{itemize}
\item{Aanduiden van welke effecten moeten gescraped worden, welke effecten zichtbaar zijn bij de spelers}
\item{Schorsen van handel in een effect}
\item{Wijzigen van de gescrapede gegevens}
\item{Overzicht van de aanwezigheid in portefeuilles bij spelers}
\end{itemize}

Het laatste grote deel van de applicatie biedt een overzicht van de systeemstatus. Daarbij kan de beheerder het volgende ondernemen:
\begin{itemize}
\item{Status van de componenten (scraper, database, website, ...) en starten/stoppen/herstarten van degene die dit aankunnen.}
\item{Statistieken bekijken}
	\begin{itemize}
	\item{Aantal ingeschreven gebruikers sinds de start}
	\item{Aantal gebruikers online}
	\item{Aantal connecties per tijdseenheid op de backend}
	\item{Aantal transacties per tijdseenheid}
	\item{Aantal (succesvol/gefaalde/...) gescrapede aandelen per tijdseenheid}
	\end{itemize}
\end{itemize}

\subsection{Backend}

De backend fungeert als een schil rond de database. Alle aanvragen van de desktopapplicatie en de website die informatie uit de database nodig hebben, of er naartoe willen schrijven worden afgeleid naar de backend. De communicatie tussen backend en zijn clients gebeurt mbv XML-RPC. 

\subsection{Scrapers}

De scrapers gaan periodiek enkele vooraf bepaalde sites ophalen en halen hieruit de huidige koersinformatie. Hiervoor wordt oa. handig gebruik gemaakt van de AJAX-requests die deze websites gebruiken om de koersen live te tonen. Dit zorgt ervoor dat bijna enkel de koersen worden opgehaald, zonder overhead (zoals de layout van de website, etc) 
De opgehaalde informatie wordt vervolgens doorgestuurd naar de backend dewelke deze koersinformatie vervolgens opslaat in de database. 
Optioneel: mobiele client voor PDA/Smartphone 
Spelers kunnen met behulp van hun PDA/Smartphone een gereduceerde set acties uitvoeren op hun portefolio.


\end{document}



% Ontwerp
\part{Ontwerp}
%
% Functioneel
%

\chapter{Functioneel ontwerp}

\section{Webapplicatie}

\todo{Sommige opsommingen herwerken tot tekst?}
Deze interface wordt gebruik om deel te nemen aan het spel. Een gebruiker surft hierbij naar de server die de interface host, en krijgt zo de spelomgeving te zien, dit zonder eerst aan bepaalde softwarevereisten (zoals een Java runtime) te moeten voldaan hebben. Een deel van de functionaliteit is beperkt tot geregistreerde gebruikers, maar meer hierover in hoofdstuk X.

\paragraph{De algemene overzichtspagina}
Deze pagina geeft de gebruiker een zicht over:
\begin{itemize}
	\item{de huidige waarde van de aangekochte effecten en de meer- of minwaarde op de aankoopprijs}
	\item{zijn cashpositie}
	\item{het totaal van zijn cashpositie en de huidige waarde van de effecten}
	\item{het huidige rendement}
	\item{een grafiek met een overzicht van hun rendement ten opzichte van het gemiddeld rendement, beste rendement, enz.}
	\item{het algemeen klassement en eventuele tussenklassementen}
\end{itemize}

\paragraph{Portefolio}
Vervolgens kan de gebruiker zijn portefolio bekijken, en daar volgende informatie uit halen:
\begin{itemize}
	\item{een overzicht van het portefolio van de gebruiker}
	\item{de effecten momenteel in bezit, aankoopprijs (per stuk en in totaal), huidige koers, rendement en winst/verlies op het aandeel}
\end{itemize}

\paragraph{Overzicht beschikbare effecten}
Er is ook een pagina die een zicht biedt op alle effecten aanwezig in het spel.
\subparagraph{Filters} Om het overzicht te behouden voorziet dat overzicht in verschillende filters:
\begin{itemize}
  \setlength{\itemsep}{1pt}
  \setlength{\parskip}{0pt}
  \setlength{\parsep}{0pt}
	\item{per beurs}
	\item{per type (aandeel, tracker, ...)}
	\item{per index}
	\item{per naam}
	\item{aandelen die als favoriet zijn gekenmerkt door de gebruiker}
	\item{per prijs}
	\item{per volume}
\end{itemize}
\subparagraph{Details per effect}Per aandeel kan vervolgens doorgeklikt worden naar een overzichtspagina, die de volgende informatie biedt:
\begin{itemize}
  \setlength{\itemsep}{1pt}
  \setlength{\parskip}{0pt}
  \setlength{\parsep}{0pt}
	\item{grafiek met koers}
	\item{hoog/laag van de dag}
	\item{huidige koers}
	\item{openingskoers}
	\item{verschil}
	\item{omzet}
	\item{mogelijkheid om te kopen/te verkopen}
\end{itemize}

\paragraph{Transactiegeschiedenis}De gebruiker kan ook een overzicht bovenhalen waarop zijn transactiegeschiedenis zichtbaar is. 
\subparagraph{Details per transactie}Ook hier kan men steeds doorklikken naar een detailpagina in kwestie:
\begin{itemize}
  \setlength{\itemsep}{1pt}
  \setlength{\parskip}{0pt}
  \setlength{\parsep}{0pt}
	\item{het tijdstip}
	\item{het effect}
	\item{het type transactie}
	\item{het aantal}
	\item{de kostprijs voor de transactie}
	\item{de winst/verlies door transactie}
	\item{totaal van uw portefeuille}
\end{itemize}
\subparagraph{Filters}Het overzicht kan worden behouden met behulp van volgende filters:
\begin{itemize}
  \setlength{\itemsep}{1pt}
  \setlength{\parskip}{0pt}
  \setlength{\parsep}{0pt}
	\item{enkel aankopen/verkopen}
	\item{enkel met winst/verlies}
\end{itemize}

\paragraph{Klassementen}Er zijn ook verschillende klassementen aanwezig:
\begin{itemize}
  \setlength{\itemsep}{1pt}
  \setlength{\parskip}{0pt}
  \setlength{\parsep}{0pt}
	\item{meest aangekochte aandelen}
	\item{meest verkochte aandelen}
	\item{spelers top 20 (waarde portefolio)}
	\item{spelers top 20 (puntentotaal)}
\end{itemize}

\paragraph{Verhandelpagina effecten}De interface voorziet ook in een pagina om aandelen te kopen of verkopen. Hiervoor zijn verschillende mogelijkheden:
\begin{itemize}
  \setlength{\itemsep}{1pt}
  \setlength{\parskip}{0pt}
  \setlength{\parsep}{0pt}
	\item{de prijs die je biedt voor het aandeel. Pas als het aandeel die koers bereikt wordt het aandeel effectief aangekocht}
	\item{de hoeveelheid aandelen die je wenst aan te kopen}
	\item{de max. geldigheidsduur van dit order (1 uur/dag/week/maand)}
\end{itemize}
Tijdens de aankoop krijgt de gebruiker een overzicht van de totale kostprijs en de verschillende taksen die erop staan.

\section{Desktopapplicatie}
Deze applicatie voorziet in het beheer van het hele systeem. De beheerder logt daarvoor in met behulp van zijn eID.

De desktopapplicatie is opgesplitst in drie grote componenten.
\paragraph{Gebruikersbeheer}Enerzijds is er het overzicht van de gebruikers. Per gebruiker zijn de volgende beheersopdrachten mogelijk: 
\begin{itemize}
	\item{wijzigen van een gebruiker}
	\item{verwijderen van een gebruiker}
	\item{aanpassen van portefolio: kopen/verkopen van effecten}
	\item{de hoeveelheid cash van de gebruiker aanpassen}
\end{itemize}

\paragraph{Effectenbeheer}Er is ook een overzicht van de effecten voorzien, die de volgende functionaliteit biedt:
\begin{itemize}
	\item{aanduiden van welke effecten moeten gescraped worden, welke effecten zichtbaar zijn bij de spelers}
	\item{schorsen van handel in een effect}
	\item{wijzigen van de gescrapede gegevens}
	\item{overzicht van de aanwezigheid in portefeuilles bij spelers}
\end{itemize}

\paragraph{Systeemstatus}Het laatste grote deel van de applicatie biedt een overzicht van de systeemstatus. Daarbij kan de beheerder het volgende ondernemen:
\begin{itemize}
\item{status van de componenten (scraper, database, website, ...) en starten/stoppen/herstarten van degene die dit aankunnen.}
\item{statistieken bekijken}
	\begin{itemize}
	\item{aantal ingeschreven gebruikers sinds de start}
	\item{aantal gebruikers online}
	\item{aantal connecties per tijdseenheid op de backend}
	\item{aantal transacties per tijdseenheid}
	\item{aantal (succesvol/gefaalde/...) gescrapede aandelen per tijdseenheid}
	\end{itemize}
\end{itemize}

\section{Backend}

De backend fungeert als een schil rond de database. Alle aanvragen van de desktopapplicatie en de website die informatie uit de database nodig hebben, of er naartoe willen schrijven worden afgeleid naar de backend. De communicatie tussen backend en zijn clients gebeurt mbv XML-RPC. 

\section{Scrapers}

De scrapers gaan periodiek enkele vooraf bepaalde sites ophalen en halen hieruit de huidige koersinformatie. Hiervoor wordt oa. handig gebruik gemaakt van de AJAX-requests die deze websites gebruiken om de koersen live te tonen. Dit zorgt ervoor dat bijna enkel de koersen worden opgehaald, zonder overhead (zoals de layout van de website, etc) 
De opgehaalde informatie wordt vervolgens doorgestuurd naar de backend dewelke deze koersinformatie vervolgens opslaat in de database. 
Optioneel: mobiele client voor PDA/Smartphone 
Spelers kunnen met behulp van hun PDA/Smartphone een gereduceerde set acties uitvoeren op hun portefolio.


%
% Technisch
%

\chapter{Technisch ontwerp}

\section{Hergebruik}

\todo{Beschrijving van het wat en waarom van libraries}

\section{Hardware}

\todo{Een vermelding van de PDA-interface}

% Realisatie
\part{Realisatie}
\chapter{Dataontwerp}

\begin{figure}[h!]
	\centering
		\includegraphics[width=0.5\textwidth]{images/realisatie/ER_Diagram}
	\caption{Entity-relationship model.}
\end{figure}

\section{Ontwerp backend}
\todo{dit hoort hier niet}
De klassen die de persistente data uit de database moeten weergeven staan opgesomd in onderstaand klassendiagram:

\begin{figure}[h!]
	\centering
		\includegraphics[width=0.5\textwidth]{images/realisatie/Class_Diagram}
	\caption{Klassendiagram persistente data in backend.}
\end{figure}


\chapter{Procedureontwerp}

\section{Backend protocol}

Zoals vermeld in de behoeftenanalyse wordt alle databasetoegang uitgevoerd via een gemeenschappelijke backend. De interface hiervoor stelt echter een aantal bijzondere eisen:
\begin{itemize}
\item{Taalonafhankelijk: aangezien de interfaces met behulp van verschillende programmeertalen gerealiseerd worden, moet de interface toegankelijk zijn vanuit een zo wijd mogelijke waaier aan programmeertalen.}
\item{Lichtgewicht: als een mobiele interface een mogelijke uitbreiding kan zijn, moet het gekozen protocol compact zijn en mogen de eventueel benodigde libraries niet te zwaar zijn.}
\item{Toegankelijk: omdat interfaces niet noodzakelijk uitgevoerd worden op hetzelfde systeem van de backend, is het mooi meegenomen als het protocol geen probleem vormt in gelimiteerde omgevingen.}
\end{itemize}

Na verschillende kandidaten overwogen te hebben, hebben we gekozen voor XML-RPC. Dit is een lichtgewicht Remote Procedure protocol, dewelke methodeaanvragen en -antwoorden verpakt in XML-data en ze als POST request verstuurd over het HTTP protocol (zie bijlage \ref{chap:xml-rpc} voor de exacte specificatie).

Het protocol voldoet aan de opgestelde eisen: aangezien het verderbouwt op het bestaande HTTP-protocol kan het gebruik maken van diens mogelijkheden (zoals compressie en encryptie), en kan het indien een specifieke bibliotheek onbestaande is eenvoudig verwerkt worden via reeds bestaande HTTP- en XML-bibliotheken. Bovendien verschilt de communicatie niet van regulier browsen waardoor de toegankelijkheid in gelimiteerde omgevingen ook toeneemt.

Voor de programmeertalen die we gaan gebruiken bij het implementeren blijken er reeds verschillende bibliotheken beschikbaar te zijn, wat het gemak van gebruik opnieuw verhoogt. Hierbij een opsomming van de specifieke bibliotheken die we zullen gebruiken om informatie te versturen en ontvangen over het XML-RPC protocol:
\begin{itemize}
\item{\textbf{Perl}: \makeurl{XML::RPC}{http://search.cpan.org/~daan/XML-RPC-0.9/lib/XML/RPC.pm}}
\item{\textbf{C\#}: \makeurl{XML-RPC.NET}{http://www.xml-rpc.net/}}
\item{\textbf{Java}: \makeurl{Apache XML-RPC}{http://ws.apache.org/xmlrpc/}}
\end{itemize}

Aangezien XML-RPC geen ondersteuning biedt voor namespaces of andere vormen van functieorganisatie, hanteren we zelf een mechanisme om dit te bekomen: een methode-naam bestaat altijd uit twee delen, gescheiden door een punt. Het deel voor het scheidingsteken duidt het pakket aan, het deel erna de specifieke methode.
Zo delen we de backend op in volgende primaire klassen:
\begin{itemize}
\item{System: functionaliteit voor beheer van het systeem.}
\item{User: beheer van gebruikers, ook voor gebruikers zelf.}
\item{Finance: functionaliteit gerelateerd met het beurswezen.}
\end{itemize}
Hogere-orde klassen zijn eventueel ook mogelijk (zoals \emph{System.Database}), maar niet verplicht. De semantiek is daarbij identiek aan primaire klassen, met een punt als scheidingsteken.

\subsection{Algemene foutcodes}

De XML-RPC specificatie biedt ondersteuning voor foutberichten, in de form van een bericht met een $<$fault$>$ tag. Die tag moet steeds twee $<$member$>$ tags bevatten, namelijk een foutcode $<$faultCode$>$ van het integer type, en een foutbericht $<$faultString$>$ van het string type. Elk van de klassen kan zo specifieke foutmeldingen vastleggen.
Maar er zijn ook generieke foutmeldingen, die van toepassing zijn op alle klassen. Deze foutmeldingen, waarvan de foutcode in het bereik $[0, 100[$ valt, worden hieronder beschreven:

\begin{table}
\begin{tabular}{| c p{5cm} p{7cm} |}
	\hline
	Foutcode & Foutbericht & Controle \\
	\hline
	
	0 & Version Not Supported & De client gebruikt een verkeerd communicatieprotocol. \\
	\hline
	
	$[1-10[$ & \emph{Subsystem failures.} & \\
	1 & Internal Failure & Verifieer de status van de backend, de log kan hierbij helpen. \\
	2 & Database Failure & Er is een probleem met de database (onbeschikbaar, corrupt, ...), zie de log voor meer details. \\
	\hline
	
	$[10-20[$ & \emph{Service issues.} & \\
	10 & Service Unavailable & De backend kan tijdelijk niet gebruikt worden (werkzaamheden, overloaded, ...). \\
	11 & Unauthorized & Meld u aan vooraleer deze functie te gebruiken. \\
	\hline
	
	$[20-30[$ & \emph{Method issues.} & \\
	20 & Not Found & Methode niet gevonden, verifieer de schrijfwijze en de klasse. \\
	21 & Bad Request & Probleem met de parameters, controleer het gebruik van de methode. \\
	\hline
\end{tabular}
\caption{Generieke foutcodes in het backend-protocol.}
\end{table}

\subsection{Authenticatie en autorisatie}

Aangezien het XML-RPC protocol gebruik maakt van het HTTP-protocol, kunnen we diens functionaliteit gebruiken om authenticatie te bekomen. Daartoe zullen we gebruik maken van \emph{basic authentication}, waarbij de client indien gevraagt een gebruikersnaam en wachtwoord naar de server doorstuurd.
\todo{Dit is tijdelijk}

Afhankelijk van de capabiliteiten van het XML-RPC pakket dat we in de backend gebruiken (Apache XML-RPC), kan dit op twee manieren verlopen. Indien de bibliotheek ondersteuning biedt voor het on-demand inschakelen van authenticatie gebaseerd op de ontvangen request, kunnen we zo wanneer benodigd \emph{basic authentication} inschakelen en de webserver zelf een HTTP-401 laten terugsturen, zonder hiervoor extra code in de backend benodigd is.
Als deze optie niet dynamisch ingeschakeld kan worden, zullen we zelf een eigen foutmelding moeten terugsturen die aanduidt dat authorisatie benodigd is. Als de client zo een $<$fault$>$-bericht ontvangt, zal die een nieuwe XML-RPC socket openen op een alternatieve URL (bijvoorbeeld \texttt{http://server.hogent.be/authenticated}). Aangezien de URL nu verschillend is, kunnen we de webserver in de backend zodanig configureren dat authenticatie vereist is voor die zone. Zo bekomen we eveneens verplichte authenticatie voor bepaalde methodes, maar dit door ze enkel beschikbaar te stellen in een subset van het serverdomein. Dit vereist enige extra code in de backend.

Autorisatie tenslotte is beperkt: in de huidige opzet ondersteunen we geen flexibel-toegekende rechten, enkel een bit die bepaalt of de gebruiker een administrator is of niet.

\todo{gebruikers-autorisatie ook via dit systeem, return rechten bitmap / ID}

\subsection{System-klasse}

Deze klasse biedt de client mogelijkheden om het systeem te beheren, met name ophalen van informatie, wijzigen van configuraties, en (her)starten of stoppen van bepaalde subsystemen.


\begin{itemize}
\item{\textbf{Gebruik}: }
\item{\textbf{Parameters}:}
	\begin{itemize}
	\item{}
	\end{itemize}
\item{\textbf{Output}:}
	\begin{itemize}
	\item{}
	\end{itemize}
\item{\textbf{Autorisatie}: }
\end{itemize}

\paragraph{Backend.Status}
\begin{itemize}
\item{\textbf{Gebruik}: status van de backend ophalen.}
\item{\textbf{Parameters}: geen.}
\item{\textbf{Output}: integer die status beschrijft:}
	\begin{itemize}
	\item{0: maintenance-mode}
	\item{1: de backend werkt}
	\end{itemize}
\item{\textbf{Autorisatie}: administrator-rechten benodigd}
\end{itemize}

\paragraph{Backend.Stats}
\begin{itemize}
\item{\textbf{Gebruik}: backend-statistieken ophalen.}
\item{\textbf{Parameters}: geen.}
\item{\textbf{Output}: struct met statistieken}
	\begin{itemize}
	\item{users: aantal gebruikers online}
	\item{req: aantal verwerkte XML-RPC requests}
	\item{uptime: hoe lang de backend al draait}
	\end{itemize}
\item{\textbf{Autorisatie}: administrator-rechten benodigd.}
\end{itemize}

\paragraph{Backend.Restart}
\begin{itemize}
\item{\textbf{Gebruik}: de backend herstarten.}
\item{\textbf{Parameters}: geen.}
\item{\textbf{Output}: bool die indiceert of de actie succesvol was.}
\item{\textbf{Autorisatie}: administrator-rechten benodigd.}
\end{itemize}

\paragraph{Backend.Stop}
\begin{itemize}
\item{\textbf{Gebruik}: de backend permanent stilleggen.}
\item{\textbf{Parameters}: geen.}
\item{\textbf{Output}: bool die indiceert of de actie succesvol was.}
\item{\textbf{Autorisatie}: administrator-rechten benodigd.}
\end{itemize}

\paragraph{Database.Status}
\begin{itemize}
\item{\textbf{Gebruik}: status van de database ophalen.}
\item{\textbf{Parameters}: geen.}
\item{\textbf{Output}: integer die status beschrijft:}
	\begin{itemize}
	\item{0: database niet te bereiken}
	\item{1: de backend werkt}
	\end{itemize}
\item{\textbf{Autorisatie}: administrator-rechten benodigd}
\end{itemize}

\paragraph{Database.Stats}
\begin{itemize}
\item{\textbf{Gebruik}: database-statistieken ophalen. Dit zijn de statistieken geleverd door de database zelf, ze zijn dus niet beperkt tot acties ondernomen door de backend.}
\item{\textbf{Parameters}: geen.}
\item{\textbf{Output}: struct met statistieken}
	\begin{itemize}
	\item{queries: aantal uitgevoerde queries}
	\item{uptime: hoe lang de database al draait}
	\item{traffic: hoeveelheid data verzonden en ontvangen}
	\item{slow\_queries: aantal queries die teveel uitvoeringstijd vergden}
	\end{itemize}
\item{\textbf{Autorisatie}: administrator-rechten benodigd.}
\end{itemize}

\paragraph{Scraper.Status}
\begin{itemize}
\item{\textbf{Gebruik}: status van de scraper ophalen.}
\item{\textbf{Parameters}: geen.}
\item{\textbf{Output}: integer die status beschrijft:}
	\begin{itemize}
	\item{0: niet geactiveerd}
	\item{1: inactief}
	\item{2: bezig met data-mining}
	\end{itemize}
\item{\textbf{Autorisatie}: administrator-rechten benodigd}
\end{itemize}

\paragraph{Scraper.Stats}
\begin{itemize}
\item{\textbf{Gebruik}: scraper-statistieken ophalen.}
\item{\textbf{Parameters}: geen.}
\item{\textbf{Output}: struct met statistieken}
	\begin{itemize}
	\item{executes: aantal voltooide plugin-uitvoeringen}
	\item{uptime: hoe lang de scraper al draait}
	\item{traffic: hoeveelheid data verzonden en ontvangen}
	\item{plugins: aantal geactiveerde plugins}
	\item{securities: aantal beschikbare effecten}
	\item{exchanges: aantal beschikbare beurzen}
	\item{indexes: aantal beschikbare indexen}
	\item{delay: tijd tot de volgende plugin-uitvoering}
	\item{memory: geheugengebruik}
	\end{itemize}
\item{\textbf{Autorisatie}: administrator-rechten benodigd.}
\end{itemize}

\paragraph{Scraper.Restart}
\begin{itemize}
\item{\textbf{Gebruik}: de scraper herstarten.}
\item{\textbf{Parameters}: geen.}
\item{\textbf{Output}: bool die indiceert of de actie succesvol was.}
\item{\textbf{Autorisatie}: administrator-rechten benodigd.}
\end{itemize}

\paragraph{Scraper.Stop}
\begin{itemize}
\item{\textbf{Gebruik}: de scraper permanent stilleggen.}
\item{\textbf{Parameters}: geen.}
\item{\textbf{Output}: bool die indiceert of de actie succesvol was.}
\item{\textbf{Autorisatie}: administrator-rechten benodigd.}
\end{itemize}


\subsection{User-klasse}

Hier vindt men de nodige methodes terug om gebruikers te beheren. Dit is echter niet beperkt tot de administrator: ook gebruikers zelf kunnen hun eigen profiel (in beperktere mate) beheren.

\paragraph{List}

\begin{itemize}
\item{\textbf{Gebruik}: lijst met publieke informatie opvragen van de gebruikers}
\item{\textbf{Parameters}: een filter}
\item{\textbf{Output}: een lijst met structs}
	\begin{itemize}
	\item{id}
	\item{nickname}
	\item{regdate}
	\item{points}
	\end{itemize}
\item{\textbf{Autorisatie}: geen benodigd}
\end{itemize}

\paragraph{Details}

\begin{itemize}
\item{\textbf{Gebruik}: lijst met publieke informatie opvragen van de gebruikers}
\item{\textbf{Parameters}: een lijst met id's}
\item{\textbf{Output}: een lijst met structs}
	\begin{itemize}
	\item{id}
	\item{firstName}
	\item{lastName}
	\item{rrn}
	\item{cash}
	\item{startkapitaal}
	\end{itemize}
\item{\textbf{Autorisatie}: Gebruiker zelf of een gebruiker met administratorrechten}
\end{itemize}

\paragraph{Create}

\begin{itemize}
\item{\textbf{Gebruik}: aanmaken van een nieuwe gebruiker}
\item{\textbf{Parameters}: struct met de gebruikers}
	\begin{itemize}
	\item{}
	\end{itemize}
\item{\textbf{Output}: id van de aangemaakte gebruiker}
\todo{specifieke foutcode genereren} 
\item{\textbf{Autorisatie}: een gebruiker met adminstrator}
\end{itemize}

\paragraph{Modify}

\paragraph{Remove}

\paragraph{Portefolio.List}

\paragraph{Portefolio.History}

1e param is struct die range definiëert, response is array van structs (datetime, points).

\paragraph{Orders.List}

\paragraph{Orders.Create}

\paragraph{Orders.Cancel}

\paragraph{Transactions.List}


\subsection{Finance-klasse}

Tenslotte zijn er nog de methodes gerelateerd met het effectieve beurswezen, die (hoofdzakelijk) terug te vinden zijn in deze klasse. Enkel het ophalen van de portefolio bevindt zich, logischerwijs, in de User-klasse.

\paragraph{Exchange.List}

\paragraph{Exchange.Modify}

\paragraph{Index.List}

\paragraph{Index.Modify}

\paragraph{Security.List}

ook flag key
output is enkel opprevlakkige informatie om tabel te maken

\paragraph{Security.Details}

input is ID uit Security.List

\paragraph{Security.Modify}

plus "hide" flag etc

\paragraph{Security.Flag}

1e param type vlag
2e param bool voor set/unset
3e param array array van security ID's




% Invoering
\part{Invoering}
%
% Oracle database
%

\chapter{Databases}

De opdracht specifi\"eerde dat we ook binnen het scholencomplex een database moesten opzetten, en aangezien StockPlay belang heeft aan veel en up-to-date data moesten we een manier vinden om de gescrapete data te synchroniseren tussen de twee databases.

Omdat we voor onze databaseservers Oracle 10g Express gebruikten, waren we gelimiteerd qua replicatie-mogelijkheden. Die limitaties hielden in dat de replicatie \emph{one-way} zou zijn, en enkel data zou gerepliceerd worden (dus geen \emph{stored procedures} of \emph{views}). Hoewel dit een serieuze beperking is, besloten we om toch door te gaan met deze piste en een replicatieschema op te zetten. Hierbij zou de "master"-database die op locatie zijn, en de databaseserver binnen het scholencomplex diens data gewoon repliceren. Aangezien de data dus vloeit van de externe naar de interne database, houdt dit in dat wijzigingen die we maken aan de interne database (met andere woorden: het spelverloop) niet zouden terechtkomen in de effectieve database.


%
% Backend
%

\chapter{Backend}

\todo{builden van branched nightlies, oid}


%
% Scraper
%

\chapter{Scraper}


%
% Interfaces
%

\chapter{Interfaces}


% Documentatie
\part{Documentatie}
\todo{Hier komt auto-gegenereerde data van onze code}

%
% Bijlagen
%

\part{Bijlagen}
\appendix

% XML-RPC
\chapter{XML-RPC specificatie}
\input{xml-rpc.tex}

% Woordenboek
\chapter{Woordenboek}
%
% Configuratie
%

% Label
\label{chap:woordenboek}


%
% Financiële terminologie
%

\section{Financi\"ele terminologie}


%
% Technische terminologie
%

\section{Technische terminologie}



\printindex
\end{document}

