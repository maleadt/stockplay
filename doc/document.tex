%
% Configuratie
%

% Preambule met standaardinstellingen
\documentclass[a4paper,oneside,final]{memoir}

% Noot: zorg ervoor dat Nederlandse woordsplitsing geactiveerd is.
\usepackage[english,dutch]{babel}

% Noot: je kan het graphicxpakket een optie dvips of pdftex doorgeven
% in dat geval moet je ze ook aan iiiscriptie doorgeven, dus bijvoorbeeld
% \usepackage[dvips]{graphicx}
% \usepackage[dvips]{iiiscriptie}

\usepackage[dvipsnames,usenames]{color} 
\usepackage{graphicx}
\usepackage{verbatim}
\usepackage{textcomp,lmodern,listings}
\usepackage{iiiscriptie}
\usepackage{a4wide}

% Nuttig pakket voor URL's
\usepackage{url}

% Extra functies
% Verkleinde margin entry
\setlength{\marginparwidth}{1.2in}
\let\oldmarginpar\marginpar
\renewcommand\marginpar[1] {\-\oldmarginpar[\raggedleft\footnotesize #1]%
{\raggedright\footnotesize #1}}

% Een TODO-entry
\newcommand{\todo}[1] {
	\addcontentsline{tdo}{todo}{\protect{#1}}
	\marginpar{#1}
}

% Een lijst van TODO-entries
\makeatletter
\newcommand \listoftodos {
	\section*{Todo list} \@starttoc{tdo}
}
\newcommand\l@todo[2] {
	\par\noindent \textit{#2}, \parbox{10cm}{#1}\par
} \makeatother

% Float defini\"eren voor codefragmenten
\usepackage{float}
\floatstyle{ruled}
\newfloat{code}{thp}{lop}
\floatname{code}{Codefragment}

% Hyperlink maken en URL in footnote tonen
\usepackage{hyperref}
\newcommand{\makeurl}[2]{\href{#2}{#1} \footnote{#2}}

% Functiedefinitie voor protocolstudie
\newcommand{\function}[5] {
	\subsubsection{#1}
	\begin{tabular}{|r p{10cm}|}
	\hline
	\textsc{Gebruik} |		& #2 \\
	\textsc{Parameters} |		& #3 \\
	\textsc{Output} |		& #4 \\
	\textsc{Autorisatie} |		& #5 \\
	\hline
	\end{tabular}
}

% Functiedefinities voor logboek handling
\usepackage{ifthen}
\newcommand{\lbdate}{}
\newcommand{\lbsetdate}[1]{
  \gdef\lbdate{#1}
}
\newcommand{\lbentry}[5] {
	\ifthenelse{\equal{\lbdate}{#1}}
	{
	}
	{
		\ifthenelse{\equal{\lbdate}{}}{}{
			& & & \\ \hline % TODO: workaround, zou niet nodig moeten zijn
			\end{tabular}
		}
		\subsection{#1}
		\begin{tabular}{|r r r p{10cm}|}
		\hline
		\textsc{Begin} & \textsc{Einde} & \textsc{Duur} & \textsc{Beschrijving} \\
		\hline
	}
	\lbsetdate{#1}
	#2 & #3 & #4 & #5 \\
}
\newcommand{\lbstop}[1] {
	& & & \\ \hline % TODO: consistentie fix, zie hierboven
	\end{tabular}
	\lbsetdate{}
	% totaal aantal uren
}

% Compacte enumeraties
\newenvironment{enumerate_compact}{
\begin{enumerate}
  \setlength{\itemsep}{1pt}
  \setlength{\parskip}{0pt}
  \setlength{\parsep}{0pt}
}{\end{enumerate}}
\newenvironment{itemize_compact}{
\begin{itemize}
  \setlength{\itemsep}{1pt}
  \setlength{\parskip}{0pt}
  \setlength{\parsep}{0pt}
}{\end{itemize}}

% Compacte environment voor use-cass
\newenvironment{compact}{\setlength{\parskip}{0pt}}{}

%stijlen voor listings
%
\lstdefinestyle{SQL}{
  breaklines=true,
  language=SQL,
  basicstyle=\normalsize,
  keywordstyle=\ttfamily\color{OliveGreen},
  identifierstyle=\ttfamily\color{CadetBlue}\bfseries,
  commentstyle=\color{Brown},
  stringstyle=\ttfamily,
  showstringspaces=true
}


\definecolor{myid}{rgb}{0.1,0.1,0.1}
\lstdefinestyle{Java}{language=java,
basicstyle=\ttfamily\normalsize,
numbers=left,stepnumber=1,numberstyle=\small\ttfamily,
numbersep=5pt,frame=tlbr,extendedchars=true,
commentstyle=\color{OliveGreen}\ttfamily,
%% stringstyle=\color{red}\ttfamily,
stringstyle=\ttfamily\color{Magenta},
keywordstyle=\ttfamily\color{Violet}\bfseries,
ndkeywordstyle=\ttfamily\color{Yellow}\bfseries,
identifierstyle=\ttfamily\color{myid},
% sensitive=false,
basicstyle=\scriptsize,
}


% Spacing properties
%\setlength{\parskip}{0.1in plus 0.2in minus 0.1in}


%
% Titelpagina
%

% Invullen velden
\departement{Departement Toegepaste Ingenieurswetenschappen}
\deptadres{Schoonmeersstraat 52 - 9000 Gent}
\studiejaar{3e Bachelor Informatica}
\soortrapport{
Analyse voor het vakoverschrijdend project Informatica
}
\title{Verslag 'StockPlay'}
\author{
Tim BESARD\\
Dieter DEFORCE\\
Laurens VAN ACKER\\
Thijs WALCARIUS
}

% Inhoudstabel maken en document starten
\setlength\cftpartnumwidth{2em}
\begin{document}
\maketitle
\pagenumbering{roman}
\tableofcontents
\pagenumbering{arabic}


%
% Inhoud
%

% Inleiding
\chapter*{Abstract}
StockPlay is een interactief spel waarmee beleggen op de beurs wordt gesimuleerd. De deelnemers krijgen de mogelijkheid om aandelen, opties, trackers, fondsen, enz. verhandelen (later ``effecten`` genoemd). De bedoeling van het spel is om een zo goed mogelijke prestatie neer te zetten op lange termijn. StockPlay is een doorlopend spel, waarbij alle spelers op gezette tijden worden ge\"evaluaeerd over hun prestaties. Hun prestaties worden beoordeeld op verschillende vlakken: presteren ze beter dan de BEL20, zijn ze bij de beste 25 beleggers van die dag, etc. Aan de hand daarvan wordt een klassement opgesteld waarmee de spelers zich met elkaar kunnen meten.

% Overzicht
\part{Overzicht project}
\label{pt:overzicht}
\chapter{Overzicht project}
Het project kan worden opgedeeld in 2 grote delen: een backend die instaat voor het beheer van alle data in het spel, en een heleboel modules die zich op deze backend enten en de data manipuleren.

\section{Backend}
De backend is het enige deel van dit project dat in rechtstreeks contact staat met de database. Hij bepaalt wie welke informatie kan opvragen aan de hand van een uitgebreid authorisatiemechanisme en laat toe om op een gebruiksvriendelijke manier informatie op te vragen en te manipuleren.
Om de overhead door deze backend zo licht mogelijk te houden, werd er gekozen voor een zo licht mogelijk protocol: XML-RPC. Dit laat toe om een grote verscheidenheid aan modules te laten communiceren met de backend: van een volwaardige ASP.NET-website, over een scrapermodule in Perl, tot een eenvoudige applet in Java ME: het protocol is in staat om aan de behoeften van elke module te voldoen.
\subsection{Filtering}
Om elke client toe te laten om gerichte aanvragen te maken, was er nood aan een manier waarop specifieke objecten opgevraagd konden worden. Daarom werd ervoor gekozen om een eigen filtermechanisme te ontwikkelen.

\section{Website}
Op het website-gedeelte van dit project kunnen spelers deelnemen aan het StockPlay-spel. Hier kunnen ze de effecten die in dit spel zitten bekijken, orders aanmaken en hun portefolio beheren.

\subsection{Dynamische grafieken}
Uit de beurskoersen van het verleden kan veel informatie gepuurd worden, daarom is een goeie tool om de historische beurskoersen te kunnen analyseren geen overbodige luxe. Er werd voor gekozen om zelf een Javascript-library te ontwikkelen om de grafieken weer te geven. Deze javascript-library communiceert via JSON met de publieke ASP.NET webservice om de nodige data op te halen.

\section{Administratie desktopclient}
Er werd bij dit project voor gekozen om de component met Java Swing in te vullen als administratieclient voor het project. Bij dit administratieprogramma krijgt een beheerder een overzicht over de status van de verschillende componenten. De beheerder kan ook de aanwezige effecten en gebruikers opvragen, en deze desgewenst aanmaken, wijzigen of verwijderen.

\section{Beurskoersen-scraper}
Het is evident dat een beursspel zonder live beurskoersen niet veel voorstelt.. De scraper is in dit project dan ook een van de meest vitale delen: de scraper haalt live koersinformatie van alle effecten uit het spel uit de AJAX-feed van de website van krant "De Tijd", en voegt deze continue toe aan de database van het spel. De scraper is zodanig opgezet dat ze volledig modulair werkt, en extra (reserve) bronnen kan aanspreken om bijkomende beursinformatie op te halen (bijvoorbeeld een tweede feed).
\subsection{Historische beurskoersen-scraper}
Er was ook nood aan meer extensieve informatie voor elk effect. Bijvoorbeeld om onze AI-speler goeie inschattingen te kunnen laten maken, alsook om voldoende informatie te kunnen tonen aan onze spelers in de grafieken, was er nood aan historische beurskoersen. Daarom werd ook een scraper gemaakt die op de website van Euronext de historische data ophaalde.

\section{AI-speler}
Om ook in de beginfase de eerste spelers van genoeg concurrentie te voorzien, werd er ook een AI-speler ontwikkeld. Deze AI-speler gebruikt een genetisch algoritme om zijn orders te bepalen. Door te vari\"eren in de aangeboden parameters aan elke AI-instantie kunnen gemakkelijk verschillende AI-spelers worden aangemaakt. Dit laat toe om op een eenvoudige wijze het spel te voorzien van genoeg spelers om het ook voor de eerste menselijke spelers direct interessant te maken, en te houden natuurlijk!

\section{Mobiele spelclient}
Om aan te tonen dat onze keuze voor het XML-RPC protocol ons veel mogelijkheden bood, werd er ook een client ontwikkeld in de vorm van een Java ME-applicatie. Java ME bevat slechts zeer rudimentaire functionaliteit, maar toch was ook daarvoor een XML-RPC-client beschikbaar.
Met deze client kan een speler zijn portfolio en zijn huidige orders opvragen. De client staat ook enkele rudimentaire bewerkingen toe zoals het annuleren van een nog niet uitgevoerd order, alsook het aanmaken van een nieuw order.

\section{Transactiemanager}
Spelers kunnen op de website orders aanmaken om een effect te kopen of te verkopen, en kunnen aangeven dat deze pas uitgevoerd mogen worden als aan bepaalde voorwaarden voldaan wordt (zie hoofdstuk \ref{chap:geavanceerde_orders}). De uitvoering van een order is echter een aparte taak: in de re�le beurswereld wordt een order doorgestuurd naar de beurs als alle voorwaarden voor het order voldaan zijn. Dit betekent dus dat aan alle technische voorwaarden is voldaan en dat er een tegenpartij is die het effect wil kopen of verkopen tegen de aangegeven prijs.

In onze applicatie gebeurt het verwerken van de orders en het omzetten naar transacties door een extra module die deze taak van de beurs simuleert: de transactiemanager. Deze module zal periodiek alle actieve orders overlopen en degene waar alle voorwaarden van voldaan zijn omzetten in transacties.

\section{Puntenmanager}
Deze applicatie zal periodiek (dit door middel van cronjobs) alle spelers in het spel evalueren en op basis van hun prestaties over een bepaalde periode (dag, week, maand, ...) punten toekennen. Met deze punten is het mogelijk om een klassement op te stellen van de spelers in het spel. Zo kan elke speler in het spel, van beginner tot de meest gevorderde speler, zich telkens bepaalde doelen stellen (bijvoorbeeld: eerste in het weekklassement worden), en blijft het spel voor iedereen een uitdaging op elk moment.


\chapter{Ontwikkelingsevolutie}


\section{Taakverdeling}
De grote omvang van het project zorgde ervoor dat al snel iedereen zich begon toe te leggen op bepaalde domeinen van het project. Met uitzondering van de backend, de website en de documentatie.

\begin{itemize}
	\item \textbf{Tim Besard: }
	\begin{itemize}
	\item Implementatie van XML-RPC-backend
	\item Toevoegen functionaliteit in de backend
	\item Live beurskoersen-scraper
	\item Historische beurskoersenscraper
	\item Artifici�le intelligentie spelers
	\item Linux Serverbeheer
	\end{itemize}
	
	\item \textbf{Dieter Deforce: } 
		\begin{itemize}
			\item Toevoegen functionaliteit in de backend
			\item ADO.NET-interface met de backend\footnote{Het gebruik van ADO.NET was een vereiste van het project, en moest worden gedemonstreerd. De ADO.NET-interface werd na de demo uit het project gehaald}
			\item De XML-RPC.NET-client die instaat voor de communicatie met de backend voor applicaties op het .NET-platform
			\item De ASP.NET-website
			\item De puntenmanager, die spelers op gezette tijden evalueert en punten toekent
		\end {itemize}
		
	\item \textbf{Laurens Van Acker: }
	\begin{itemize}
 	\item Toevoegen functionaliteit en foutstucturen in de backend
	\item Javascript-library ontwikkelen voor de dynamische grafieken
	\item De publieke ASP.NET Webservice
	\item Grafisch ontwerp en Javascript ondersteuning
	\item Transactiemanager: uitbouw en aanpassen structuur, verwerking van de geavanceerde orders.
 	\item Filtering: toevoegen van extra functionaliteit
	\item Contracten en presentaties
	 \end{itemize}
	 
	 \item \textbf{Thijs Walcarius: }
	 \begin{itemize}
	 	\item Ontwerp Data Access Objects voor communicatie tussen backend en database
	 	\item Instellen van Oracle-database, ontwerp triggers
		\item Toevoegen functionaliteit en testcases in de backend
	 	\item Java Data Access Objects
	 	\item Java Administratieclient
	 	\item Java ME mobiele client
	 	\item Rudimentaire versie transactiemanager
	 \end{itemize}
\end{itemize}


\section{Planning}

Door de strikte timing van het project met zijn verschillende iteraties kon er niet gewacht worden op een volledig functionele backend om vanaf te starten.. De ontwikkeling verliep vooral rond de verschillende iteraties.

\subsection{Alfa-iteratie}
Voor elk groot onderdeel van het project moest er iets gedemonstreerd kunnen worden. We waren dus verplicht om zeer snel te bepalen op welke fundatielaag we gingen werken, om zo voor elk onderdeel een stuk rudimentair te kunnen uitwerken.

\subsubsection{Backend}
Er werd gezocht naar een nette en abstracte methode om de backend op te bouwen: hoe wordt de database best benaderd? Wat is een goeie structuur voor de backend zelf? Hoe bouwen we de XML-RPC-berichten op?
Uiteindelijk werd het deel van de backend dat zich bezig houdt met het opvragen, aanmaken en wijzigen van effecten gekozen als pilootproject.

\paragraph{Filtering}
Om clients de mogelijkheid te bieden om op een eenvoudige manier objecten te selecteren werd een eigen filter ontwikkeld.

\paragraph{Data Access Objects}
Omdat de voorstelling van data in de backend nogal verschilt van de voorstelling in de database, werd er gezocht naar een systeem dat de omzetting zo transparant mogelijk maakte.. Omdat dit een probleem is waarmee bijna elk softwarepakket dat gebruik maakt van een database wordt geconfronteerd, bestaan hier ook design patterns voor: er werd gekozen om het 'Data Access Object'-pattern te implementeren.

\subsubsection{Website}
Ontwikkeling van een ADO.NET-interface met de backend. Eerste rudimentaire functionaliteit inbouwen voor het ophalen en weergeven van effecten. Ook werd er gezocht naar manieren om de gegenereerde HTML-code vanuit ASP.NET wat op te schonen zodat er voor de vormgeving gemakkelijker gebruik kon worden gemaakt van CSS-stylesheets.

\paragraph{Dynamische grafieken}
Er werd gezocht naar een goeie manier om de historische koersinformatie grafisch weer te geven. Daarom werden verschillende javascript grafieklibraries met elkaar vergeleken en uitgetest.

\subsubsection{Administratie desktopclient}
Voor de Java Swing-applicatie werd er gezocht naar een manier om een mooi en functioneel programma te maken. Er werd eerst voor geopteerd om zelf een verticale menubalk te maken vergelijkbaar met de menubalk uit MS Office Outlook 2003, maar we liepen al snel tegen beperkingen van Java in verband met precieze timing. Uiteindelijk werd er geopteerd om de SwingX-library te gebruiken.
Bij deze iteratie was de basisfunctionaliteit om de effecten op te halen aanwezig. Hiervoor werd ook de XML-RPC-client module ontwikkeld die zich met het beheer van effecten bezig houdt.

\subsubsection{Beurskoersen-scraper}
Verschillende bronnen van koersinformatie werden gezocht, en er gebeurde een eerste implementatie om de AJAX-feed van Tijd.be te verwerken.

\subsubsection{Mobiele spelclient}
Er werd uitgezocht of er goeie XML-RPC-clients beschikbaar waren. Ook werd de functionaliteit van het Java ME-platform onderzocht, en de mogelijkheden van Rapid Application Development-tools onderzocht (in casu: de Netbeans IDE hiervoor)


\subsection{Beta-iteratie}
Bij deze iteratie werd er vooral gefocust om zoveel mogelijk functionaliteiten in te bouwen. De bedoeling was om tegen de beta-iteratie een spel te kunnen aanbieden waarbij de speler beschikte over alle basisfunctionaliteiten: beheer van zijn portfolio, aanmaken van orders, bekijken van aandelen, etc.
Om tegemoet te komen aan de vraag om een algoritmisch aspect in te bouwen, werd er ook voor geopteerd om een kunstmatige artifici�le intelligentie speler te ontwikkelen.

\subsubsection{Backend}
Het volledige protocol werd geimplementeerd in de backend. Ook gebeurden er aanpassingen aan het protocol om het aantal requests van de clientmodules te beperken, en zo een vlottere gebruikerservaring te kunnen aanbieden.

\subsubsection{Website}
De website werd verder uitgewerkt: gebruikersregistratie, beheer van portfolio, aanmaken van orders.

\paragraph{Dynamische grafieken}
De dynamische grafieken module werd verder uitgewerkt in object geori�nteerde javascript. Er werd een menu toegevoegd en de functionaliteit achter de knoppen werd ontwikkeld. Het is mogelijk om referenties toe te voegen, in te zoomen via knoppen of muisbewegingen een bereik te selecteren.
Ook werd een weergave historiek ingebouwd zodat weergavewijzingen ongedaan kunnen gemaakt worden. De ASP.NET Webservice werd ontwikkeld om de data via AJAX en JSON te kunnen leveren. En een tweede gekoppelde grafiek werd toegevoegd met volume data.

\subsubsection{Administratie desktopclient}
De clientlibrary werd volledig afgewerkt, zodat ze bijna alle functionaliteiten uit de backend bevatte.
Vervolgens werd er gewerkt aan het bewerken van de opgehaalde effecten, alsook werd er begonnen aan het gebruikersbeheer. Omdat de benodigde functionaliteit tussen het effectenbeheer en het gebruikersbeheer danig verschilde, werd er voor geopteerd om geen generiek CRUD-design te gebruiken, maar aparte implementaties te voorzien.

\subsubsection{Beurskoersen-scraper}
De scraper werd verder afgewerkt en kan nu ook zelfstandig de nodige effecten en beurzen aanmaken.

\paragraph{Historische beurskoersen-scraper}
Er werd ook een historische beurskoersen-scraper gemaakt die alle historische koersen vanaf Euronext ophaalt (een koers per dag)

\subsubsection{Kunstmatige artifici�le intelligentie speler}
Vooreerst werd er gezocht naar wat de beste manier is om generische algoritmen toe te passen op beurskoersen. Daarvoor werd er beroep gedaan op vakliteratuur die daarover beschikbaar was. Vervolgens werd er ge\"experimenteerd met verschillende programmeertalen om uit te zoeken wat het snelst resultaat zou opleveren. Uiteindelijk werd er gekozen voor Perl omdat deze programmeertaal op eenvoudige wijze toelaat om complexe gegevensstructuren te verwerken met een goede performantie.

\subsubsection{Mobiele spelclient}
Er werd besloten om de mobiele client niet op te nemen in de beta-iteratie omdat andere onderdelen een hogere prioriteit hadden.

\subsubsection{Transactiemanager}
De transactiemanager was lange tijd uit het oog verloren. Omdat het echter toch een essentieel element is om het volledige spel te simuleren werd uiteindelijk last-minute een prototype gemaakt. Die kon enkel eenvoudige types orders aan (immediate order, en orders met een enkelvoudige limiet) en had nog geen gemeenschappelijke datacontainer.

\subsection{Release}

\subsubsection{Backend}
De backend reageerde vaak niet snel genoeg op aanvragen, bij nader onderzoek bleek er een serieuze bottle-neck aanwezig te zijn bij de connectie tussen de database en de backend. Oracle stuurde vaak aan zijn bovenlimiet qua snelheid informatie door, waardoor kostbare seconden verloren gingen. Er werd daarom voor gekozen om een cachingmechanisme in te bouwen. Dit werd echter niet van de grond af gedaan, maar er werd gebruik gemaakt van een reeds publiek beschikbare library 'cache4j'.
Ook werd er een beveiligingsmechanisme ingebouwd, zodat afhankelijk van de aangemelde gebruiker maar een beperkte subset van de backend-API aanspreekbaar werd.

\subsubsection{Website}
De nog ontbrekende stukken werden toegevoegd: de ranking, [VUL AAN]. Ook werd er functionaliteit toegevoegd om complexe orders in te voeren.

\paragraph{Dynamische grafieken}
Er werd de mogelijkheid toegevoegd om extra effecten en indexes op de grafiek toe te voegen als referentie, zodat effecten gemakkelijker met elkaar kunnen worden vergeleken. 
De module werd ook groter gemaakt en de bibliotheek werd sterk geoptimaliseerd voor snelheidswinsten.

Er werden ook aanpassingen gedaan zodat nu slechts maximum 100 punten teruggegeven worden en niet zomaar alle punten in een bepaalde range.

\subsubsection{Administratie desktopclient}
De mogelijkheid om met een elektronische identiteitskaart in te loggen werd toegevoegd. Omdat er te weinig tijd restte om uit te zoeken hoe vervolgens op een met certificaten beveiligde manier een beveiligde sessie op de backend kon worden aangevraagd, is het nodig dat lokaal de logingegevens van een administrator-account worden opgeslagen om een nieuw sessie-id te kunnen aanmaken.

\subsubsection{Beurskoersen-scraper}
\paragraph{Historische beurskoersen-scraper}

\todo TIM

\subsubsection{Kunstmatige artifici�le intelligentie speler}

\todo TIM

\subsubsection{Mobiele spelclient}
De mobiele spelclient werd weer terug opgenomen. De client-library die beschikbaar was in Java werd geport naar Java ME. Aanvankelijk ging dat gepaard met de ene teleurstelling na de andere door het gebrek aan functionaliteit (geen enums, geen generics, etc), maar die problemen zijn overwonnen.. Wel werd er besloten om wat extra functionaliteit uit de mobiele client te schrappen. Het is immers een nogal log systeem om te debuggen en deze module is eerder bedoeld als een proof-of-concept en vormt geen essentieel onderdeel in het project.

\subsubsection{Transactiemanager}
De transactiemanager werd herschreven zodat alle data via een class opgevraagd kon worden. Alsook werd deze uitgebreid met allerhande geavanceerde orders.
Zodoende wordt de benodigde data slechts eenmalig per iteratie opgevraagd en niet voor elk order dat gecontroleerd moet worden. Deze werd ook herschreven om te voldoen aan de aangepaste beveiligde backend en er werden controles ingebouwd om te kijken of de speler wel over een voldoende cashpositie bezit om een transactie uit te voeren.

\subsubsection{Puntenmanager}
De puntenmanager was het laatste grote ontbrekende deel aan het project. Deze werd dan ook snel in deze iteratie onder handen genomen.
\todo DIETER

\chapter{Projectevaluatie}


\section{SWOT-analyse}

\subsection{Strenghts}

De grootste troef van dit project is onze backend. In dit deel van het project werd het meeste tijd gestopt, en het resultaat is een elegant, flexibel en coherent geheel. 
Zo is de datalaag van onze backend erop voorzien dat er gemakkelijk van onderliggende database kan worden gewisseld. Ook zijn er faciliteiten ingebouwd om een vlotte conversie van de interne objecten naar de gebruikte gegevensstructuren in het XML-RPC-protocol te voorzien.

\subsection{Weaknesses}

\paragraph{Backend} Het beveiligingsmodel werkt goed, maar de categorieën in dewelke de functies zijn ondergebracht zouden best herbekeken worden, zodat ze beter aansluiten bij wat de verschillende modules van het project effectief nodig hebben. Dit zou dan toelaten om een meer nauwkeurige afsluiting van de API naar de modules toe te doen, zodat de beveiligingsrisico's worden verminderd.. Ook is er nog ruimte voor verbetering op het gebied van controles in functies zelf. Er gebeuren al heel wat controles, maar het kan nog beter. (heeft gebruiker X wel het recht om bepaalde informatie van gebruiker Y op te vragen?)

Onze ervaring leert dat wanneer de backend enkele uren heeft gedraaid zonder dat er veel activiteit op was, dat de verbindingen met de Oracle-database komen te vervallen zonder dat de Apache Database Connection Pool dit merkt.. Dit zorgde ervoor dat requests naar onze backend vaak na 20 minuten (!) een timeout gaven met een Oracle-exceptie. Ondertussen is de Apache Database Connection Pool al zodanig geconfigureerd dat de verbindingen worden getest voordat ze worden teruggegeven, waardoor een request na 20 minuten toch een geldig antwoord krijgt.. Dat is al beter, maar er zou best nog verder worden gezocht achter een manier om de timeout in te korten. Dit is tijdens de laatste dagen van het ontwikkelingsproces ook gebeurd, maar omdat deze fout maar erg sporadisch optreedt kan er niet met zekerheid gezegd worden dat ze is opgelost..

Het cachingmodel dat werd ingebouwd in de backend kan nog verder worden verfijnd. Momenteel wordt er nog te weinig rekening gehouden met mogelijke veranderingen van de achterliggende data. Er wordt best gezocht naar een meer geavanceerd model, of een eigen cachingmechanisme. Het filtersysteem dat werd ingebouwd in de backend is flexibel gemaakt, en er dus ook op voorzien om ook opzoekingen in andere gegevensstructeren mogelijk te maken, wat natuurlijk goede perspectieven biedt..

\paragraph{Administratie desktopclient} De implementatie van de eID-loginprocedure moet nog verder worden verbeterd.  Door gebrek aan tijd was het niet mogelijk om te zoeken naar een manier om aan de hand van de certificaten bij de elektronische identiteitskaart een sessie aan te maken op de backend.. Daarom is het nu vereist om de logingegevens van een administratoraccount permanent op te slaan, om met deze gegevens een geldige sessie te kunnen opvragen voor de gebruiker die inlogde met de eID.

\paragraph{Mobiele client} De mobiele client werd niet opgebouwd op een erg robuuste manier. Zou deze verder ontwikkeld worden, dan zou er al snel nood zijn aan een grote herorganisatie van de code om ze flexibeler te maken.

\paragraph{Dynamische grafieken} Als deze module gebruikt zou worden op een productie site zouden er nog een aantal optimalisaties moeten gebeuren om ze bruikbaar te maken. Zo zouden de referenties moeten gelijkgetrokken worden met het effect dat getoond wordt. Referenties op een gelijke schaal onthullen meer dan het vergelijken van een effect met een koersprijs van 20 met een index die aan 4000 punten noteert.

De volumes die getoond worden zijn nog absolute volumes, daar zouden nog relatieve volumes van gemaakt kunnen worden. En er zouden nog een aantal snelheidsoptimalisaties doorgevoerd kunnen worden. Zo kan een request geannuleerd worden (of vertraagd worden) als deze nog aan het binnenkomen is en er is al een nieuwe aanvraag gebeurd (vormt een probleem bij het veel en snel scrollen van de muis). De eerste data zou via de HTML pagina reeds meegestuurd kunnen worden, wat een HTTP request zou besparen. De data en alle request zouden ook nog via een HTTP pagina kunnen binnenkomen, nu gebeurd dit nog via verschillende requests.
Er zou client side ook nog veel meer caching ingebouwd kunnen worden.

\todo Known issues / bugs

\subsection{Opportunities}

\paragraph{Algemeen} Het doel van het spel is om zoveel mogelijk de beurs in het echte leven de simuleren. Dit is echter zo'n danig complex gegeven dat we nog lang niet een levensecht spel hebben gecre\"erd. Elementen die nog in het spel ontbreken zijn bijvoorbeeld het aanrekenen van transactiekosten en bewaargelden. 
Momenteel is er ook maar een beperkt aantal beurzen aanwezig in het spel. We werden hiertoe gedwongen omdat de gescrapete data van andere beurzen vaak niet granulair genoeg was. Zou er echter beschikking zijn over betere datafeeds, dan kan het spel op dit vlak snel uitbreiden. Indien er ook beurzen zouden bijkomen die met andere munteenheden werken, zou ook dit een extra dimensie aan het spel kunnen toevoegen, het werken met vreemde munten kan gelukkig snel worden ingebouwd in de bestaande voorzieningen.

\paragraph{Puntenmanager} De puntenmanager voorziet in alle faciliteiten om gebruikers op verschillende vlakken te quoteren. Om het spel interessanter en aantrekkelijker te maken is het een goed idee om nog verder te zoeken naar domeinen waar een speler op kan worden gequoteerd. Het is misschien ook een goed idee om aparte klassementen te voorzien voor bepaalde criteria (bijvoorbeeld: met zo weinig mogelijk transacties toch zo goed mogelijk scoren).

\paragraph{Transactionmanager} De transactionmanager zou een aantal optimalisaties kunnen ondergaan. Er zouden orders gegroepeerd kunnen worden die dan in een keer gecontroleerd kunnen worden. Of een soort afhankelijke orders met triggers zouden aangemaakt kunnen worden. Zodoende moeten dan niet alle lopende orders continue bij elke iteratie gecontroleerd worden.

\todo Uitbreidingsmogelijkheden

\subsection{Threats}

\todo zwakheden in ons model (probeer hier creatief te zijn en zwakheden te zetten die eigenlijk positief zijn (vb. als je later op een sollicitatiegesprek zo'n analyse van jezelf moet maken is een goeie ``door mijn oog voor detail durf ik wel eens te lang doorgaan op een deelprobleem'')

\paragraph{Algemeen} Het door ons gekozen onderwerp was zodanig breed en complex, dat er al snel de verleiding kwam om teveel elementen vanuit de echte wereld van de financi\"ele markten in ons spel te verwerken. Dit heeft ertoe geleid dat we ons project soms iets te breed hebben gezien, waardoor de workload onaangenaam hoog is komen te liggen.. Het was misschien beter geweest om een iets nauwere scope te nemen, zodat we ons ook wat meer op de afwerking van onze code konden richten.
\emph{Ter illustratie: }ons project bevat volgens \emph{StatSVN} 54652 lijnen code, waarbij gepoogd werd enkel de zelfgemaakte bestanden in rekening te nemen (geen gecompileerde klassen, geen externe libraries, word documenten, presentaties, afbeeldingen, enzovoort). 

\paragraph{Backend}
Het cachingmodel dat werd ingebouwd in de backend kan nog verder worden verfijnd. Momenteel wordt er nog te weinig rekening gehouden met mogelijke veranderingen van de achterliggende data. Er zou best worden gezocht naar een meer geavanceerd model, of een eigen cachingmechanisme. Het filtersysteem dat werd ingebouwd in de backend is flexibel gemaakt en er dus ook op voorzien om ook opzoekingen in andere gegevensstructeren (andere databases of bijvoorbeeld een eigen opslagsysteem) mogelijk te maken, wat natuurlijk goede perspectieven biedt..>>>>>>> .r653


% Realisatie
\part{Realisatie}
\label{pt:realisatie}
\chapter{Dataontwerp}

\begin{figure}[h!]
	\centering
		\includegraphics[width=0.5\textwidth]{images/realisatie/ER_Diagram}
	\caption{Entity-relationship model.}
\end{figure}

\section{Ontwerp backend}
\todo{dit hoort hier niet}
De klassen die de persistente data uit de database moeten weergeven staan opgesomd in onderstaand klassendiagram:

\begin{figure}[h!]
	\centering
		\includegraphics[width=0.5\textwidth]{images/realisatie/Class_Diagram}
	\caption{Klassendiagram persistente data in backend.}
\end{figure}


\chapter{Procedureontwerp}

\section{Backend protocol}

Zoals vermeld in de behoeftenanalyse wordt alle databasetoegang uitgevoerd via een gemeenschappelijke backend. De interface hiervoor stelt echter een aantal bijzondere eisen:
\begin{itemize}
\item{Taalonafhankelijk: aangezien de interfaces met behulp van verschillende programmeertalen gerealiseerd worden, moet de interface toegankelijk zijn vanuit een zo wijd mogelijke waaier aan programmeertalen.}
\item{Lichtgewicht: als een mobiele interface een mogelijke uitbreiding kan zijn, moet het gekozen protocol compact zijn en mogen de eventueel benodigde libraries niet te zwaar zijn.}
\item{Toegankelijk: omdat interfaces niet noodzakelijk uitgevoerd worden op hetzelfde systeem van de backend, is het mooi meegenomen als het protocol geen probleem vormt in gelimiteerde omgevingen.}
\end{itemize}

Na verschillende kandidaten overwogen te hebben, hebben we gekozen voor XML-RPC. Dit is een lichtgewicht Remote Procedure protocol, dewelke methodeaanvragen en -antwoorden verpakt in XML-data en ze als POST request verstuurd over het HTTP protocol (zie bijlage \ref{chap:xml-rpc} voor de exacte specificatie).

Het protocol voldoet aan de opgestelde eisen: aangezien het verderbouwt op het bestaande HTTP-protocol kan het gebruik maken van diens mogelijkheden (zoals compressie en encryptie), en kan het indien een specifieke bibliotheek onbestaande is eenvoudig verwerkt worden via reeds bestaande HTTP- en XML-bibliotheken. Bovendien verschilt de communicatie niet van regulier browsen waardoor de toegankelijkheid in gelimiteerde omgevingen ook toeneemt.

Voor de programmeertalen die we gaan gebruiken bij het implementeren blijken er reeds verschillende bibliotheken beschikbaar te zijn, wat het gemak van gebruik opnieuw verhoogt. Hierbij een opsomming van de specifieke bibliotheken die we zullen gebruiken om informatie te versturen en ontvangen over het XML-RPC protocol:
\begin{itemize}
\item{\textbf{Perl}: \makeurl{XML::RPC}{http://search.cpan.org/~daan/XML-RPC-0.9/lib/XML/RPC.pm}}
\item{\textbf{C\#}: \makeurl{XML-RPC.NET}{http://www.xml-rpc.net/}}
\item{\textbf{Java}: \makeurl{Apache XML-RPC}{http://ws.apache.org/xmlrpc/}}
\end{itemize}

Aangezien XML-RPC geen ondersteuning biedt voor namespaces of andere vormen van functieorganisatie, hanteren we zelf een mechanisme om dit te bekomen: een methode-naam bestaat altijd uit twee delen, gescheiden door een punt. Het deel voor het scheidingsteken duidt het pakket aan, het deel erna de specifieke methode.
Zo delen we de backend op in volgende primaire klassen:
\begin{itemize}
\item{System: functionaliteit voor beheer van het systeem.}
\item{User: beheer van gebruikers, ook voor gebruikers zelf.}
\item{Finance: functionaliteit gerelateerd met het beurswezen.}
\end{itemize}
Hogere-orde klassen zijn eventueel ook mogelijk (zoals \emph{System.Database}), maar niet verplicht. De semantiek is daarbij identiek aan primaire klassen, met een punt als scheidingsteken.

\subsection{Algemene foutcodes}

De XML-RPC specificatie biedt ondersteuning voor foutberichten, in de form van een bericht met een $<$fault$>$ tag. Die tag moet steeds twee $<$member$>$ tags bevatten, namelijk een foutcode $<$faultCode$>$ van het integer type, en een foutbericht $<$faultString$>$ van het string type. Elk van de klassen kan zo specifieke foutmeldingen vastleggen.
Maar er zijn ook generieke foutmeldingen, die van toepassing zijn op alle klassen. Deze foutmeldingen, waarvan de foutcode in het bereik $[0, 100[$ valt, worden hieronder beschreven:

\begin{table}
\begin{tabular}{| c p{5cm} p{7cm} |}
	\hline
	Foutcode & Foutbericht & Controle \\
	\hline
	
	0 & Version Not Supported & De client gebruikt een verkeerd communicatieprotocol. \\
	\hline
	
	$[1-10[$ & \emph{Subsystem failures.} & \\
	1 & Internal Failure & Verifieer de status van de backend, de log kan hierbij helpen. \\
	2 & Database Failure & Er is een probleem met de database (onbeschikbaar, corrupt, ...), zie de log voor meer details. \\
	\hline
	
	$[10-20[$ & \emph{Service issues.} & \\
	10 & Service Unavailable & De backend kan tijdelijk niet gebruikt worden (werkzaamheden, overloaded, ...). \\
	11 & Unauthorized & Meld u aan vooraleer deze functie te gebruiken. \\
	\hline
	
	$[20-30[$ & \emph{Method issues.} & \\
	20 & Not Found & Methode niet gevonden, verifieer de schrijfwijze en de klasse. \\
	21 & Bad Request & Probleem met de parameters, controleer het gebruik van de methode. \\
	\hline
\end{tabular}
\caption{Generieke foutcodes in het backend-protocol.}
\end{table}

\subsection{Authenticatie en autorisatie}

Aangezien het XML-RPC protocol gebruik maakt van het HTTP-protocol, kunnen we diens functionaliteit gebruiken om authenticatie te bekomen. Daartoe zullen we gebruik maken van \emph{basic authentication}, waarbij de client indien gevraagt een gebruikersnaam en wachtwoord naar de server doorstuurd.
\todo{Dit is tijdelijk}

Afhankelijk van de capabiliteiten van het XML-RPC pakket dat we in de backend gebruiken (Apache XML-RPC), kan dit op twee manieren verlopen. Indien de bibliotheek ondersteuning biedt voor het on-demand inschakelen van authenticatie gebaseerd op de ontvangen request, kunnen we zo wanneer benodigd \emph{basic authentication} inschakelen en de webserver zelf een HTTP-401 laten terugsturen, zonder hiervoor extra code in de backend benodigd is.
Als deze optie niet dynamisch ingeschakeld kan worden, zullen we zelf een eigen foutmelding moeten terugsturen die aanduidt dat authorisatie benodigd is. Als de client zo een $<$fault$>$-bericht ontvangt, zal die een nieuwe XML-RPC socket openen op een alternatieve URL (bijvoorbeeld \texttt{http://server.hogent.be/authenticated}). Aangezien de URL nu verschillend is, kunnen we de webserver in de backend zodanig configureren dat authenticatie vereist is voor die zone. Zo bekomen we eveneens verplichte authenticatie voor bepaalde methodes, maar dit door ze enkel beschikbaar te stellen in een subset van het serverdomein. Dit vereist enige extra code in de backend.

Autorisatie tenslotte is beperkt: in de huidige opzet ondersteunen we geen flexibel-toegekende rechten, enkel een bit die bepaalt of de gebruiker een administrator is of niet.

\todo{gebruikers-autorisatie ook via dit systeem, return rechten bitmap / ID}

\subsection{System-klasse}

Deze klasse biedt de client mogelijkheden om het systeem te beheren, met name ophalen van informatie, wijzigen van configuraties, en (her)starten of stoppen van bepaalde subsystemen.


\begin{itemize}
\item{\textbf{Gebruik}: }
\item{\textbf{Parameters}:}
	\begin{itemize}
	\item{}
	\end{itemize}
\item{\textbf{Output}:}
	\begin{itemize}
	\item{}
	\end{itemize}
\item{\textbf{Autorisatie}: }
\end{itemize}

\paragraph{Backend.Status}
\begin{itemize}
\item{\textbf{Gebruik}: status van de backend ophalen.}
\item{\textbf{Parameters}: geen.}
\item{\textbf{Output}: integer die status beschrijft:}
	\begin{itemize}
	\item{0: maintenance-mode}
	\item{1: de backend werkt}
	\end{itemize}
\item{\textbf{Autorisatie}: administrator-rechten benodigd}
\end{itemize}

\paragraph{Backend.Stats}
\begin{itemize}
\item{\textbf{Gebruik}: backend-statistieken ophalen.}
\item{\textbf{Parameters}: geen.}
\item{\textbf{Output}: struct met statistieken}
	\begin{itemize}
	\item{users: aantal gebruikers online}
	\item{req: aantal verwerkte XML-RPC requests}
	\item{uptime: hoe lang de backend al draait}
	\end{itemize}
\item{\textbf{Autorisatie}: administrator-rechten benodigd.}
\end{itemize}

\paragraph{Backend.Restart}
\begin{itemize}
\item{\textbf{Gebruik}: de backend herstarten.}
\item{\textbf{Parameters}: geen.}
\item{\textbf{Output}: bool die indiceert of de actie succesvol was.}
\item{\textbf{Autorisatie}: administrator-rechten benodigd.}
\end{itemize}

\paragraph{Backend.Stop}
\begin{itemize}
\item{\textbf{Gebruik}: de backend permanent stilleggen.}
\item{\textbf{Parameters}: geen.}
\item{\textbf{Output}: bool die indiceert of de actie succesvol was.}
\item{\textbf{Autorisatie}: administrator-rechten benodigd.}
\end{itemize}

\paragraph{Database.Status}
\begin{itemize}
\item{\textbf{Gebruik}: status van de database ophalen.}
\item{\textbf{Parameters}: geen.}
\item{\textbf{Output}: integer die status beschrijft:}
	\begin{itemize}
	\item{0: database niet te bereiken}
	\item{1: de backend werkt}
	\end{itemize}
\item{\textbf{Autorisatie}: administrator-rechten benodigd}
\end{itemize}

\paragraph{Database.Stats}
\begin{itemize}
\item{\textbf{Gebruik}: database-statistieken ophalen. Dit zijn de statistieken geleverd door de database zelf, ze zijn dus niet beperkt tot acties ondernomen door de backend.}
\item{\textbf{Parameters}: geen.}
\item{\textbf{Output}: struct met statistieken}
	\begin{itemize}
	\item{queries: aantal uitgevoerde queries}
	\item{uptime: hoe lang de database al draait}
	\item{traffic: hoeveelheid data verzonden en ontvangen}
	\item{slow\_queries: aantal queries die teveel uitvoeringstijd vergden}
	\end{itemize}
\item{\textbf{Autorisatie}: administrator-rechten benodigd.}
\end{itemize}

\paragraph{Scraper.Status}
\begin{itemize}
\item{\textbf{Gebruik}: status van de scraper ophalen.}
\item{\textbf{Parameters}: geen.}
\item{\textbf{Output}: integer die status beschrijft:}
	\begin{itemize}
	\item{0: niet geactiveerd}
	\item{1: inactief}
	\item{2: bezig met data-mining}
	\end{itemize}
\item{\textbf{Autorisatie}: administrator-rechten benodigd}
\end{itemize}

\paragraph{Scraper.Stats}
\begin{itemize}
\item{\textbf{Gebruik}: scraper-statistieken ophalen.}
\item{\textbf{Parameters}: geen.}
\item{\textbf{Output}: struct met statistieken}
	\begin{itemize}
	\item{executes: aantal voltooide plugin-uitvoeringen}
	\item{uptime: hoe lang de scraper al draait}
	\item{traffic: hoeveelheid data verzonden en ontvangen}
	\item{plugins: aantal geactiveerde plugins}
	\item{securities: aantal beschikbare effecten}
	\item{exchanges: aantal beschikbare beurzen}
	\item{indexes: aantal beschikbare indexen}
	\item{delay: tijd tot de volgende plugin-uitvoering}
	\item{memory: geheugengebruik}
	\end{itemize}
\item{\textbf{Autorisatie}: administrator-rechten benodigd.}
\end{itemize}

\paragraph{Scraper.Restart}
\begin{itemize}
\item{\textbf{Gebruik}: de scraper herstarten.}
\item{\textbf{Parameters}: geen.}
\item{\textbf{Output}: bool die indiceert of de actie succesvol was.}
\item{\textbf{Autorisatie}: administrator-rechten benodigd.}
\end{itemize}

\paragraph{Scraper.Stop}
\begin{itemize}
\item{\textbf{Gebruik}: de scraper permanent stilleggen.}
\item{\textbf{Parameters}: geen.}
\item{\textbf{Output}: bool die indiceert of de actie succesvol was.}
\item{\textbf{Autorisatie}: administrator-rechten benodigd.}
\end{itemize}


\subsection{User-klasse}

Hier vindt men de nodige methodes terug om gebruikers te beheren. Dit is echter niet beperkt tot de administrator: ook gebruikers zelf kunnen hun eigen profiel (in beperktere mate) beheren.

\paragraph{List}

\begin{itemize}
\item{\textbf{Gebruik}: lijst met publieke informatie opvragen van de gebruikers}
\item{\textbf{Parameters}: een filter}
\item{\textbf{Output}: een lijst met structs}
	\begin{itemize}
	\item{id}
	\item{nickname}
	\item{regdate}
	\item{points}
	\end{itemize}
\item{\textbf{Autorisatie}: geen benodigd}
\end{itemize}

\paragraph{Details}

\begin{itemize}
\item{\textbf{Gebruik}: lijst met publieke informatie opvragen van de gebruikers}
\item{\textbf{Parameters}: een lijst met id's}
\item{\textbf{Output}: een lijst met structs}
	\begin{itemize}
	\item{id}
	\item{firstName}
	\item{lastName}
	\item{rrn}
	\item{cash}
	\item{startkapitaal}
	\end{itemize}
\item{\textbf{Autorisatie}: Gebruiker zelf of een gebruiker met administratorrechten}
\end{itemize}

\paragraph{Create}

\begin{itemize}
\item{\textbf{Gebruik}: aanmaken van een nieuwe gebruiker}
\item{\textbf{Parameters}: struct met de gebruikers}
	\begin{itemize}
	\item{}
	\end{itemize}
\item{\textbf{Output}: id van de aangemaakte gebruiker}
\todo{specifieke foutcode genereren} 
\item{\textbf{Autorisatie}: een gebruiker met adminstrator}
\end{itemize}

\paragraph{Modify}

\paragraph{Remove}

\paragraph{Portefolio.List}

\paragraph{Portefolio.History}

1e param is struct die range definiëert, response is array van structs (datetime, points).

\paragraph{Orders.List}

\paragraph{Orders.Create}

\paragraph{Orders.Cancel}

\paragraph{Transactions.List}


\subsection{Finance-klasse}

Tenslotte zijn er nog de methodes gerelateerd met het effectieve beurswezen, die (hoofdzakelijk) terug te vinden zijn in deze klasse. Enkel het ophalen van de portefolio bevindt zich, logischerwijs, in de User-klasse.

\paragraph{Exchange.List}

\paragraph{Exchange.Modify}

\paragraph{Index.List}

\paragraph{Index.Modify}

\paragraph{Security.List}

ook flag key
output is enkel opprevlakkige informatie om tabel te maken

\paragraph{Security.Details}

input is ID uit Security.List

\paragraph{Security.Modify}

plus "hide" flag etc

\paragraph{Security.Flag}

1e param type vlag
2e param bool voor set/unset
3e param array array van security ID's




% Invoering
%\part{Invoering}
%\label{pt:invoering}
%%
% Oracle database
%

\chapter{Databases}

De opdracht specifi\"eerde dat we ook binnen het scholencomplex een database moesten opzetten, en aangezien StockPlay belang heeft aan veel en up-to-date data moesten we een manier vinden om de gescrapete data te synchroniseren tussen de twee databases.

Omdat we voor onze databaseservers Oracle 10g Express gebruikten, waren we gelimiteerd qua replicatie-mogelijkheden. Die limitaties hielden in dat de replicatie \emph{one-way} zou zijn, en enkel data zou gerepliceerd worden (dus geen \emph{stored procedures} of \emph{views}). Hoewel dit een serieuze beperking is, besloten we om toch door te gaan met deze piste en een replicatieschema op te zetten. Hierbij zou de "master"-database die op locatie zijn, en de databaseserver binnen het scholencomplex diens data gewoon repliceren. Aangezien de data dus vloeit van de externe naar de interne database, houdt dit in dat wijzigingen die we maken aan de interne database (met andere woorden: het spelverloop) niet zouden terechtkomen in de effectieve database.


%
% Backend
%

\chapter{Backend}

\todo{builden van branched nightlies, oid}


%
% Scraper
%

\chapter{Scraper}


%
% Interfaces
%

\chapter{Interfaces}


%
% Bijlagen
%

\part{Bijlagen}
\label{pt:bijlagen}
\appendix
% Analyse
\chapter{Analyse}
\label{pt:analyse}
%Preambule met standaardinstellingen
\documentclass[a4paper,oneside]{report}
%Noot: zorg ervoor dat Nederlandse woordsplitsing geactiveerd is.
\usepackage[dutch]{babel}
% Noot: je kan het graphicxpakket een optie dvips of pdftex doorgeven
% in dat geval oet je ze ook aan iiiscriptie doorgeven, dus bijvoorbeeld
% \usepackage[dvips]{graphicx}
% \usepackage[dvips]{iiiscriptie}
\usepackage{graphicx}
\usepackage{iiiscriptie}
%Nuttig pakket voor URL's
\usepackage{url}
%extra functies door Tim
% Verkleinde margin entry
\setlength{\marginparwidth}{1.2in}
\let\oldmarginpar\marginpar
\renewcommand\marginpar[1] {\-\oldmarginpar[\raggedleft\footnotesize #1]%
{\raggedright\footnotesize #1}}

% Een TODO-entry
\newcommand{\todo}[1] {
	\addcontentsline{tdo}{todo}{\protect{#1}}
	\marginpar{#1}
}

% Een lijst van TODO-entries
\makeatletter
\newcommand \listoftodos {
	\section*{Todo list} \@starttoc{tdo}
}
\newcommand\l@todo[2] {
	\par\noindent \textit{#2}, \parbox{10cm}{#1}\par
} \makeatother

% Float defini\"eren voor codefragmenten
\usepackage{float}
\floatstyle{ruled}
\newfloat{code}{thp}{lop}
\floatname{code}{Codefragment}

% Hyperlink maken en URL in footnote tonen
\usepackage{hyperref}
\newcommand{\makeurl}[2]{\href{#2}{#1} \footnote{#2}}

% Functiedefinitie voor protocolstudie
\newcommand{\function}[5] {
	\subsubsection{#1}
	\begin{tabular}{|r p{10cm}|}
	\hline
	\textsc{Gebruik} |		& #2 \\
	\textsc{Parameters} |		& #3 \\
	\textsc{Output} |		& #4 \\
	\textsc{Autorisatie} |		& #5 \\
	\hline
	\end{tabular}
}

% Functiedefinities voor logboek handling
\usepackage{ifthen}
\newcommand{\lbdate}{}
\newcommand{\lbsetdate}[1]{
  \gdef\lbdate{#1}
}
\newcommand{\lbentry}[5] {
	\ifthenelse{\equal{\lbdate}{#1}}
	{
	}
	{
		\ifthenelse{\equal{\lbdate}{}}{}{
			& & & \\ \hline % TODO: workaround, zou niet nodig moeten zijn
			\end{tabular}
		}
		\subsection{#1}
		\begin{tabular}{|r r r p{10cm}|}
		\hline
		\textsc{Begin} & \textsc{Einde} & \textsc{Duur} & \textsc{Beschrijving} \\
		\hline
	}
	\lbsetdate{#1}
	#2 & #3 & #4 & #5 \\
}
\newcommand{\lbstop}[1] {
	& & & \\ \hline % TODO: consistentie fix, zie hierboven
	\end{tabular}
	\lbsetdate{}
	% totaal aantal uren
}

% Compacte enumeraties
\newenvironment{enumerate_compact}{
\begin{enumerate}
  \setlength{\itemsep}{1pt}
  \setlength{\parskip}{0pt}
  \setlength{\parsep}{0pt}
}{\end{enumerate}}
\newenvironment{itemize_compact}{
\begin{itemize}
  \setlength{\itemsep}{1pt}
  \setlength{\parskip}{0pt}
  \setlength{\parsep}{0pt}
}{\end{itemize}}

% Compacte environment voor use-cass
\newenvironment{compact}{\setlength{\parskip}{0pt}}{}

%stijlen voor listings
%
\lstdefinestyle{SQL}{
  breaklines=true,
  language=SQL,
  basicstyle=\normalsize,
  keywordstyle=\ttfamily\color{OliveGreen},
  identifierstyle=\ttfamily\color{CadetBlue}\bfseries,
  commentstyle=\color{Brown},
  stringstyle=\ttfamily,
  showstringspaces=true
}


\definecolor{myid}{rgb}{0.1,0.1,0.1}
\lstdefinestyle{Java}{language=java,
basicstyle=\ttfamily\normalsize,
numbers=left,stepnumber=1,numberstyle=\small\ttfamily,
numbersep=5pt,frame=tlbr,extendedchars=true,
commentstyle=\color{OliveGreen}\ttfamily,
%% stringstyle=\color{red}\ttfamily,
stringstyle=\ttfamily\color{Magenta},
keywordstyle=\ttfamily\color{Violet}\bfseries,
ndkeywordstyle=\ttfamily\color{Yellow}\bfseries,
identifierstyle=\ttfamily\color{myid},
% sensitive=false,
basicstyle=\scriptsize,
}

%
%Invullen velden voor titelpagina.
%
\departement{Departement Toegepaste Ingenieurswetenschappen}
\deptadres{Schoonmeersstraat 52 - 9000 Gent}
\studiejaar{3e Bachelor Informatica}
\soortrapport{
Analyse voor het vakoverschrijdend project Informatica
}
\title{Analyse 'Stockplay'}
\author{
Tim BESARD\\
Dieter DEFORCE\\
Laurens VAN ACKER\\
Thijs WALCARIUS
}
\begin{document}
\maketitle
\pagenumbering{roman}
\tableofcontents
\pagenumbering{arabic}
% Behoeften-analyse
\chapter{Behoefteanalyse}
Bij dit spel is het de bedoeling dat deelnemers tijdens de duur van het spel (dat loopt over enkele weken/maanden) virtueel aandelen, opties, trackers, fondsen, enz. kunnen verhandelen om zo op het einde van het spel een zo hoog mogelijk rendement neer te zetten. Hiervoor krijgt iedereen aan het begin van het spel eenzelfde budget toegewezen. 

Voorafgaand aan de start van het spel is een inschrijvingsperiode. Hierbij spelers kunnen zich registeren met behulp van hun eID. 

Het spel start op een vastgestelde dag en tijdstip: 1 maart 2010 om 8u 's morgens. Elke speler krijgt een startbudget van €100.000.
\todo{Waarom fixed registratieperiode?}
Vanaf dat moment hebben de gebruikers toegang tot hun portefolio en kunnen effecten worden verhandeld.

De gebruiker kan effecten aankopen van verschillende bronnen:
\begin{itemize}
\item{aandelen op de Continumarkt van Brussel}
\item{aandelen op de Eurolist van Parijs}
\item{aandelen op de "Lokaal" van Amsterdam}
\item{trackers op Euronext Amsterdam}
\end{itemize}

Het spel eindigt op 31 mei 2010, waarna een eindklassement wordt opgemaakt. De koersen op de website worden niet meer ge\"updatet, en er kunnen ook geen transacties meer gedaan worden. Gebruikers kunnen wel nog steeds inloggen op de website om statistieken te raadplegen.


% Functionele analyse
\chapter{Functionele analyse}
\subsection{Webapplicatie}

Deze interface wordt gebruik om deel te nemen aan het spel. Een gebruiker surft hierbij naar de server die de interface host, en krijgt direct de spelomgeving te zien, dit zonder eerst aan bepaalde softwarevereisten (zoals een Java runtime) te moeten voldaan hebben. Een deel van de functionaliteit is beperkt tot geregistreerde gebruikers, maar hier meer over later.

Eerst en vooral is er de algemene overzichtspagina. Die geeft de gebruiker een zicht over:
\begin{itemize}
\item{de huidige waarde van de aangekochte effecten en de meer/minwaarde op de aankoopprijs}
\item{cashpositie}
\item{totaal van cashpositie + huidige waarde van de effecten}
\item{huidig rendement}
\item{grafiek met een overzicht van hun rendement tov het gemiddeld rendement, beste rendement, enz}
\item{algemeen klassement, tussenklassementen}
\end{itemize}

Vervolgens kan de gebruiker zijn portefolio bekijken, en daar volgende informatie uit halen:
\begin{itemize}
\item{overzicht van het portefolio van de gebruiker}
\item{effecten momenteel in bezit, aankoopprijs (per stuk en in totaal), huidige koers, rendement en winst/verlies op het aandeel}
\end{itemize}

Er is ook een pagina die een zicht biedt op alle effecten aanwezig in het spel. Om het overzicht te behouden voorziet dat overzicht in verschillende filters:
\begin{itemize}
\item{Per beurs}
\item{Per type (aandeel, tracker, ...)}
\item{Per index}
\item{Per naam}
\item{Aandelen die als favoriet zijn gekenmerkt door de gebruiker}
\item{Per prijs}
\item{Per volume}
\end{itemize}

Per aandeel kan vervolgens doorgeklikt worden naar een overzichtspagina, die volgende informatie biedt:
\begin{itemize}
\item{grafiek met koers}
\item{hoog/laag van de dag}
\item{huidige koers}
\item{openingskoers}
\item{verschil}
\item{omzet}
\item{mogelijkheid om te kopen/te verkopen}
\end{itemize}

De gebruiker kan ook een overzicht bovenhalen waarop zijn transactiegeschiedenis zichtbaar is. Daarbij krijgt hij per transactie het volgende te zien:
\begin{itemize}
\item{het tijdstip}
\item{het effect}
\item{het type transactie}
\item{het aantal}
\item{de kostprijs voor de transactie}
\item{de winst/verlies door transactie}
\item{totaal van uw portefeuille}
\end{itemize}
Ook hier kan men steeds doorklikken naar een detailpagina in kwestie, en ook het overzicht behouden met behulp van volgende filters:
\begin{itemize}
\item{enkel aankopen/verkopen}
\item{enkel met winst/verlies}
\end{itemize}

Er zijn ook verschillende klassementen aanwezig:
\begin{itemize}
\item{Meest aangekochte aandelen}
\item{Meest verkochte aandelen}
\item{Spelers top}
\end{itemize}

De interface voorziet ook in een pagina om aandelen te kopen of verkopen. Hiervoor zijn verschillende mogelijkheden:
\begin{itemize}
\item{de prijs die je biedt voor het aandeel. Pas als het aandeel die koers bereikt wordt het aandeel effectief aangekocht}
\item{de hoeveelheid aandelen die je wenst aan te kopen}
\item{de max. geldigheidsduur van dit order (1 uur/dag/week/maand)}
\end{itemize}
Tijdens de aankoop krijgt de gebruiker een overzicht van de totale kostprijs en de verschillende taksen die erop staan.


\subsection{Desktopapplicatie}

Deze applicatie voorziet in het beheer van het hele systeem. De beheerder logt daarvoor in met behulp van zijn eID.

De desktopapplicatie is opgesplitst in drie grote componenten.
Enerzijds is er het overzicht van de gebruikers. Per gebruiker zijn de volgende beheersopdrachten mogelijk: 
\begin{itemize}
\item{wijzigen van}
\item{verwijderen van gebruiker}
\item{aanpassen van portefolio: kopen/verkopen van effecten}
\item{de hoeveelheid cash van de gebruiker aanpassen}
\end{itemize}

Er is ook een overzicht van de effecten voorzien, die de volgende functionaliteit biedt:
\begin{itemize}
\item{Aanduiden van welke effecten moeten gescraped worden, welke effecten zichtbaar zijn bij de spelers}
\item{Schorsen van handel in een effect}
\item{Wijzigen van de gescrapede gegevens}
\item{Overzicht van de aanwezigheid in portefeuilles bij spelers}
\end{itemize}

Het laatste grote deel van de applicatie biedt een overzicht van de systeemstatus. Daarbij kan de beheerder het volgende ondernemen:
\begin{itemize}
\item{Status van de componenten (scraper, database, website, ...) en starten/stoppen/herstarten van degene die dit aankunnen.}
\item{Statistieken bekijken}
	\begin{itemize}
	\item{Aantal ingeschreven gebruikers sinds de start}
	\item{Aantal gebruikers online}
	\item{Aantal connecties per tijdseenheid op de backend}
	\item{Aantal transacties per tijdseenheid}
	\item{Aantal (succesvol/gefaalde/...) gescrapede aandelen per tijdseenheid}
	\end{itemize}
\end{itemize}

\subsection{Backend}

De backend fungeert als een schil rond de database. Alle aanvragen van de desktopapplicatie en de website die informatie uit de database nodig hebben, of er naartoe willen schrijven worden afgeleid naar de backend. De communicatie tussen backend en zijn clients gebeurt mbv XML-RPC. 

\subsection{Scrapers}

De scrapers gaan periodiek enkele vooraf bepaalde sites ophalen en halen hieruit de huidige koersinformatie. Hiervoor wordt oa. handig gebruik gemaakt van de AJAX-requests die deze websites gebruiken om de koersen live te tonen. Dit zorgt ervoor dat bijna enkel de koersen worden opgehaald, zonder overhead (zoals de layout van de website, etc) 
De opgehaalde informatie wordt vervolgens doorgestuurd naar de backend dewelke deze koersinformatie vervolgens opslaat in de database. 
Optioneel: mobiele client voor PDA/Smartphone 
Spelers kunnen met behulp van hun PDA/Smartphone een gereduceerde set acties uitvoeren op hun portefolio.


\end{document}



% Ontwerp
\chapter{Ontwerp}
\label{chap:ontwerp}
%
% Functioneel
%

\chapter{Functioneel ontwerp}

\section{Webapplicatie}

\todo{Sommige opsommingen herwerken tot tekst?}
Deze interface wordt gebruik om deel te nemen aan het spel. Een gebruiker surft hierbij naar de server die de interface host, en krijgt zo de spelomgeving te zien, dit zonder eerst aan bepaalde softwarevereisten (zoals een Java runtime) te moeten voldaan hebben. Een deel van de functionaliteit is beperkt tot geregistreerde gebruikers, maar meer hierover in hoofdstuk X.

\paragraph{De algemene overzichtspagina}
Deze pagina geeft de gebruiker een zicht over:
\begin{itemize}
	\item{de huidige waarde van de aangekochte effecten en de meer- of minwaarde op de aankoopprijs}
	\item{zijn cashpositie}
	\item{het totaal van zijn cashpositie en de huidige waarde van de effecten}
	\item{het huidige rendement}
	\item{een grafiek met een overzicht van hun rendement ten opzichte van het gemiddeld rendement, beste rendement, enz.}
	\item{het algemeen klassement en eventuele tussenklassementen}
\end{itemize}

\paragraph{Portefolio}
Vervolgens kan de gebruiker zijn portefolio bekijken, en daar volgende informatie uit halen:
\begin{itemize}
	\item{een overzicht van het portefolio van de gebruiker}
	\item{de effecten momenteel in bezit, aankoopprijs (per stuk en in totaal), huidige koers, rendement en winst/verlies op het aandeel}
\end{itemize}

\paragraph{Overzicht beschikbare effecten}
Er is ook een pagina die een zicht biedt op alle effecten aanwezig in het spel.
\subparagraph{Filters} Om het overzicht te behouden voorziet dat overzicht in verschillende filters:
\begin{itemize}
  \setlength{\itemsep}{1pt}
  \setlength{\parskip}{0pt}
  \setlength{\parsep}{0pt}
	\item{per beurs}
	\item{per type (aandeel, tracker, ...)}
	\item{per index}
	\item{per naam}
	\item{aandelen die als favoriet zijn gekenmerkt door de gebruiker}
	\item{per prijs}
	\item{per volume}
\end{itemize}
\subparagraph{Details per effect}Per aandeel kan vervolgens doorgeklikt worden naar een overzichtspagina, die de volgende informatie biedt:
\begin{itemize}
  \setlength{\itemsep}{1pt}
  \setlength{\parskip}{0pt}
  \setlength{\parsep}{0pt}
	\item{grafiek met koers}
	\item{hoog/laag van de dag}
	\item{huidige koers}
	\item{openingskoers}
	\item{verschil}
	\item{omzet}
	\item{mogelijkheid om te kopen/te verkopen}
\end{itemize}

\paragraph{Transactiegeschiedenis}De gebruiker kan ook een overzicht bovenhalen waarop zijn transactiegeschiedenis zichtbaar is. 
\subparagraph{Details per transactie}Ook hier kan men steeds doorklikken naar een detailpagina in kwestie:
\begin{itemize}
  \setlength{\itemsep}{1pt}
  \setlength{\parskip}{0pt}
  \setlength{\parsep}{0pt}
	\item{het tijdstip}
	\item{het effect}
	\item{het type transactie}
	\item{het aantal}
	\item{de kostprijs voor de transactie}
	\item{de winst/verlies door transactie}
	\item{totaal van uw portefeuille}
\end{itemize}
\subparagraph{Filters}Het overzicht kan worden behouden met behulp van volgende filters:
\begin{itemize}
  \setlength{\itemsep}{1pt}
  \setlength{\parskip}{0pt}
  \setlength{\parsep}{0pt}
	\item{enkel aankopen/verkopen}
	\item{enkel met winst/verlies}
\end{itemize}

\paragraph{Klassementen}Er zijn ook verschillende klassementen aanwezig:
\begin{itemize}
  \setlength{\itemsep}{1pt}
  \setlength{\parskip}{0pt}
  \setlength{\parsep}{0pt}
	\item{meest aangekochte aandelen}
	\item{meest verkochte aandelen}
	\item{spelers top 20 (waarde portefolio)}
	\item{spelers top 20 (puntentotaal)}
\end{itemize}

\paragraph{Verhandelpagina effecten}De interface voorziet ook in een pagina om aandelen te kopen of verkopen. Hiervoor zijn verschillende mogelijkheden:
\begin{itemize}
  \setlength{\itemsep}{1pt}
  \setlength{\parskip}{0pt}
  \setlength{\parsep}{0pt}
	\item{de prijs die je biedt voor het aandeel. Pas als het aandeel die koers bereikt wordt het aandeel effectief aangekocht}
	\item{de hoeveelheid aandelen die je wenst aan te kopen}
	\item{de max. geldigheidsduur van dit order (1 uur/dag/week/maand)}
\end{itemize}
Tijdens de aankoop krijgt de gebruiker een overzicht van de totale kostprijs en de verschillende taksen die erop staan.

\section{Desktopapplicatie}
Deze applicatie voorziet in het beheer van het hele systeem. De beheerder logt daarvoor in met behulp van zijn eID.

De desktopapplicatie is opgesplitst in drie grote componenten.
\paragraph{Gebruikersbeheer}Enerzijds is er het overzicht van de gebruikers. Per gebruiker zijn de volgende beheersopdrachten mogelijk: 
\begin{itemize}
	\item{wijzigen van een gebruiker}
	\item{verwijderen van een gebruiker}
	\item{aanpassen van portefolio: kopen/verkopen van effecten}
	\item{de hoeveelheid cash van de gebruiker aanpassen}
\end{itemize}

\paragraph{Effectenbeheer}Er is ook een overzicht van de effecten voorzien, die de volgende functionaliteit biedt:
\begin{itemize}
	\item{aanduiden van welke effecten moeten gescraped worden, welke effecten zichtbaar zijn bij de spelers}
	\item{schorsen van handel in een effect}
	\item{wijzigen van de gescrapede gegevens}
	\item{overzicht van de aanwezigheid in portefeuilles bij spelers}
\end{itemize}

\paragraph{Systeemstatus}Het laatste grote deel van de applicatie biedt een overzicht van de systeemstatus. Daarbij kan de beheerder het volgende ondernemen:
\begin{itemize}
\item{status van de componenten (scraper, database, website, ...) en starten/stoppen/herstarten van degene die dit aankunnen.}
\item{statistieken bekijken}
	\begin{itemize}
	\item{aantal ingeschreven gebruikers sinds de start}
	\item{aantal gebruikers online}
	\item{aantal connecties per tijdseenheid op de backend}
	\item{aantal transacties per tijdseenheid}
	\item{aantal (succesvol/gefaalde/...) gescrapede aandelen per tijdseenheid}
	\end{itemize}
\end{itemize}

\section{Backend}

De backend fungeert als een schil rond de database. Alle aanvragen van de desktopapplicatie en de website die informatie uit de database nodig hebben, of er naartoe willen schrijven worden afgeleid naar de backend. De communicatie tussen backend en zijn clients gebeurt mbv XML-RPC. 

\section{Scrapers}

De scrapers gaan periodiek enkele vooraf bepaalde sites ophalen en halen hieruit de huidige koersinformatie. Hiervoor wordt oa. handig gebruik gemaakt van de AJAX-requests die deze websites gebruiken om de koersen live te tonen. Dit zorgt ervoor dat bijna enkel de koersen worden opgehaald, zonder overhead (zoals de layout van de website, etc) 
De opgehaalde informatie wordt vervolgens doorgestuurd naar de backend dewelke deze koersinformatie vervolgens opslaat in de database. 
Optioneel: mobiele client voor PDA/Smartphone 
Spelers kunnen met behulp van hun PDA/Smartphone een gereduceerde set acties uitvoeren op hun portefolio.


%
% Technisch
%

\chapter{Technisch ontwerp}

\section{Hergebruik}

\todo{Beschrijving van het wat en waarom van libraries}

\section{Hardware}

\todo{Een vermelding van de PDA-interface}

% Handleiding
\chapter{Handleiding}
\label{chap:installatie}
\chapter{Handleiding desktop applicatie}

De desktopapplicatie bevindt zich op de pc 'Nernst', inloggen op deze pc met gebruikers 'Administrator' en wachtwoord 'e=mc**2'. De desktopapplicatie bevindt zich in de map 'VOP Groep 7' op het bureaublad, bereikbaar met de snelkoppeling 'StockPlay administratie'.

\section{Algemene zaken}

Inloggen:
De standaardinloggegevens voor deze demo zijn "Administrator" en "chocolademousse". Je kan ook zelf een account aanmaken in gebruikersbeheer, en indien dit is gepromoveerd tot 'Administrator', dan kan je hiermee ook inloggen.

\begin{figure}[h!]
	\centering
		\includegraphics[width=\textwidth]{images/handleiding/handleiding1.gif}
	\caption{Abstract Syntax Tree van een voorbeeldfilter.}
\end{figure}

Het standaard openingsscherm is een overzicht van de status van de verschillende componenten, met enkele belangrijke statistieken erbij:
Opmerking: de logica in de 'backend' die achter de Start/Stop/Restart-knoppen zit is nog niet uitgewerkt.

\begin{figure}[h!]
	\centering
		\includegraphics[width=\textwidth]{images/handleiding/handleiding7.gif}
	\caption{Abstract Syntax Tree van een voorbeeldfilter.}
\end{figure}

\section{Beheer van effecten}

Als je op "Effectenbeheer" in de linkerkolom klikt verschijnt onderstaand scherm (het inladen van de effecten duurt een grote seconde, dus even geduld). 
In het uitschuifmenu bevinden zich enkele filters waarmee je de effecten kan filteren.

\begin{figure}[h!]
	\centering
		\includegraphics[width=\textwidth]{images/handleiding/handleiding6.gif}
	\caption{Abstract Syntax Tree van een voorbeeldfilter.}
\end{figure}

Op elk effect kan je de volgende acties uitvoeren:

\begin{itemize}
\item{Naam wijzigen (in de tabel klikken)}
\item{Type wijzigen (in de tabel klikken)}
\item{Zichtbaarheid wijzigen (in de tabel klikken, of de aandelen selecteren en werken met de knoppen onderaan)}
\item{Geschorst-zijn wijzigen (in de tabel klikken, of de aandelen selecteren en werken met de knoppen onderaan)}
\end{itemize}

Om de wijzigingen permanent te maken klik je vervolgens op de knop "Opslaan".

\section{Beheer van gebruikers}

Als je op "Gebruikersbeheer" in de linkerkolom klikt verschijnt onderstaand scherm.
Links klapt er nu ook weer een scala aan filters uit waarop je de gebruikers kan filteren.

\begin{figure}[h!]
	\centering
		\includegraphics[width=\textwidth]{images/handleiding/handleiding5.gif}
	\caption{Abstract Syntax Tree van een voorbeeldfilter.}
\end{figure}

Opmerking: In tegenstelling tot bij effectenbeheer zijn alle acties hier direct definitief!!

De mogelijk acties zijn:
\begin{itemize}
\item{Aanmaken van een nieuwe gebruiker met 'Registreer gebruiker'. Dan verschijnt onderstaand scherm}
\end{itemize}

\begin{figure}[h!]
	\centering
		\includegraphics[width=\textwidth]{images/handleiding/handleiding4.gif}
	\caption{Abstract Syntax Tree van een voorbeeldfilter.}
\end{figure}

Om de nieuwe gebruiker te registeren moet je de velden Nickname, Paswoord, Achternaam, Voornaam en Emailadres invullen. 
De velden startbedrag, cash en punten worden bij het registeren van de gebruiker in de backend ingevuld.

Het bewerken van een gebruiker, dan verschijnt een gelijkaardig veld aan dat van "Registreer gebruiker", maar met enkele verschillen:

\begin{itemize}
\item{Om het wachtwoord te veranderen, klik je op de knop "Verander paswoord", pas daarna wordt het paswoord-veld beschikbaar}
\item{De hoeveelheid cash en punten kan worden veranderd. Onze applicatie vereist echter wel dat dit gelogged wordt, en daarom verschijnt er een nieuw scherm waar je de reden moet opgeven..}
\end{itemize}

\begin{figure}[h!]
	\centering
		\includegraphics[width=\textwidth]{images/handleiding/handleiding3.gif}
	\caption{Abstract Syntax Tree van een voorbeeldfilter.}
\end{figure}

\begin{figure}[h!]
	\centering
		\includegraphics[width=\textwidth]{images/handleiding/handleiding2.gif}
	\caption{Abstract Syntax Tree van een voorbeeldfilter.}
\end{figure}

\chapter{Handleiding webapplicatie}

De website is bereikbaar op 'http://nernst/' 

Het eerste wat je moet doen is het je registeren. Dit kan door de "Register"-knop te gebruiken in de linkerzijbalk. Geef je gegevens in en druk op "Registreer". Log vervolgens in met je vers aangemaakte gegevens.

\section{Overzicht van effecten - opvragen van details}

Klik links op "Securities Overview" of een van de beurzen die aanwezig is in het spel om een lijst te krijgen van effecten.

Als je op de naam van een effect klikt krijg je een detailinformatie over het aandeel met het pie�e de resistance van Laurens: zijn grafiek.

De grafiek kent vele opties en laadt dynamisch zijn data in.
De grafiek kan:

\begin{itemize}
\item{uitzoomen, maximaal tot wanneer er geen data meer is voor}
\end{itemize}

\section{Aankopen van effecten}

Klik links op "Securities Overview" of een van de beurzen die aanwezig is in het spel om een lijst te krijgen van effecten. Klik op de "Buy" knop van het gewenste aandeel.

\chapter{Handleiding grafieken module op de website}

Standaard wordt een weergave gegeven van het beursverloop van de laatste 

% Logboeken
\chapter{Logboeken}
\label{chap:logboeken}
\section{Tim Besard}
\begin{compact}
\lbentry{dinsdag 9 februari 2010}{13:30}{15:30}{02:00}{Initi\"ele briefing over het project en groepsverdeling.}
\lbentry{dinsdag 9 februari 2010}{15:45}{17:45}{02:00}{Kennismaking, brainstorming, verkenning beschikbare databronnen, opstellen draft behoeftenanalyse.}
\lbentry{dinsdag 9 februari 2010}{18:45}{19:30}{00:45}{Proof-of-concept webcrawler voor De Tijd.}
\lbentry{woensdag 10 februari 2010}{17:00}{17:30}{00:30}{Afwerken webcrawler.}
\lbentry{vrijdag 12 februari 2010}{13:30}{17:45}{04:15}{Verdere uitbreiding van de analyse.}
\lbentry{vrijdag 12 februari 2010}{23:00}{23:30}{00:30}{Conversie van Google Doc naar \LaTeX en het maken van Google Code project.}
\lbentry{zaterdag 13 februari 2010}{11:30}{12:00}{00:30}{Herwerken tekst (toevoegen van enkele functies, en een abstract).}
\lbentry{zaterdag 13 februari 2010}{12:15}{12:30}{00:15}{Opzoekingswerk over verschillende codedocumentatiemethoden en conversie naar \LaTeX.}
\lbentry{dinsdag 16 februari 2010}{13:00}{17:45}{04:45}{Afwerken analyse en initieel ontwerp van database en klassenhi\"erarchie.}
\lbentry{woensdag 17 februari 2010}{11:30}{12:30}{01:00}{Begin van XML-RPC backend protocol en reorganisatie van document gebaseerd op cursus Systeemanalyse.}
\lbentry{woensdag 17 februari 2010}{20:30}{21:30}{01:00}{Onderverdelen van backend in verschillende pseudo-klassen en schrijven van initi\"ele documentatie.}
\lbentry{donderdag 18 februari 2010}{10:35}{12:13}{01:38}{Verder werken aan het backend protocol.}
\lbentry{vrijdag 19 februari 2010}{13:30}{17:45}{04:15}{Bespreking en afwerking van het backend-protocol.}
\lbentry{zaterdag 20 februari 2010}{10:00}{10:40}{00:40}{\LaTeX documentatie voorbije week gepusht naar SVN.}
\lbentry{zondag 21 februari 2010}{10:00}{11:00}{01:00}{Afwerken van protocoldefinitie.}
\lbentry{dinsdag 23 februari 2010}{13:30}{14:45}{01:15}{Les in verband met het opstellen van het verslag.}
\lbentry{dinsdag 23 februari 2010}{14:45}{17:45}{03:00}{Opzoekingswerk over de verschillende XML-RPC bibliotheken van Java, en wat afspraken over het gebruik van logging bibliotheken zoals log4j.}
\lbentry{woensdag 24 februari 2010}{15:00}{16:50}{01:50}{Initi\"ele code voor het XML-RPC gedeelte van de backend, en een eerste kijk naar log4j.}
\lbentry{vrijdag 26 februari 2010}{13:30}{17:45}{04:15}{Verdere implementatie van het XML-RPC gedeelte.}
\lbentry{zaterdag 27 februari 2010}{11:00}{12:00}{01:00}{Implementatie van alternatieve ServletServer zodat we interne errors kunnen opvangen, loggen, en ``gereduceerd'' naar de gebruiker te sturen.}
\lbentry{zaterdag 27 februari 2010}{12:00}{12:30}{00:30}{Dummy code die een XML-RPC struct teruggeeft.}
\lbentry{zaterdag 27 februari 2010}{13:00}{13:45}{00:45}{Documenteren van de backendklassen via Doxygen.}
\lbentry{zondag 28 februari 2010}{11:00}{12:00}{01:00}{Makefile toevoegen aan het \LaTeX gedeelte, herorganisatie implementatieklassen, en een eerste versie van de User klasse.}
\lbentry{dinsdag 2 maart 2010}{13:30}{17:45}{04:15}{Verder documenteren van XML-RPC, en de start van een Perl scraper.}
\lbentry{donderdag 4 maart 2010}{17:30}{18:54}{01:24}{Verder afwerken van de Perl scraper.}
\lbentry{vrijdag 5 maart 2010}{13:30}{17:45}{04:15}{Opzet van lokale server (installatie Oracle en Java), plus verbeteren scrape-methodiek scraper.}
\lbentry{dinsdag 9 maart 2010}{13:30}{17:45}{04:15}{Bespreken van demo behoeften, afwerken server, en meer werk aan de scraper.}
\lbentry{dinsdag 9 maart 2010}{18:00}{18:30}{00:30}{Toevoeging van ISIN-nummer aan scraper.}
\lbentry{woensdag 10 maart 2010}{14:30}{15:30}{01:00}{Herwerken van het \LaTeX document.}
\lbentry{woensdag 10 maart 2010}{15:46}{16:13}{00:27}{Business objects integreren in de backend.}
\lbentry{woensdag 10 maart 2010}{16:34}{17:19}{00:45}{Werken aan een dummy implementatie van het backendprotocol.}
\lbentry{woensdag 10 maart 2010}{18:11}{19:16}{01:05}{Design van de filters.}
\lbentry{woensdag 10 maart 2010}{19:18}{19:28}{00:10}{Documenteren van het filterdesign.}
\lbentry{woensdag 10 maart 2010}{20:00}{21:12}{01:12}{Initi\"ele parser voor filters (momenteel enkel de tokenizer) en verder afwerken van POC matching zonder de parser.}
\lbentry{donderdag 11 maart 2010}{12:15}{13:05}{00:50}{Start van parser FSM.}
\lbentry{donderdag 11 maart 2010}{17:53}{20:23}{02:30}{Opnieuw verder werken aan de filter code, nu met een initi\"ele implementatie van een variabele database-connector.}
\lbentry{vrijdag 12 maart 2010}{11:15}{12:56}{01:41}{GraphViz als debug-methode van de filter AST.}
\lbentry{vrijdag 12 maart 2010}{13:30}{15:25}{01:55}{Introductie van Data objecten in de filter.}
\lbentry{vrijdag 12 maart 2010}{15:25}{17:45}{02:20}{Afstellen van filter: toevoegen van datatypes en introduceren van excepties.}
\lbentry{vrijdag 12 maart 2010}{22:00}{22:39}{00:39}{Kleine veranderingen van de filter (herorganisatie, en extra datatypes).}
\lbentry{zaterdag 13 maart 2010}{10:30}{11:19}{00:49}{Parser afgewerkt zodanig dat we ons kunnen richten op belangrijkere zaken. Veel dingen moeten nog toegevoegd worden (relation chaining, l en rvalues), maar het werkt voldoende voor de demo.}
\lbentry{zaterdag 13 maart 2010}{11:46}{12:10}{00:24}{Integratie van filters in andere projecten (backend en BusinessObjects).}
\lbentry{zaterdag 13 maart 2010}{14:40}{16:02}{01:22}{Integratie van filters in DAO laag + interne conversie naar structs ten behoeve van XML-RPC (vermijden van codeduplicatie).}
\lbentry{zondag 14 maart 2010}{09:22}{10:15}{00:53}{Modify routines in DAO laag (fromStruct).}
\lbentry{zondag 14 maart 2010}{10:22}{10:58}{00:36}{Introductie van enkele concrete implementaties in de backend.}
\lbentry{dinsdag 16 maart 2010}{13:30}{17:45}{04:15}{Afwerken van scraper, herwerken van code met als doel een striktere API.}
\lbentry{woensdag 17 maart 2010}{14:46}{15:30}{00:44}{Werken aan scraper, opnieuw re-organisatie gebaseerd op werkelijke beursstructuur.}
\lbentry{woensdag 17 maart 2010}{18:12}{19:53}{01:41}{Parsen via shunting-yard! Algemene methodiek ook een pak verbeterd.}
\lbentry{woensdag 17 maart 2010}{20:00}{20:17}{00:17}{Verbeteren van de signatuur-controle via interface-methoden.}
\lbentry{woensdag 17 maart 2010}{20:20}{20:37}{00:17}{Dynamische instantiatie van operator en functie handler objects.}
\lbentry{woensdag 17 maart 2010}{20:53}{21:33}{00:40}{Implementatie van basis operatorprecedentie (links impliciet).}
\lbentry{donderdag 18 maart 2010}{10:00}{11:45}{01:45}{Realisatieverslag schrijven van de filters.}
\lbentry{donderdag 18 maart 2010}{13:16}{14:24}{01:08}{Parametercontrole via reflectie van statisch methoden (in tegenstelling tot de vorige interface structuur) en constructie van objecten robuuster maken via constructors.}
\lbentry{vrijdag 19 maart 2010}{13:30}{16:00}{02:30}{Afwerken van de scraper -- Parijse beurs toegevoegd en gebruik maken van hard-coded informatie die anders maar moeilijk af te leiden is.}
\lbentry{vrijdag 19 maart 2010}{16:00}{17:45}{01:45}{Documenteren van scraper.}
\lbentry{zaterdag 20 maart 2010}{10:10}{11:10}{01:00}{Wegwerken van de overbodige genericiteit, en bekijken van het singleton probleem.}
\lbentry{zaterdag 20 maart 2010}{10:10}{11:08}{00:58}{Syslog code toevoegen aan de backend.}
\lbentry{zaterdag 20 maart 2010}{13:01}{13:18}{00:00}{Debuggen van een serverside syslog problemen.}
\lbentry{zaterdag 20 maart 2010}{13:43}{15:58}{02:15}{Toevoegen van een syslog webinterface (gebruik makende van phpLogCon) + ``uitbreiding'' van de standaard syslog appender om tagging bug te verhelpen.}
\lbentry{zaterdag 20 maart 2010}{16:33}{17:02}{00:29}{Verplaatsen van libraries, logcode toevoegen aan Parser en implementatie van een initi\"ele FinanceHandler.}
\lbentry{zondag 21 maart 2010}{08:30}{09:17}{00:47}{Backend fixes.}
\lbentry{maandag 22 maart 2010}{16:15}{18:14}{01:59}{Arduino XML-RPC probeersel (blijkt onmogelijk door de SDP limitatie van de bijgeleverde TCP/IP stack).}
\lbentry{vrijdag 26 maart 2010}{13:15}{17:45}{04:30}{Integratie van scraper in backend en fixen van vele kleine bugs.}
\lbentry{vrijdag 26 maart 2010}{20:00}{22:15}{02:15}{Verder afwerken integratie en een oplossing zoeken voor de dateTime.iso8601 limitatie XML::RPC.}
\lbentry{zaterdag 27 maart 2010}{09:50}{10:43}{00:53}{Bugfixes in de backend en scraper, algemene kwaliteit gaat er op vooruit.}
\lbentry{zaterdag 27 maart 2010}{11:56}{12:24}{00:28}{Wegwerken ambigu\"iteiten van constructoren in backend DAO laag.}
\lbentry{zaterdag 27 maart 2010}{13:27}{13:47}{00:20}{Case problematiek weggewerkt, instanti\"eren van objecten verbeterd en robuuster gemaakt.}
\lbentry{zaterdag 27 maart 2010}{13:47}{13:55}{00:08}{Updaten van de testsuite.}
\lbentry{zaterdag 27 maart 2010}{14:10}{15:13}{01:03}{Verbeteren van de manier waarop tijdzones gehanteerd worden binnen de scraper.}
\lbentry{zaterdag 27 maart 2010}{18:47}{20:01}{01:14}{Statistieken toevoegen in de System klasse.}
\lbentry{dinsdag 30 maart 2010}{13:30}{17:45}{04:15}{Bedrijfsbezoek KBC.}
\lbentry{vrijdag 2 april 2010}{13:30}{17:45}{04:15}{Literatuur opzoeken voor implementatie AI, en enig werk aan de scraper.}
\lbentry{vrijdag 2 april 2010}{19:05}{19:45}{00:40}{Lezen van paper (comparing artificial intelligence systems for stock portfolio selection).}
\lbentry{zaterdag 3 april 2010}{14:06}{14:45}{00:39}{Test-suite filters maken.}
\lbentry{zaterdag 3 april 2010}{16:02}{17:29}{01:27}{Lezen van papers (capital market applications of neural networks, fuzzy logic and genetic algorithms en stock market prediction using artifial neural networks).}
\lbentry{zondag 4 april 2010}{16:05}{16:40}{00:35}{Lezen van paper (using neural networks to forecast stock market prices).}
\lbentry{maandag 5 april 2010}{10:35}{11:45}{01:10}{Lezen van lectuur omtrent constructie neurale netten.}
\lbentry{maandag 5 april 2010}{12:25}{14:00}{01:35}{Zoeken van bibliotheek voor gebruik van neurale netten (uiteindelijk gekozen: Flood).}
\lbentry{maandag 5 april 2010}{15:00}{16:24}{01:24}{Neuraal net construeren (input data kiezen, datasets construeren, historische data om te leren opzoeken).}
\lbentry{maandag 5 april 2010}{17:20}{19:10}{01:50}{Zoeken van XMLRPC library om te gebruiken uit C.}
\lbentry{dinsdag 6 april 2010}{11:10}{13:45}{02:35}{Werken aan de scraper (foutafhandeling, indexen, bugfixes, netwerk-optimalisaties, CAC40 en BEL20 toegevoegd, ...)}
\lbentry{dinsdag 6 april 2010}{14:02}{15:06}{01:04}{Migratie van de backend naar het Tomcat framework.}
\lbentry{dinsdag 6 april 2010}{19:00}{19:30}{00:30}{Kleine statistiekenpagina toegevoegd.}
\lbentry{woensdag 7 april 2010}{10:28}{13:25}{02:57}{Implementeren van een Datum object in Parser, schrijven vna een syntaxreferentie voor de filters, toevoegen van een debug pagina, en wegwerken van bugs.}
\lbentry{woensdag 7 april 2010}{14:25}{15:43}{01:18}{One-time scraper voor historische data maken en bulk quote update routines implementeren (factor 15 snelheidswinst).}
\lbentry{woensdag 7 april 2010}{18:16}{20:27}{02:11}{Porten van scraper naar een alternatieve library (RPC::XML) en toevoegen van DAO achtige laag.}
\lbentry{donderdag 8 april 2010}{21:00}{00:00}{03:00}{Deployen van Tomcat.}
\lbentry{vrijdag 9 april 2010}{11:00}{13:00}{02:00}{Server toch omzetten naar Tomcat 6 met de Sun JRE.}
\lbentry{vrijdag 9 april 2010}{14:14}{14:52}{00:38}{Scraper fixes.}
\lbentry{vrijdag 9 april 2010}{15:12}{16:35}{01:23}{Compressie van requests implementeren.}
\lbentry{vrijdag 9 april 2010}{18:22}{19:55}{01:33}{Afwerken compressie van requests.}
\lbentry{maandag 12 april 2010}{09:00}{12:00}{03:00}{Finaliseren deployment backend.}
\lbentry{maandag 12 april 2010}{13:00}{17:00}{04:00}{Afwerken van de scraper.}
\lbentry{maandag 12 april 2010}{19:00}{20:00}{01:00}{Deployment van scraper.}
\lbentry{dinsdag 13 april 2010}{15:37}{16:57}{01:20}{Syslog logging aan de scraper toevoegen.}
\lbentry{dinsdag 13 april 2010}{17:05}{17:36}{00:31}{Datum fixes in scraper en backend.}
\lbentry{woensdag 14 april 2010}{11:00}{12:20}{01:20}{Deployment en bugfixes scraper.}
\lbentry{donderdag 15 april 2010}{10:00}{11:00}{01:00}{Verbeterde foutafhandeling, en verbeteren deployment scraper door er een CPAN compatibele module van te maken.}
\lbentry{donderdag 15 april 2010}{19:30}{21:00}{01:30}{Data visualiseren en bugs fixen}
\lbentry{vrijdag 16 april 2010}{10:00}{11:04}{01:04}{Scraper robuuster maken.}
\lbentry{vrijdag 16 april 2010}{11:54}{13:43}{01:49}{Parser precedentie verbeteren waardoor operatoren makkelijker in gebruik worden.}
\lbentry{vrijdag 16 april 2010}{14:04}{14:52}{00:48}{Bugs in de parser fixen.}
\lbentry{vrijdag 16 april 2010}{14:55}{15:47}{00:52}{Nieuwe filters gebruiken in scraper, enkele bugs fixen, en code robuuster maken voor moest de service wegvallen.}
\lbentry{vrijdag 16 april 2010}{17:49}{19:24}{01:35}{AI implementeren.}
\lbentry{vrijdag 16 april 2010}{20:34}{21:39}{01:05}{AI implementatie afwerken.}
\lbentry{zaterdag 17 april 2010}{10:40}{11:30}{00:50}{PK indexen van autogen-ID naar ISIN veranderen in database en backend.}
\lbentry{zaterdag 17 april 2010}{11:45}{12:11}{00:26}{UserSecurity en IndexSecurity handlers toevoegen.}
\lbentry{zaterdag 17 april 2010}{12:16}{13:08}{00:52}{Indexen toevoegen aan het Perl framework.}
\lbentry{zaterdag 17 april 2010}{13:35}{14:32}{00:57}{Afwerken van het Perl framework, en integreren van de AI.}
\lbentry{zaterdag 17 april 2010}{14:35}{15:15}{00:40}{Werk aan de AI.}
\lbentry{maandag 19 april 2010}{18:46}{20:47}{02:01}{Herstructurering van de AI.}
\lbentry{dinsdag 20 april 2010}{13:30}{17:45}{04:15}{Deployment van backend en scraper + introductie reguliere expressies in parser.}
\lbentry{woensdag 21 april 2010}{14:39}{15:27}{00:48}{Herstructureren van Perl codebase.}
\lbentry{woensdag 21 april 2010}{15:27}{16:29}{01:02}{Bugfixes in backend, consistent maken van perl scripts.}
\lbentry{woensdag 21 april 2010}{20:48}{22:00}{01:12}{Aanvullen documentatie perl framework.}
\lbentry{donderdag 22 april 2010}{11:00}{12:12}{01:12}{Ontwikkelen van nieuwe functie die quotes aan bepaalde resolutie kan ophalen.}
\lbentry{vrijdag 23 april 2010}{19:47}{20:39}{00:52}{Toevoegen van configuratie van perl framework.}
\lbentry{zaterdag 24 april 2010}{10:00}{11:30}{01:30}{Host spelen.}
\lbentry{zaterdag 24 april 2010}{11:30}{13:00}{01:30}{Testen.}
\lbentry{zaterdag 24 april 2010}{16:30}{17:00}{00:30}{Type fixes en updaten testsuite}
\lbentry{maandag 26 april 2010}{16:00}{17:00}{01:00}{Overzichtspagina versnellen.}
\lbentry{dinsdag 27 april 2010}{13:30}{17:45}{04:15}{Bespreken taakverdeling, werken aan perl framework, debuggen backend.}
\lbentry{dinsdag 27 april 2010}{19:00}{19:45}{00:45}{perl POD fixes en deployment.}
\lbentry{woensdag 28 april 2010}{15:33}{17:33}{02:00}{Werken aan AI.}
\lbentry{woensdag 28 april 2010}{11:00}{12:30}{01:30}{Fixes aan webinterface + introduceren namespaces.}
\lbentry{woensdag 28 april 2010}{21:00}{22:00}{01:00}{Es gespeeld met ASP.NET onder Mono.}
\lbentry{donderdag 29 april 2010}{18:00}{19:00}{01:00}{Verder modulariseren AI.}
\lbentry{donderdag 29 april 2010}{21:45}{22:33}{00:48}{Implementeren Latest Quotes in scraper + maken initi\"le portfolio.}
\lbentry{vrijdag 30 april 2010}{13:30}{17:45}{04:15}{Toevoegen van cache aan de backend.}
\lbentry{vrijdag 30 april 2010}{18:02}{20:13}{02:11}{Caching afwerken.}
\lbentry{zondag 2 mei 2010}{10:30}{13:00}{02:30}{Cache monitor.}
\lbentry{zondag 2 mei 2010}{14:00}{17:47}{03:47}{Meer werk aan cache en backend (hot reload, cache clearing, etc) + documentatie schrijven.}
\lbentry{maandag 3 mei 2010}{11:00}{12:00}{01:00}{Toevoegen van native filter compiler + fixes aan testsuite.}
\lbentry{maandag 3 mei 2010}{18:30}{19:30}{01:00}{Verbeterde toString weergave van filters.}
\lbentry{maandag 3 mei 2010}{19:30}{21:45}{02:15}{Documentatie.}
\lbentry{dinsdag 4 mei 2010}{13:30}{15:15}{01:45}{Documentatie toevoegen aan backend.}
\lbentry{dinsdag 4 mei 2010}{15:15}{17:45}{02:30}{Fixen van bug in scraper + toevoegen van index quotes.}
\lbentry{vrijdag 7 mei 2010}{13:30}{17:45}{04:15}{Afwerken security framework.}
\lbentry{vrijdag 7 mei 2010}{18:00}{19:00}{01:00}{Afwerken security framework.}
\lbentry{zondag 9 mei 2010}{15:30}{16:00}{00:43}{Backend bugfixes.}
\lbentry{maandag 10 mei 2010}{18:00}{18:43}{00:43}{Fixen van autorisatie/authenticatie document + verbeterde error messags.}
\lbentry{dinsdag 11 mei 2010}{13:30}{17:45}{04:15}{Schrijven installer Backend.}
\lbentry{woensdag 12 mei 2010}{18:00}{19:00}{01:00}{Fixen span query.}
\lbentry{dinsdag 13 april 2010}{13:00}{16:00}{03:00}{Overlezen documentatie en fixen van bugs.}
\lbentry{vrijdag 14 mei 2010}{11:00}{14:00}{03:00}{Fixen scraper en invoegen logboeken.}
\lbentry{vrijdag 14 mei 2010}{18:00}{18:30}{00:30}{Fixen namespaces website.}
\lbentry{vrijdag 14 mei 2010}{19:30}{21:00}{01:30}{Verbeteren van verslag, verwijderen van redundante \LaTeX opties.}
\lbentry{vrijdag 14 mei 2010}{21:00}{21:30}{00:30}{Overlezen logboek entries en herschrijven hergebruik sectie van Perl XML-RPC modules.}
\lbentry{vrijdag 14 mei 2010}{21:30}{22:00}{00:30}{Compacter maken van \LaTeX document.}
\lbentry{vrijdag 15 mei 2010}{08:30}{09:10}{00:40}{Uitdiepen van de handleiding en toevoegen van extra screenshots.}
\lbentry{vrijdag 15 mei 2010}{09:45}{10:45}{01:00}{Meer werk aan de protocoldefinitie in het \LaTeX verslag.}
\lbentry{vrijdag 15 mei 2010}{12:00}{15:15}{03:15}{Afwerken van de protocoldefinitie in het \LaTeX verslag.}
\lbentry{zondag 16 mei 2010}{11:30}{12:45}{01:15}{Fixen van hi\"erarchie-foutje in \LaTeX document, verwijderen van redundante methode uit webinterface, en toevoegen van gebruikers en orders aan het Perl framework.}
\lbentry{zondag 16 mei 2010}{15:30}{16:30}{01:00}{Finaliseren deployment-code van het Perl framework, en nog eens code reviewen met behulp van Perl::Critic.}
\lbentry{zondag 16 mei 2010}{13:00}{19:30}{00:30}{Gebruik maken van code floats voor triggers, en omzetten van verbatims uit XML-RPC specificatie naar lstlistings.}
\lbentry{maandag 17 mei 2010}{12:30}{12:50}{00:20}{Toevoegen van plannings-documentatie.}
\lbentry{maandag 17 mei 2010}{18:00}{18:25}{00:25}{Contracten omgezet naar \LaTeX formaat.}
\lbentry{maandag 17 mei 2010}{18:30}{19:45}{01:15}{Schoonheidsfoutjes in \LaTeX verslag wegwerken, waaronder het refactoren van de triggers.}
\lbentry{maandag 17 mei 2010}{20:20}{20:40}{00:20}{Toevoegen van weakness analyse wat betreft de kunstmatige intelligentie.}
\lbstop{0}
\end{compact}

\section{Dieter Deforce}
\begin{compact}
\lbentry{dinsdag 9 februari 2010}{13:30}{15:30}{02:00}{Presentatie vakoverschrijvend project bijwonen }
\lbentry{dinsdag 9 februari 2010}{15:30}{17:45}{02:15}{Brainstormen over mogelijke onderwerpen en deel van de behoefteanalyse maken }
\lbentry{vrijdag 12 februari 2010}{13:30}{17:45}{04:15}{Verderwerken aan de behoefteanalyse. Voorstellen en verdedigen van projectvoorstel }
\lbentry{dinsdag 16 februari 2010}{13:00}{17:45}{04:45}{Afwerken behoefteanalyse. Maken van de functionele analyse en verdelen van resterende taken onder de groepsleden }
\lbentry{donderdag 18 februari 2010}{09:00}{11:00}{02:00}{Visual Paradigm installeren en uitproberen. Tekenen van use-case diagram met VP. Uitschrijven van verschillende gebruiksgevallen van de desktopapplicatie.}
\lbentry{vrijdag 19 februari 2010}{13:30}{17:45}{04:15}{Bekijken hoe de back-end en XML-RPC protocol kunnen geimplementeerd worden. Starten met prototype van de geanimeerde menubalk voor de desktopapplicatie.}
\lbentry{dinsdag 23 februari 2010}{13:30}{14:30}{01:00}{Presentatie over het schrijven van een bachelor/master-proef door Kathleen Pollefliet }
\lbentry{dinsdag 23 februari 2010}{14:30}{17:45}{03:15}{Bespreken van project met de docenten en vastleggen van de functionaliteit die aanwezig moet zijn in het prototype. Verderwerken aan menubalk.}
\lbentry{woensdag 24 februari 2010}{09:00}{10:30}{01:30}{Prototype van menubalk proberen af te werken. Wegens problemen met de layoutmanagers van Swing, ben ik opnieuw begonnen met eigen paint-code.}
\lbentry{vrijdag 26 februari 2010}{11:00}{12:00}{01:00}{Opzoeken van bestaande biblithoken met controls voor Java Swing. De SwingX-bibliotheek van swinglabs.org heeft gelijkaardige geanimeerde controls aan diegene die we nodig hebben.}
\lbentry{vrijdag 26 februari 2010}{13:30}{17:45}{04:15}{Uitproberen SwingX-bibliotheek. Menucontrol aanpassen naar wens. Starten met het maken van de interfacepanelen van de desktopapplicatie. Beginnen aan de tablemodels voor gebruikers en effecten.}
\lbentry{zondag 28 februari 2010}{11:00}{13:00}{02:00}{Downloaden van Java ME SDK van Samsung en Sun. Proberen beide SDK's op te zetten en een eenvoudig programma werkend te krijgen op mijn GSM.}
\lbentry{maandag 1 maart 2010}{19:45}{22:45}{03:00}{Enkele tutorials over Java ME doorlezen. Een paar interfacepanelen aanmaken en uitproberen op mijn GSM. Uitproberen van eigen paint-code in Java ME.}
\lbentry{dinsdag 2 maart 2010}{13:30}{17:45}{04:15}{Beginnen aan de tablemodels voor de desktopapplicatie. Problemen bespreken met begeleiders. Samen met groepsleden verdere takenverdeling plannen.}
\lbentry{woensdag 3 maart 2010}{16:00}{18:30}{02:30}{Afwerken van tablemodels voor de desktopapplicatie en verder aanpassen van de menubalk. Samenbrengen van menubalk en interfacepanelen in een samenhangende GUI.}
\lbentry{vrijdag 5 maart 2010}{13:30}{17:45}{04:15}{Interfaceklassen aanmaken in ASP.NET applicatie. Overzichtstabel van effecten maken met dummygegevens. Bespreken van mogelijke uitbreidingen van het project met algoritmen (computerspelers mogelijk maken).}
\lbentry{dinsdag 9 maart 2010}{13:30}{17:45}{04:15}{Omzetten van het HTML/CSS design voor de website naar de ASP.NET website.}
\lbentry{donderdag 11 maart 2010}{09:00}{10:00}{01:00}{Herschrijven van enkele use cases voor de desktopapplicatie op vraag van de leerkrachten.}
\lbentry{vrijdag 12 maart 2010}{13:30}{17:45}{04:15}{Aanmaken DataAccess interface. Connectie opzetten met de Oracle-server op school en weergeven van gegevens op de website die uit de databank gehaald worden.}
\lbentry{maandag 15 maart 2010}{13:30}{17:45}{04:15}{Methodes uit de DataAccess interface verder uitschrijven in ADO.NET. Proberen om CSS-opmaak toe te passen op een Gridview in ASP}
\lbentry{maandag 15 maart 2010}{21:00}{22:00}{01:00}{Uitzoeken hoe de nutteloze HTML-tags kunnen verwijderd worden uit de ASP server controls. Dit toepassen op een gridview.}
\lbentry{dinsdag 16 maart 2010}{15:30}{18:30}{03:00}{Connectiestring toevoegen aan ADO.NET om verbinding te kunnen maken met Oracleserver van Laurens. Invullen van enkele dummywaardes in de databank. Verder uitzoeken hoedat je meer controle kunt krijgen over de HTML-output van ASP server controls. Documentatie lezen van de "Css-friendly control adapters" van Micrsoft. Proberen deze adapters te integereren met ons project.}
\lbentry{dinsdag 16 maart 2010}{20:00}{21:30}{01:30}{De CSS-friendly controls blijken toch niet zo "vriendelijk" te zijn. Na verder opzoekwerk blijkt dat je de HTML-output van een ASP control kunt herdefinieren met behulp van een HtmlTextWriter. Deze techniek heb ik uitgeprobeerd op een gridview. Momenteel is de output hiervan nogal gebrekkig, maar het functioneert.}
\lbentry{woensdag 17 maart 2010}{11:00}{12:00}{01:00}{De HtmlTextWriter verder uitwerken zodat de output correct kan werken met de CSS-stylesheet. Ik probeer ook de indentatie in orde te krijgen maar dit blijkt niet mogelijk te zijn}
\lbentry{woensdag 17 maart 2010}{20:00}{22:00}{02:00}{Invoegen van de SiteMapPath aan onze website. Ook deze heeft heel gebrekkige HTML-output en hiervoor heb ik ook een adapter geschreven. Oplossen van een bug met het effectenoverzicht.}
\lbentry{vrijdag 19 maart 2010}{13:30}{17:45}{04:15}{Infosessie over de masterproef. Aanmaken van de detailpagina van een effect. Sitemappath aanpassen zodat deze ook pagina's kan tonen die niet in de SiteMapDataSource vermeld staan.}
\lbentry{zondag 21 maart 2010}{11:00}{12:30}{01:30}{Database query schrijven in ADO.NET om de nodige detailinformatie van een effect op te vragen. Oplossen van probleem met het opvullen van de parameters van deze query (bepaalde variabelenamen zijn gereserveerde woorden bij de Oracle provider)}
\lbentry{zondag 21 maart 2010}{15:00}{18:00}{03:00}{Overzichtspagina van effecten aanpassen zodat de percentages correct ingekleurd worden. Herschikken van de kolommen en toevoegen van enkele gegevens aan de tabel. Opmaak van de detailpagina aanpassen. Extra informatie toevoegen voor een effect.}
\lbentry{dinsdag 23 maart 2010}{13:30}{17:45}{04:15}{Voorbereiden van presentatie van de demo, er waren enkele problemen met het verbinden met de databank. Presentatie geven. Bespreken wat er nog allemaal gedaan moet worden en taken verdelen.}
\lbentry{donderdag 25 maart 2010}{09:00}{10:00}{01:00}{Verslag doornemen en fouten aanduiden.}
\lbentry{vrijdag 26 maart 2010}{10:30}{11:30}{01:00}{\LaTeX en TEXnicCenter installeren en bekijken hoe \LaTeX-documenten zijn opgebouwd.}
\lbentry{vrijdag 26 maart 2010}{13:30}{17:45}{04:15}{Backend uitproberen. Code van backend lezen en proberen te begrijpen. Werking van XML-RPC bekijken en uitproberen. Schrijven van een JUnit test voor de XML-RPC laag.}
\lbentry{zaterdag 27 maart 2010}{09:00}{11:00}{02:00}{Verslag verder lezen en nog enkele fouten verbeteren. Fouten aanpassen in \LaTeX-editor en commiten.}
\lbentry{dinsdag 30 maart 2010}{13:30}{17:45}{04:15}{Bedrijfsbezoek KBC}
\lbentry{donderdag 1 april 2010}{20:00}{22:20}{02:20}{Login pagina toevoegen. Aanpassingen van de databank doorvoeren in ADO.NET. ASP website aanpassen om nu overal ISIN nummers te gebruiken als identificatie van effecten.}
\lbentry{vrijdag 2 april 2010}{13:30}{17:45}{04:15}{Enkele problemen met de Loginpagina oplossen. Binary van XML-RPC.NET importeren in het project. Dummy membershipprovider aanmaken en bekijken hoe deze in elkaar zit.}
\lbentry{zondag 4 april 2010}{20:00}{22:00}{02:00}{Proberen een verbinding op te zetten met de back-end via XML-RPC.NET. Jammergenoeg zit er een ontwerpfout in de back-end waardoor bijna alle functies fouten geven: XML-RPC ondersteunt geen nullwaardes en hier wordt nergens rekening mee gehouden. Teamleden op de hoogte stellen van probleem en wachten op een oplossing.}
\lbentry{maandag 5 april 2010}{16:00}{17:00}{01:00}{Klassenstructuur voor de XML-RPC datalaag maken in Visual Studio.}
\lbentry{dinsdag 6 april 2010}{11:00}{12:00}{01:00}{Huidige datalaag herschikken om performantie te verbeteren. Andere code ook refactoren.}
\lbentry{dinsdag 6 april 2010}{15:00}{18:30}{03:30}{Verder refactoren en gemaakte wijzigingen in de structuur doorvoeren in de rest van het project. Performantieproblemen van ADO.NET laag oplossen (aangezien de huidige structuur ook ongeschikt is voor XML-RPC). Handlerklassen maken die de methodes van XML-RPC vertalen naar C\# methodes.}
\lbentry{woensdag 7 april 2010}{10:00}{11:30}{01:30}{Factory maken voor de datalaag zodat makkelijk wisselen tussen ADO en XML-RPC mogelijk is. Interfaces van de datalaag uitschrijven voor XML-RPC. De nodige methodes om het effectenoverzicht te maken werden eerst geschreven. Hier komen ook enkele bugs in de back-end naar boven en deze moet ik ook oplossen.}
\lbentry{donderdag 8 april 2010}{14:30}{18:30}{04:00}{Constructors schrijven die objecten kunnen maken uit XmlRpcStructs. User klasse schrijven en de nodige methodes die betrekking hebben op inloggen en registreren schrijven. MembershipProvider uitschrijven en uittesten met behulp van een CreateUser-control van ASP. Email veld toevoegen aan een User in backend en databank. }
\lbentry{vrijdag 9 april 2010}{20:30}{22:00}{01:30}{Login mogelijk maken via backend (methode in backend maken die validatie mogelijk maakt). Fouten in paswoordencryptie zoeken samen met Thijs.}
\lbentry{zaterdag 10 april 2010}{10:00}{12:00}{02:00}{Transactiepagina maken. IIS installeren en configureren op laptop en desktop.}
\lbentry{zaterdag 10 april 2010}{14:00}{18:00}{04:00}{Enkele problemen met de MembershipProvider oplossen Website proberen te deployen. Wegens het ontbreken van een 64-bit Oracle Data Provider werkt deze enkel op mijn 32-bit laptop, maar niet op mijn 64-bit desktop. SMTP server opstellen binnen IIS zodat de gebruiker een email ontvangt wanneer hij zich registreert. Omdat de SMTP poort door de meeste providers geblokkeerd wordt is het eenvoudiger om gewoon de SMTP server van de provider/school te gebruiken.}
\lbentry{maandag 12 april 2010}{09:00}{12:00}{03:00}{Status van het project bespreken met groepsleden. Vooruitgang van de verschillende componenten bekijken. Enkele bugs oplossen die nog naar boven komen in de website. Log4NET proberen te configureren}
\lbentry{maandag 12 april 2010}{12:30}{14:00}{01:30}{Log4NET doen loggen naar de Syslog van be03.kapti.com.}
\lbentry{maandag 19 april 2010}{20:00}{23:30}{03:30}{Website klaarmaken voor de deadline van dinsdag: - Verkooporders mogelijk maken - 'Forgot password'-functie toevoegen - CreateUser-wizard vervangen door eigen formulier - Overview-pagina voor de gebruiker maken met zijn belangrijkste informatie en een formulier toevoegen om zijn accountgegevens up te daten - Gegevens portfoliopagina uitbreiden}
\lbentry{dinsdag 20 april 2010}{12:00}{13:15}{01:15}{Tekst en uitleg typen bij de verschillende pagina's van de website. Helppagina toevoegen met wat info over het project}
\lbentry{dinsdag 20 april 2010}{13:30}{17:30}{04:00}{Wijzigingen in CSS om weergave te verbeteren Sell order pagina uitbreiden Zoekfunctie implementeren samen met Laurens en Tim Problemen met javascript oplossen. Bugs in de orderpagina's oplossen en controleren ofdat orders correct worden omgezet in transacties}
\lbentry{zaterdag 24 april 2010}{10:00}{13:30}{03:30}{Projecten van andere groepen uitproberen en hosten opendeurdag}
\lbentry{zondag 25 april 2010}{18:00}{19:30}{01:30}{Internationalisering in ASP.NET bekijken. Uitproberen op een pagina}
\lbentry{maandag 26 april 2010}{20:00}{22:15}{02:15}{Internationalisering doorvoeren op de andere pagina's van de website en tegelijkertijd de vertaling naar het nederlands doen. Problemen met de links proberen op te lossen.}
\lbentry{dinsdag 27 april 2010}{08:15}{09:00}{00:45}{Nog wat verderwerken aan vertaling en werk tot nu toe committen.}
\lbentry{dinsdag 27 april 2010}{13:30}{17:45}{04:15}{Bespreken met groepsleden wat er nog allemaal moet gedaan worden. Laatste pagina's internationaliseren en vertalen. Uitzoeken hoe de sitemap moet vertaald worden. Pagina's overlopen en kijken of er geen tekst ontbreekt en de kapotte links fixen. Zoeken hoe de ASP.NET website vanop afstand kan herstart worden. Implemantie van herstartpagina}
\lbentry{woensdag 28 april 2010}{16:00}{18:00}{02:00}{Logging toevoegen op verschillende plaatsen in de ASP.NET website Nog enkele bugs op verschillende pagina's oplossen en fouten in de vertaling verbeteren.}
\lbentry{vrijdag 30 april 2010}{13:30}{17:45}{04:15}{Begin van het puntenbeheer. Aanmaken van de pointmanager en bekijken van de SPClient die toegang geeft tot de backend. Uitproberen van enkele aanvragen via SPClient en bugs oplossen die hierbij naar boven komen.}
\lbentry{zondag 2 mei 2010}{16:00}{19:00}{03:00}{Verder uitwerken van de pointsmanager. Proberen de winst van spelers terug te rekenen aan de hand van de transacties en hun portfolio. Na een tijdje blijkt dat dit toch niet zo eenvoudig is en bespreek ik met de anderen ofdat we beter niet een extra tabel maken in de databank met de cashposities van de spelers. Uiteindelijk lukt het toch nog om betrouwbaar de winst van de voorbije dag te berekenen (alhoewel het weinig performant is).}
\lbentry{dinsdag 4 mei 2010}{13:30}{17:45}{04:15}{Controleren ofdat de winstberekening nog altijd correct is ook als er complexe koop/verkooporders op dezelfde dag plaatsvinden. Structuur van pointsmanager verbeteren en enkele fouten oplossen in berekeningen. Enkele bugs oplossen in backend, website en SPClient}
\lbstop{0}
\end{compact}

\section{Laurens Van Acker}
\begin{compact}
\lbentry{dinsdag 9 februari 2010}{13:30}{15:30}{02:00}{Project omschrijving, groepsverdeling en vragen}
\lbentry{dinsdag 9 februari 2010}{15:45}{17:45}{02:00}{Kennismaking, brainstorming, opzoeken datafeeds}
\lbentry{vrijdag 12 februari 2010}{18:00}{18:15}{00:15}{Bewerken behoefteanalyse}
\lbentry{zaterdag 13 februari 2010}{17:00}{18:15}{01:15}{Installatie database, ftp hosting en crontab, aanmaken database}
\lbentry{zaterdag 13 februari 2010}{18:15}{18:30}{00:15}{Opstellen werkuren fiche}
\lbentry{dinsdag 16 februari 2010}{13:10}{18:00}{04:50}{GUI uittekenen, databasestructuur en klassenstructuur bepalen}
\lbentry{vrijdag 12 februari 2010}{13:30}{17:45}{04:15}{Eerste evalutie, bespreking en opstellen van analyse}
\lbentry{donderdag 18 februari 2010}{12:00}{12:30}{00:30}{Mookup website}
\lbentry{donderdag 18 februari 2010}{20:20}{00:00}{03:40}{Use cases website: eerste draft opstellen}
\lbentry{vrijdag 19 februari 2010}{13:30}{17:45}{04:15}{Bespreking en afwerking van het backend-protocol.}
\lbentry{zondag 21 februari 2010}{23:00}{23:15}{00:15}{Herwerken tekst}
\lbentry{maandag 22 februari 2010}{19:00}{19:30}{00:30}{Nalezen tekst realisatie en ontwerp.tex}
\lbentry{dinsdag 23 februari 2010}{13:30}{14:30}{01:00}{Marleen Polfliet: Tips bij het schrijven van je Masterproef}
\lbentry{dinsdag 23 februari 2010}{14:30}{17:45}{03:15}{Verdere afwerking protocol, bespreking aangemaakte documenten, evalutatie bij de jury}
\lbentry{dinsdag 23 februari 2010}{21:00}{23:30}{02:30}{Nalezen en herwerken tekst realisatie en ontwerp.tex}
\lbentry{donderdag 25 februari 2010}{14:00}{14:15}{00:15}{Installatie Oracle}
\lbentry{donderdag 25 februari 2010}{19:30}{20:00}{00:30}{Opstellen contract}
\lbentry{vrijdag 26 februari 2010}{11:00}{12:00}{01:00}{Opzoeken bestaande bibliotheken met controls voor Java Swing zoals de SwingX bibliotheek, korte bespreking, onderzoek bibliotheek Plot}
\lbentry{vrijdag 26 februari 2010}{13:30}{17:45}{04:15}{Korte bespreking status, Programmatie Flot statistiekenmodule voor de website, openen poorten FireWall}
\lbentry{dinsdag 2 maart 2010}{13:30}{17:45}{04:15}{Implementatie design, grafiek bibiotheek, evalutatie, }
\lbentry{dinsdag 2 maart 2010}{19:00}{20:00}{01:00}{Implementatie design}
\lbentry{woensdag 3 maart 2010}{17:20}{18:20}{01:00}{Aanmaken en schrijven modules.tex}
\lbentry{donderdag 4 maart 2010}{20:45}{23:00}{02:15}{Installatie server, verder afwerken modules.tex, verder werken design, installatie Visual Studio}
\lbentry{vrijdag 5 maart 2010}{11:00}{12:15}{01:15}{Verder werken design template}
\lbentry{vrijdag 5 maart 2010}{13:30}{17:45}{04:15}{Design template, configureren Oracle server, samen overlopen}
\lbentry{zaterdag 6 maart 2010}{19:00}{21:00}{02:00}{Design}
\lbentry{zondag 7 maart 2010}{14:50}{15:30}{00:40}{Eerste visie van finaal contract opgesteld}
\lbentry{dinsdag 9 maart 2010}{13:30}{17:30}{04:00}{Bespreking, asp website js fouten, oracle configuratie, }
\lbentry{vrijdag 12 maart 2010}{13:30}{17:30}{04:00}{Bespreking, aanvullen en bewerken use cases, filterdemo}
\lbentry{vrijdag 12 maart 2010}{17:00}{17:30}{00:30}{Opzetten server thuis}
\lbentry{zondag 14 maart 2010}{13:00}{16:00}{03:00}{Tussenlaag DOA en XML-RPC een aanpassingen DOA}
\lbentry{donderdag 18 maart 2010}{21:45}{00:00}{02:15}{Javascript grafieken verder werken}
\lbentry{vrijdag 19 maart 2010}{10:30}{12:45}{02:15}{Aanpassingen in de tussenlaag, omzetten van String objecten die de XML-RPC krijgt naar Filter objecten}
\lbentry{vrijdag 19 maart 2010}{13:30}{17:45}{04:15}{Infosessie masterproef, invullen peer-evalution, aanvullen van condities in filter, testen van filter en oplossen van foutjes in filter}
\lbentry{vrijdag 19 maart 2010}{20:00}{21:30}{01:30}{Filter condities toevoegen}
\lbentry{zondag 21 maart 2010}{21:00}{23:00}{02:00}{Voorbereiden demosessie, aanmaken slided}
\lbentry{dinsdag 23 maart 2010}{00:00}{00:45}{00:45}{Wijzingen in presentatie, afdrukken, etc..}
\lbentry{dinsdag 23 maart 2010}{13:30}{17:45}{04:15}{Overlopen presentatie, geven van presentatie, groepsvergadering, bezwaren onderzoeken en samenzetten}
\lbentry{woensdag 24 maart 2010}{21:20}{23:10}{01:50}{Projecten samenvoegen, herschrijven en aanpassen bezwaren}
\lbentry{donderdag 25 maart 2010}{11:00}{11:15}{00:15}{Presentatie in wiki toevoegen en oplossen afhankelijkheidsprobleem van een bibliotheek}
\lbentry{vrijdag 26 maart 2010}{13:45}{17:45}{04:00}{Vergadering, aanpassen exceptionclassen}
\lbentry{zaterdag 27 maart 2010}{15:00}{23:00}{08:00}{Excepties afwerken, hoofdletterongevoeligdheid ingebouwd, aanpassingen realisatie.tex, grafieken object ge\"orienteerd gemaakt en stukken herschreven, grafieken min max functie in alle events, grafieken extra knoppen, etc..}
\lbentry{donderdag 8 april 2010}{20:15}{23:30}{03:15}{Opzetten Tomcat server op arch linux (windows xp wou niet lukken)}
\lbentry{dinsdag 30 maart 2010}{13:30}{17:45}{04:15}{Bedrijfsbezoek KBC}
\lbentry{vrijdag 2 april 2010}{14:00}{17:45}{03:45}{XML-RPC.NET onderzoek en opbouwen ASP.NET interfaces}
\lbentry{vrijdag 9 april 2010}{13:00}{18:00}{05:00}{Herwerken knoppen en functionaliteit grafiek, herschrijven min en max functies, meerdere refferenties tonen, etc..}
\lbentry{maandag 12 april 2010}{16:00}{22:00}{06:00}{Referentiegrafiek, volumegrafiek onderaan en koppeling ertussen, uitdelen van opties en grafieken in verschillende modules met gemeenschappelijke delen}
\lbentry{donderdag 15 april 2010}{19:00}{23:00}{04:00}{Aanpassen ASP.NET deel zodat quotes voor de grafieken opgehaald kunnen worden en onderzoeken en testen JSON service in ASP.NET}
\lbentry{vrijdag 16 april 2010}{10:00}{12:00}{02:00}{Opzetten ASP.NET omgeving, testen deployen ervan, verderwerken grafieken JSON AJAX ophalen van data bij het triggeren van events en het tekenen daarvan}
\lbentry{vrijdag 16 april 2010}{13:30}{21:00}{07:30}{Opzetten ASP.NET omgeving, testen deployen ervan, verderwerken grafieken JSON AJAX ophalen van data bij het triggeren van events en het tekenen daarvan}
\lbentry{zondag 18 april 2010}{14:00}{17:00}{03:00}{Volume ophalen in grafieken, min max functies is ASP en de backend, integratie detail pagina}
\lbentry{zondag 18 april 2010}{22:15}{23:15}{01:00}{Start transactionmanager}
\lbentry{maandag 19 april 2010}{18:45}{19:15}{00:30}{Structuur transactionmanager}
\lbentry{dinsdag 20 april 2010}{13:30}{17:45}{04:15}{Installatie IIS op schoolserver, installatie Oracle connector op thuisserver en schoolserver, deployen van ASP op beide, schrijven documentatie grafieken, onderzoek submit knop searchfield in ASP}
\lbentry{donderdag 22 april 2010}{00:00}{00:40}{00:40}{Documentatie omzetten naar \LaTeX, en ASP.NET onderzoeken hoe Javascript variabelen invullen met ASP.NET}
\lbentry{vrijdag 23 april 2010}{23:00}{00:00}{01:00}{Zorgen dat op elke ASP pagina de juiste grafiek weergegeven wordt}
\lbentry{zaterdag 24 april 2010}{00:00}{02:30}{02:30}{Schrijven handleiding en inpassen nieuw databankschema en installatie en compilatie van \LaTeX handleiding en analyse en afdrukken van dit alles}
\lbentry{vrijdag 23 april 2010}{14:00}{17:00}{03:00}{Presentatie masterproeven}
\lbentry{zaterdag 24 april 2010}{08:30}{09:00}{00:30}{Aanpassen beschrijvingen handleiding en vervangen class diagram}
\lbentry{maandag 26 april 2010}{12:00}{13:00}{01:00}{Documenteren Transaction Manager}
\lbentry{dinsdag 27 april 2010}{13:30}{17:45}{04:15}{Verder ontwikkelen transaction manager, port forwarding, peer assignment}
\lbentry{vrijdag 30 april 2010}{13:30}{17:45}{04:15}{Uitbreiden transaction manager met geavanceerde orders en verder debuggen ervan}
\lbentry{zondag 2 mei 2010}{14:00}{20:00}{06:00}{Uitbreiden transaction manager met geavanceerde orders en verder debuggen ervan, testing cache manager}
\lbentry{maandag 3 mei 2010}{11:10}{12:45}{01:35}{Toevoegen secondairy limit op orders in alle lagen, bracket limit orders}
\lbentry{maandag 3 mei 2010}{18:30}{20:15}{01:45}{Bracket Limit en Trailing stop manager}
\lbentry{dinsdag 4 mei 2010}{13:30}{17:45}{04:15}{Herwerken documentatie SPClient en TransactionManager en aanpassen ASP site: ordermodule}
\lbentry{vrijdag 7 mei 2010}{13:30}{17:45}{04:15}{Aanpassen ASP site ordermodule}
\lbentry{zaterdag 8 mei 2010}{16:00}{00:00}{08:00}{Herschrijven en optimalisatie grafiekenfuncties, webservice, verkoopmodule ASP, overlay op grafief voor refferentie}
\lbentry{dinsdag 11 mei 2010}{13:30}{17:45}{04:15}{Dynamische vertalingen grafieken + Reference menu grafieken}
\lbentry{donderdag 13 mei 2010}{11:10}{16:30}{05:20}{Aanpassingen website: Index ophalen en interfaces toegevoegd Grafieken refentiemenu aanpassingen en weghalen deprecated functions. Index objecten waren nodig voor de koppeling tussen ISIN en naam van een index bij een ref}
\lbentry{zondag 16 mei 2010}{17:00}{21:00}{04:00}{Fixen transactionmanager en toevoegen nieuwe loginfunctie en cashcontrole
}
\lbentry{maandag 17 mei 2010}{12:00}{20:30}{08:30}{Documentatie doorlezen en herwerken. Schrijven documentatie grafieken en transactionmanager en andere aanvullingen. Zoeken Javascriptfout website. Bijwerken data inlezen Transactionmanager. Bijwerken logo website.}
\lbentry{dinsdag 18 mei 2010}{13:00}{15:00}{02:00}{Installatie onderdelen en afdrukken documenten.}
\lbentry{donderdag 20 mei 2010}{13:00}{19:00}{06:00}{Voorbereiden presentatie en demo}
\lbentry{donderdag 21 mei 2010}{16:00}{17:00}{01:00}{Presentatie en demo}
\lbstop{0}
\end{compact}

\section{Thijs Walcarius}
\begin{compact}
\lbentry{dinsdag 9 februari 2010}{13:30}{15:30}{02:00}{Briefing over VOP, groepsverdeling, etc}
\lbentry{dinsdag 9 februari 2010}{15:45}{17:45}{02:00}{Kennismaking, brainstorming, verkenning beschikbare databronnen, opstellen draft behoeftenanalyse}
\lbentry{vrijdag 12 februari 2010}{13:30}{15:30}{02:00}{Verdere uitwerking van het voorstel van het beursspel}
\lbentry{zondag 14 februari 2010}{09:00}{10:30}{01:30}{TexNicCenter en Miktex installeren, \LaTeX handleiding doorlezen. Omzetten analyse in formaat van hogeschool}
\lbentry{dinsdag 16 februari 2010}{13:00}{18:00}{05:00}{Bijeenkomst met andere groepsleden om technische aspecten van programma te beschrijven: uittekenen GUI, databasestructuur, klassenstructuur, etc}
\lbentry{woensdag 17 februari 2010}{08:15}{12:15}{04:00}{Aanvullen van documentatie op basis van bijeenkomst gisteren: uitzoeken geschikte software voor tekenen klassendiagrammen, herstructureren analysedocument, opzoeken informatie over \LaTeX}
\lbentry{woensdag 17 februari 2010}{19:30}{20:00}{00:30}{Vergelijken documenten Tim en ik om na te gaan of afstemming mogelijk is}
\lbentry{vrijdag 19 februari 2010}{13:30}{17:45}{04:15}{Bespreken gemaakte documentatie, bespreken van implementatie backend en XML-RPC. Mogelijkheden hergebruik voor XML-RPC onderzoeken. Aanmaken van stubs voor businessobjects}
\lbentry{dinsdag 23 februari 2010}{13:30}{17:45}{04:15}{Integreren netbeansproject in SVN, wachten op gesprek met docenten, implementeren van interfaces mbv tijdelijke MySQL-database}
\lbentry{dinsdag 23 februari 2010}{17:45}{19:00}{01:15}{Kijken naar mogelijkheden van gelijktijdige databasetoegang dmv connection pooling zoals mogelijk is in JDBC 3.0 (mbv JDNI)}
\lbentry{vrijdag 26 februari 2010}{13:30}{17:45}{04:15}{Programmeren van business objects, uitzoeken van beste manier om deze met database te laten communiceren. Data Access Objects bekijken, verveeld zijn dat we geen ORM-mappers kunnen gebruiken.}
\lbentry{zaterdag 27 februari 2010}{10:30}{12:00}{01:30}{Implementeren van data access objects voor Exchanges, Securities en SharePrices, zoeken naar bruikbaar alternatief voor Connection Pooling.}
\lbentry{zaterdag 27 februari 2010}{13:00}{16:00}{03:00}{(Her)Installeren VM met Oracle XE, configureren en aanmaken tabellen}
\lbentry{zaterdag 27 februari 2010}{16:15}{18:15}{02:00}{Schrijven van JUnit-tests voor Data Access Objects, testen en corrigeren van fouten}
\lbentry{zaterdag 27 februari 2010}{22:45}{23:30}{00:45}{Opruimen van geschreven rommel, generiek maken van aanspreken DAO}
\lbentry{dinsdag 2 maart 2010}{13:30}{17:45}{04:15}{Inrichten van 1 schoolserver, verder programmeren aan DAO, Sequences in Oracle bekijken}
\lbentry{woensdag 3 maart 2010}{08:45}{10:45}{02:00}{DAO-laag verder afwerken}
\lbentry{vrijdag 5 maart 2010}{11:00}{12:00}{01:00}{Bekijken van templates in ASP.NET}
\lbentry{vrijdag 5 maart 2010}{13:30}{17:45}{04:15}{Experimenteren met ADO.Net, Oracle-installatie proberen porten naar nieuwe server}
\lbentry{maandag 8 maart 2010}{18:00}{19:00}{01:00}{Documentatie schrijven over DAO, verder zoeken op porten van Oracle-installatie}
\lbentry{dinsdag 9 maart 2010}{13:30}{17:45}{04:15}{Effectief overzetten van database uit VM naar echte server. Experimenteren met XML-RPC-client voor JavaSE}
\lbentry{donderdag 11 maart 2010}{10:30}{12:00}{01:30}{Experimenteren met XML-RPC-client voor Java ME en werking Java ME in het geheel}
\lbentry{vrijdag 12 maart 2010}{13:30}{17:45}{04:15}{XML-RPC-server analyseren, functies schrijven om XML-RPC-requests af te handelen}
\lbentry{maandag 15 maart 2010}{20:30}{22:30}{02:00}{SwingX-componenten bekijken en documentatie doorlezen. Uittesten van mogelijkheden}
\lbentry{dinsdag 16 maart 2010}{07:30}{08:00}{00:30}{Eerste aanzet voor desktopapplicatie}
\lbentry{dinsdag 16 maart 2010}{13:30}{17:45}{04:15}{Ontwerpen van gridbag-layout voor desktopapplicatie. Begin van XML-RPC-client}
\lbentry{woensdag 17 maart 2010}{09:00}{12:00}{03:00}{Ontwerpen menu van Administratie-client. Securities-weergave maken. Licht herwerken structuur desktopapplicatie.}
\lbentry{woensdag 17 maart 2010}{20:30}{22:00}{01:30}{Proberen doorgronden van werking XML-server en filters..}
\lbentry{donderdag 18 maart 2010}{09:00}{12:00}{03:00}{Uitzoeken van werking CPAN, downloaden en installeren van Oracle Instantclient, proberen aan de praat krijgen van DBD::Oracle-module, schrijven van import-scriptje voor historische data van aandelen}
\lbentry{vrijdag 19 maart 2010}{10:30}{12:45}{02:15}{Verder afwerken import-scriptje, proberen extra omgeving thuis op te zetten om importeren uit te voeren, klein beetje schrijven aan use-cases}
\lbentry{vrijdag 19 maart 2010}{13:30}{17:45}{04:15}{Infosessie masterproef, invullen peer-evaluation, aanpassen databank, kleine wijzigingen aan desktopapplicatie}
\lbentry{vrijdag 19 maart 2010}{19:00}{20:15}{01:15}{En masse binnenhalen van historische data van enkele aandelen NASDAQ, Euronext amsterdam en Euronext Brussel}
\lbentry{zaterdag 20 maart 2010}{11:00}{13:00}{02:00}{Werking van backend bestuderen, proberen JUnit-tests te schrijven}
\lbentry{zaterdag 20 maart 2010}{16:00}{18:30}{02:30}{Verder JUnit-tests schrijven, werken aan desktopapplicatie, foutjes oplossen}
\lbentry{dinsdag 23 maart 2010}{13:30}{17:45}{04:15}{Presentatie demo. Verdere uitstippeling van agenda voor komende weken}
\lbentry{vrijdag 26 maart 2010}{13:30}{17:45}{04:15}{Werken aan XML-RPC-backend, testen van aanvragen met JUnit}
\lbentry{dinsdag 30 maart 2010}{13:30}{17:45}{04:15}{Bedrijfsbezoek KBC}
\lbentry{vrijdag 2 april 2010}{13:30}{17:45}{04:15}{Debuggen van XML-RPC-backend: oplossen fouten in User-gedeelte}
\lbentry{dinsdag 6 april 2010}{10:00}{12:15}{02:15}{Herschikken van code desktopapplicatie: flexibilisering en opkuis}
\lbentry{dinsdag 6 april 2010}{15:00}{17:30}{02:30}{Uitbreiden van effectenbeheer-gedeelte desktopapplicatie}
\lbentry{donderdag 8 april 2010}{09:00}{11:30}{02:30}{Geen bericht.}
\lbentry{zondag 11 april 2010}{14:00}{17:00}{03:00}{Afwerken componentenbeheer bij desktopapplicatie}
\lbentry{maandag 12 april 2010}{09:00}{15:00}{06:00}{Begin van gebruikersbeheer in desktopapplicatie: bijwerken van Securitiesbeheer, en omzetten naar gebruikersbeheer, debuggen java-xml-rpc-client}
\lbentry{dinsdag 13 april 2010}{13:30}{16:45}{03:15}{Geen bericht}
\lbentry{woensdag 14 april 2010}{14:00}{17:00}{03:00}{Vervangen van admin-boolean door rollen in backend, ook bijwerken in desktopapplicatie. Zorgen dat gegenereerde id's worden teruggegeven}
\lbentry{donderdag 15 april 2010}{09:00}{11:00}{02:00}{Verder werken aan gebruikersbeheer in desktopapplicatie - implementeren generieke dialoogvenster voor veranderingen met reden}
\lbentry{donderdag 15 april 2010}{20:00}{22:45}{02:45}{Zoeken op werking triggers, diverse problemen  proberen oplossen}
\lbentry{vrijdag 16 april 2010}{09:00}{11:30}{02:30}{Aanmaken puntentransacties - zoeken op triggers}
\lbentry{vrijdag 16 april 2010}{14:00}{17:00}{03:00}{Gebruikersbeheer verder afwerken, fouten uit backend halen}
\lbentry{zaterdag 17 april 2010}{13:00}{15:30}{02:30}{Herwerken van client-backend naar eleganter systeem}
\lbentry{zaterdag 17 april 2010}{16:00}{20:00}{04:00}{Debuggen backend, herwerken en toevoegen van functionaliteit voor transacties - toevoegen van triggers}
\lbentry{zondag 18 april 2010}{13:30}{16:30}{03:00}{Toevoegen order- en transactionstukken aan Java XML-RPC clientlibrary}
\lbentry{maandag 19 april 2010}{18:15}{19:00}{00:45}{Uittesten van functionaliteit voor ophalen quotes, debuggen}
\lbentry{maandag 19 april 2010}{20:30}{23:45}{03:15}{Eerste functionele versie maken van transactiemanager}
\lbentry{dinsdag 20 april 2010}{13:30}{17:30}{04:00}{Deployen van betaversie en oplossen van bugs}
\lbentry{donderdag 22 april 2010}{19:30}{20:30}{01:00}{Uitsplitsen van Java XML-RPC-client naar aparte library}
\lbentry{zaterdag 24 april 2010}{10:00}{13:30}{03:30}{Testen betaversies en host spelen op opendeurdag}
\lbentry{zondag 25 april 2010}{13:30}{17:30}{04:00}{Herwerken Java XML-RPC-client naar Java ME}
\lbentry{dinsdag 27 april 2010}{13:30}{17:45}{04:15}{Bespreken verdere planning project, uitzoeken werking eID-libraries}
\lbentry{woensdag 28 april 2010}{09:15}{11:15}{02:00}{Werking onderzoeken van java eID-library-wrapper, uitzoeken hoe te integreren in netbeansproject}
\lbentry{woensdag 28 april 2010}{14:00}{15:30}{01:30}{Brielen met eerste implementatie loginscherm, zal opnieuw moeten gedaan worden..}
\lbentry{donderag 29 april 2010}{9:30}{12:00}{02:30}{Begin internationalisatie van desktopapplicatie}
\lbentry{vrijdag 30 april 2010}{13:30}{17:45}{04:15}{Herwerken desktopapplicatie: we maken een menubalk om instellingen toe te laten, maar herwerken daarvoor het menu aan de linkerkant tot acties zodat het ook kan gebruikt worden voor extra items in de menubalk. Uitzoeken van beste manier om instellingen op te slaan}
\lbentry{zaterdag 1 mei 2010}{10:00}{15:00}{05:00}{Toevoegen van rollen aan database om beveiliging mogelijk te maken}
\lbentry{zondag 2 mei 2010}{14:00}{16:00}{02:00}{Onderzoeken van functionaliteit Apache XML-RPC-server op gebied van authenticatie, en hoe ik best mijn eigen hook ervoor maak, uittesten van verschillende mogelijkheden}
\lbentry{dinsdag 4 mei 2010}{13:30}{17:45}{04:15}{Aanpassen van backend en SPClient zodat deze kunnen omgaan met sessies}
\lbentry{donderdag 7 mei 2010}{10:00}{12:00}{02:00}{Bugfixen van vervelende fouten in sessiehandeling, herwerken van SPClient om manier van opslaan authenticatie te verbeteren}
\lbentry{vrijdag 8 mei 2010}{13:30}{14:15}{00:45}{Zoeken naar manier om invalid connecties uit de Apache DB Connection Pool te halen, en onze backend er niet meer over te doen struikelen}
\lbentry{vrijdag 8 mei 2010}{14:15}{17:45}{03:30}{Afwerken van SPClient om hem compatibel te maken met sessies op de backend}
\lbentry{zaterdag 9 mei 2010}{9:00}{12:00}{03:00}{Herwerken van SPClient zodat er ondersteuning komt voor de private-url}
\lbentry{zaterdag 9 mei 2010}{14:00}{18:30}{04:30}{Herwerken loginprocedure van desktopapplicatie zodat meerdere loginmogelijkheden kunnen worden aangeboden. Configureerbaar maken van loginprocedures. Starten aan implementatie van eID-login}
\lbentry{zondag 10 mei 2010}{9:15}{11:45}{02:30}{Functionaliteit aanmaken zodat met de eID kan worden ingelogd op de beheersapplicatie. Om de bestaande beveiliging op de backend niet onderuit te halen is het wel niet zo mooi geworden}
\lbentry{zondag 10 mei 2010}{13:30}{17:00}{03:30}{Java ME-applicatie uitbreiden qua functionaliteit, en wat herstructureren}
\lbentry{maandag 11 mei 2010}{7:45}{8:30}{00:45}{Snel maken van een before trigger}
\lbentry{dinsdag 12 mei 2010}{12:15}{13:15}{01:00}{Afwerken van eID-login, controle van functionaliteit van Preferences}
\lbentry{dinsdag 12 mei 2010}{13:30}{17:45}{04:15}{Aanpassen van kXMLRPC library zodat deze om kan gaan met authenticatie met de backend (er was nog geen funcationaliteit aanwezig om aan Basic Authentication te doen}
\lbentry{donderdag 13 mei 2010}{9:30}{12:00}{02:30}{GSM-applicatie opschonen en proberen afwerken, debuggen van fouten door verkeerde aanvragen naar de backend}
\lbentry{donderdag 13 mei 2010}{15:45}{18:00}{02:15}{Opschonen van administratieprogramma: code wat herwerken, slecht functionerende onderdelen verwijderen}
\lbentry{vrijdag 14 mei 2010}{9:00}{10:00}{01:00}{Maken van betere before trigger om iets nuttigs te controleren..}
\lbentry{vrijdag 14 mei 2010}{10:00}{12:00}{02:00}{Controle functionaliteit administratieprogramma, afwerken internationalisatie}
\lbentry{vrijdag 14 mei 2010}{13:30}{15:30}{02:00}{Verder testen van het administratieprogramma, proberen foutjes te ontdekken en recht te trekken}

\lbstop{0}
\end{compact}



% XML-RPC
\chapter{XML-RPC specificatie}
\label{chap:xmlrpc}
\input{xml-rpc.tex}

\end{document}

