%
% Dataontwerp
%

\chapter{Dataontwerp}

\begin{figure}[h!]
	\centering
		\includegraphics[width=0.5\textwidth]{images/realisatie/ER_Diagram}
	\caption{Entity-relationship model.}
\end{figure}

\section{Afscheiding datatoegang}

Door het gebruik van een aparte backend, wordt rechtstreekse toegang tot de effectieve data verhinderd. Zo kunnen we doorgedreven controles en verificatie van de ingegeven data toepassen, alsook eventuele optimalisaties doorvoeren. Onderstaand diagram illustreert de samenwerking van de componenten in kwestie:

\begin{figure}[h!]
	\centering
		\includegraphics[width=0.5\textwidth]{images/realisatie/Class_Diagram}
	\caption{Overzicht werking backend}
\end{figure}

Op onze databank hebben we ook een aantal triggers. De punten mogen niet zomaar mogen aangepast in de gebruikerstabel. Elke wijziging gebeurt via de pointshistory tabel. Het toevoegen van een nieuw record daarin zorgt ervoor dat een trigger de delta waarde in de pointshistory optelt bij de punten in de userstabel. Zodoende hebben we een cache van het puntentotaal die we snel kunnen raadplegen.

Ook bij het omzetten van een order naar een transactie wordt een trigger geactiveerd die nakijkt of de gebruiker wel voldoende cash positie bezit.

De cash positie van een gebruiker kan niet zomaar aangepast worden, dit wordt ook beveiligd met een trigger.


%
% Procedureontwerp
%

\chapter{Procedureontwerp}

\section{XML-RPC interface}

Zoals vermeld bij het technisch ontwerp, gebruiken we het XML-RPC protocol als communicatiemiddel tussen de backend en zijn interfaces. Hiertoe moeten we de set aan functies vastleggen die een client kan uitvoeren, en de bijhorende signatuur documenteren.

Aangezien XML-RPC geen ondersteuning biedt voor namespaces of andere vormen van functieorganisatie, hanteren we zelf een mechanisme om dit te bekomen: een methode-naam bestaat altijd uit twee delen, gescheiden door een punt. Het deel voor het scheidingsteken duidt het pakket aan, het deel erna de methode die we willen oproepen.
Zo delen we de backend op in de volgende drie primaire klassen:
\begin{itemize}
\item{System: functionaliteit voor beheer en de statusinformatie van het systeem.}
\item{User: beheer van gebruikers en ophalen gebruikersinformatie.}
\item{Finance: functionaliteit gerelateerd aan het beurswezen.}
\end{itemize}
Hogere-orde klassen zijn eventueel ook mogelijk (zoals \emph{System.Database}), maar niet verplicht. De semantiek is daarbij identiek aan primaire klassen, met een punt als scheidingsteken.

\subsection{Algemene foutcodes}

De XML-RPC specificatie biedt ook ondersteuning voor foutberichten, in de form van een bericht met een $<$fault$>$ tag. Die tag moet steeds twee $<$member$>$ tags bevatten, namelijk een foutcode $<$faultCode$>$ van het integer type, en een foutbericht $<$faultString$>$ van het string type. Elk van de klassen kan zo specifieke foutmeldingen vastleggen.
Maar er zijn ook generieke foutmeldingen, die van toepassing zijn op alle klassen. Deze foutmeldingen, waarvan de foutcode in het bereik $[0, 100[$ valt, worden hieronder beschreven:
\todo{De lijst van foutmeldingen staat er niet rechstreeks onder. -Dieter}

\begin{table}
\begin{tabular}{| c p{5cm} p{7cm} |}
	\hline
	Foutcode & Foutbericht & Controle \\
	\hline
	
	0 & Internal Failure & Verifieer de status van de backend, de log kan hierbij helpen. \\
	\hline
	
	$[1-10[$ & \emph{Subsystem failures.} & \\
	1 & Database Failure & Er is een probleem met de database (onbeschikbaar, corrupt, ...), zie de log voor meer details. \\
	2 & Scraper Failure & Er is een probleem met de scraper (onbeschikbaar, uitgevallen, ...), zie de log voor meer details. \\
	\hline
	
	$[10-20[$ & \emph{Service issues.} & \\
	10 & Service Unavailable & De backend kan tijdelijk niet gebruikt worden (werkzaamheden, overloaded, ...). \\
	11 & Unauthorized & Meld u aan vooraleer deze functie te gebruiken. \\
	12 & Not Enough Information & Er is niet genoeg informatie om het object aan te maken \\
	\hline
	
	$[20-30[$ & \emph{Invocation issues.} & \\
	20 & Version Not Supported & De client gebruikt een verkeerd communicatieprotocol. \\
	21 & Not Found & Methode niet gevonden, verifieer de schrijfwijze en de klasse. \\
	22 & Bad Request & Probleem met de parameters, controleer het gebruik van de methode. \\
	23 & Non Existing Entity & Het item dat je opvroeg bestaat niet \\
	24 & Pre Existing Entity & Er bestaat reeds zo'n item \\
	25 & Read Only Key & Er is een aanvraag gedaan om een key aan te passen die niet aangepast mag worden \\
	26 & Key does not exist & Er is een aanvraag gedaan om een key aan te passen die niet bestaat \\
	\hline

	$[30-40[$ & \emph{Filter issues.} & \\
	31 & Filter Failure & Er is een probleem met de doorgegeven filter, controleer deze. \\
	\hline
\end{tabular}
\caption{Generieke foutcodes in het backend-protocol.}
\end{table}

\subsection{Authenticatie en autorisatie}

Aangezien het XML-RPC protocol gebruik maakt van het HTTP-protocol, kunnen we diens functionaliteit gebruiken om authenticatie te bekomen. Daartoe zullen we gebruik maken van \emph{basic authentication}, waarbij de client indien gevraagd een gebruikersnaam en wachtwoord naar de server doorstuurd. Normaal zou dit geen extra code teweegbrengen, en zouden we volledig op de authenticatiemogelijkheden van het HTTP protocol kunnen berusten.

Deze denkpiste gaf echter enkele problemen. Vooreerst is het sterk onveilig aangezien het passwoord continu in (quasi) plaintext doorgestuurd wordt. Het maakt echter verfijnde autorisatie moeilijk, tenzij we voor elk authenticatie-niveau een aparte URL zouden voorzien. Daarom hebben we gekozen om gebruik te maken van een eigen authenticatie-laag, waarbij een XML-RPC functie gebruikt wordt om de gebruiker aan te melden (dit in tegenstelling tot het gebruik maken van de HTTP laag). Na aanmelding geeft de functie in kwestie een sessie-ID terug, die een sessie serverside linkt aan een specifieke gebruiker. Zo kunnen we, opnieuw binnenin de XML-RPC laag, controleren of een gebruiker wel correct aangemeld is, en heel dynamisch requests toelaten of weigeren. Dit is gebaseerd op een flexibel rechten-systeem, waarbij elke gebruiker in de database gelinkt wordt aan een rol die bepaalde rechten toekent.

Om de veiligheid te verhogen gebruiken we verschillende technieken. Eerst en vooral is de sessie tijdelijk, en worden ongebruikte sessie-keys regelmatig geinvalideerd. Ook hebben we het moeilijk gemaakt om sessie-keys te genereren, door ze op een unieke manier te linken aan data die enkel de gebruiker zelf kent (zijn wachtwoord). Om het tenslotten ook moeilijk te maken om een sessie-key te linken aan een gebruiker, implementeren we in onze sessie-hash een variabele seed.

Authenticatie en autorisatie gaat ook gepaard met een specifieke set aan mogelijke errormessages. Allemaal geklasseerd binnen de Service Exceptions, zijn de volgende subtypes mogelijk: ``invalid credentials'' wanneer het authenticeren mislukt is, ``unauthorized'' wanneer een gebruiker niet geautoriseerd is tot het gebruik van een bepaalde functie, en tenslotte ``session corrupt'' wanneer een gespecificeerde sessie ongeldig blijkt te zijn.

Hoewel een administrator gebruikersprofielen (en bijhorende autorisatiebits) manueel kan aanpassen, onderscheiden we volgende categorie\"en:
\begin{itemize}
\item{Gast}
\item{Speler}
\item{Administrator}
\item{Scraper}
\end{itemize}

\subsection{System-klasse}

Deze klasse biedt de client mogelijkheden om het systeem te beheren. Zoals het ophalen van informatie, wijzigen van configuraties, en (her)starten of stoppen van bepaalde subsystemen.

\function{Backend.Status}
	{ status van de backend ophalen }
	{ geen }
	{ integer die de status beschrijft:
		\begin{itemize}
		\item{0: maintenance-mode}
		\item{1: geen problemen gemeld}
		\end{itemize} }
	{ administrator-rechten }

\function{Backend.Stats}
	{ backend-statistieken ophalen }
	{ geen }
	{ struct met statistieken:
		\begin{itemize}
		\item{users: aantal gebruikers online}
		\item{req: aantal verwerkte XML-RPC requests sinds de start}
		\item{uptime: hoe lang de backend al draait}
		\end{itemize} }
	{ administrator-rechten }

\function{Backend.Restart}
	{ de backend opnieuw opstarten }
	{ geen }
	{ bool die indiceert of de actie succesvol was }
	{ administrator-rechten }

\function{Backend.Stop}
	{ de backend stoppen }
	{ geen }
	{ bool die indiceert of de actie succesvol was }
	{ administrator-rechten }

\function{Backend.ClearCache}
	{ de backend cache legen }
	{ geen }
	{ bool die indiceert of de actie succesvol was }
	{ administrator-rechten }

\function{Database.Status}
	{ informatie van de database ophalen }
	{ geen }
	{ integer die status beschrijft:
		\begin{itemize}
		\item{0: er kon geen verbinding naar de databank gelegd worden}
		\item{1: succesvol verbonden met de database}
		\end{itemize} }
	{ administrator-rechten benodigd}

\function{Database.Stats}
	{ database-statistieken ophalen. Dit zijn de statistieken geleverd door de databank zelf, ze zijn dus niet beperkt tot acties ondernomen door de backend }
	{ geen }
	{ struct met statistieken:
		\begin{itemize}
		\item{uptime: hoe lang de databank al draait}
		\item{rate: aantal transacties per seconde}
		\end{itemize} }
	{ administrator-rechten }

\function{Scraper.Status}
	{ status van de scraper ophalen }
	{ geen }
	{ integer die status beschrijft:
		\begin{itemize}
		\item{0: niet geactiveerd}
		\item{1: inactief}
		\item{2: bezig met data-mining}
		\end{itemize} }
	{ administrator-rechten }

\function{Scraper.Stats}
	{ scraper-statistieken ophalen }
	{ geen }
	{ struct met statistieken:
		\begin{itemize}
		\item{executes: aantal voltooide plugin-uitvoeringen}
		\item{uptime: hoe lang de scraper al draait}
		\item{traffic: hoeveelheid data verzonden en ontvangen}
		\item{plugins: aantal geactiveerde plugins}
		\item{securities: aantal beschikbare effecten}
		\item{exchanges: aantal beschikbare beurzen}
		\item{indexes: aantal beschikbare indexen}
		\item{delay: tijd tot de volgende plugin-uitvoering}
		\item{memory: geheugengebruik}
		\end{itemize} }
	{ administrator-rechten }

\function{Scraper.Restart}
	{ de scraper herstarten }
	{ geen }
	{ bool die indiceert of de actie succesvol was }
	{ administrator-rechten }

\function{Scraper.Stop}
	{ de scraper permanent stilleggen }
	{ geen }
	{ bool die indiceert of de actie succesvol was }
	{ administrator-rechten }


\subsection{User-klasse}

Hier vindt men de nodige methodes terug om gebruikers te beheren. Dit is echter niet beperkt tot de administrator: ook gebruikers zelf kunnen hun eigen profiel (in beperktere mate) beheren.

\function{List}
	{ lijst met publieke informatie opvragen van de gebruikers }
	{ een filter, met de volgende mogelijke sleutels:
		\begin{itemize}
		\item{nickname: gebruikersnaam}
		\item{regdate: datum van registratie}
		\item{points: aantal behaalde punten}
		\end{itemize} }
	{ een lijst met structs:
		\begin{itemize}
		\item{id: gebruikers-id}
		\item{nickname: gebruikersnaam}
		\item{regdate: datum van registratie}
		\item{points: aantal behaalde punten}
		\end{itemize} }
	{ geen benodigd }

\function{Details}
	{ lijst met publieke informatie opvragen van een gebruiker }
	{ een filter, met de volgende mogelijke sleutels:
		\begin{itemize}
		\item{id: een gebruikers-id}
		\end{itemize} }
	{ een lijst met structs:
		\begin{itemize}
		\item{id: gebruikers-id}
		\item{firstName: voornaam}
		\item{lastName: achternaam}
		\item{rrn: rijksregisternummer (nodig voor de eid authenticatie)}
		\item{cash: huidige cashpositie}
		\item{startkapitaal: start kapitaal}
		\end{itemize} }
	{ administrator-rechten of identificatie als de gebruiker beschreven is in het 'id' veld van de filter }

\todo{specifieke foutcode genereren} 
\function{Create}
	{ aanmaken van een nieuwe gebruiker }
	{ struct met de gebruikersinformatie:
		\begin{itemize}
		\item{}
		\end{itemize} }
	{ id van de aangemaakte gebruiker }
	{ administrator-rechten }

\function{Modify}
	{ wijzigen van een bestaande gebruiker }
	{ een filter die gebruikers selecteert, en een struct met de te-wijzigen gebruikersinformatie:
		\begin{itemize}
		\item{}
		\end{itemize} }
	{ een boolean die aangeeft of de actie al dan niet gelukt is }
	{ administrator-rechten of identificatie als de gebruiker beschreven is in het 'id' veld van de filter }

\function{Remove}
	{ verwijderen van een bestaande gebruiker }
	{ een filter die gebruikers selecteert }
	{ een boolean die aangeeft of de actie al dan niet gelukt is }
	{ administrator-rechten }

\function{Portfolio.List}
	{ basisinformatie van effecten in bezit opsommen (gebruik de functie Finance.Security.Details voor meer informatie) }
	{ een filter, met de volgende mogelijke sleutels:
		\begin{itemize}
		\item{id: een gebruikers-id}
		\end{itemize} }
	{ array van structs met basisinformatie van de effecten:
		\begin{itemize}
		\item{symbool: identificatiecode van het aandeel}
		\end{itemize} }
	{ administrator-rechten of identificatie als de gebruiker beschreven is in het 'id' veld van de filter }

\todo{behoudt Orders.List ook uitgevoerde orders van een bepaald timeframe? - we hadden normaal gezegd dat we hier alle order tonen, ook de uitgevoerde. Het timeframe/resente orders zetten is dan kwestie van het zetten van de juiste filter en kan de client dan zelf bepalen. Kan je je hier in vinden? - Laur}
\function{Order.List}
	{ wachtende orders bekijken }
	{ een filter, met de volgende mogelijke sleutels:
		\begin{itemize}
		\item{id: gebruikers-id}
		\end{itemize} }
	{ lijst van structs met order-informatie:
		\begin{itemize}
		\item{id: id van order}
		\item{type: aankoop of verkoop}
		\item{effect: symbool van het effect}
		\item{aantal: het aantal effecten dat het order verhandelt}
		\end{itemize} }
	{ administrator-rechten of identificatie als de gebruiker beschreven is in het 'id' veld van de filter }

\todo{specifieke foutmelding als order aanmaken faalde}
\function{Order.Create}
	{ een nieuw order aanmaken }
	{ struct met orderdetails:
		\begin{itemize}
		\item{id: gebruikers-id}
		\item{type: aankoop of verkoop}
		\item{effect: symbool van het effect}
		\item{aantal: het aantal effecten dat het order verhandelt}
		\end{itemize} }
	{ id van het aangemaakte order }
	{ administrator-rechten of identificatie als de gebruiker beschreven is in het 'id' veld van de filter }

\function{Order.Cancel}
	{ een bestaande order annuleren }
	{ een filter, met de volgende mogelijjke sleutels:
		\begin{itemize}
		\item{id: gebruikers-id}
		\item{order: order-id}
		\end{itemize} }
	{ geen }
	{ administrator-rechten of identificatie als de gebruiker beschreven is in het 'id' veld van de filter }

\function{Transaction.List}
	{ lijst van uitgevoerde transacties weergeven }
	{ een filter, met de volgende mogelijke sleutels:
		\begin{itemize}
		\item{id: gebruikers-id}
		\end{itemize} }
	{ array van structs die transactie-informatie bevatten:
		\begin{itemize}
		\item{id: identificatie van de transactie}
		\item{timestamp: datum van de transactie}
		\item{price: aankoopprijs}
		\item{amount: hoeveelheid}
		\item{symbol: symbool van het effect}
		\item{type: soort effect}
		\item{name: naam van het effect}
		\end{itemize} }
	{ administrator-rechten of identificatie als de gebruiker beschreven is in het 'id' veld van de filter }

\subsection{Finance-klasse}

Tenslotte zijn er nog de methodes gerelateerd met het effectieve beurswezen, die (hoofdzakelijk) terug te vinden zijn in deze klasse. Enkel het ophalen van de portfolio bevindt zich, logischerwijs, in de User-klasse.

\function{Exchange.List}
	{ lijst van beurzen opvragen }
	{ een filter, met de volgende mogelijke sleutels:
		\begin{itemize}
		\item{}
		\end{itemize} }
	{ array van structs met beurs-info:
		\begin{itemize}
		\item{id: identificatie-symbool van de beurs}
		\item{discription: een langere beschrijving van de beurs}
		\item{location: locatie van de beurs zich bevindt}
		\end{itemize} }
	{ gebruikers-rechten }

\function{Exchange.Modify}
	{ informatie van een beurs wijzigen }
	{ een filter, en een struct met beursinformatie:
		\begin{itemize}
		\item{description}
		\item{location}
		\end{itemize} }
	{ niks }
	{ administrator-rechten of scraper-rechten }

\function{Exchange.Create}
	{ een beurs toevoegen }
	{ een struct met beursinformatie:
		\begin{itemize}
		\item{description}
		\item{location}
		\end{itemize} }
	{ niks }
	{ administrator-rechten of scraper-rechten }

\function{Index.List}
	{ lijst van indexen opvragen }
	{ een filter, met de volgende mogelijke sleutels:
		\begin{itemize}
		\item{}
		\end{itemize} }
	{ array van structs met index-informatie:
		\begin{itemize}
		\item{id: identificatie-symbool van de index}
		\item{exchange: identificatie-symbool van de beurs waarop de index zich bevindt}
		\item{description}
		\end{itemize} }
	{ gebruikers-rechten }

\function{Index.Modify}
	{ informatie van een index wijzigen }
	{ een filter, en een struct met indexinformatie:
		\begin{itemize}
		\item{description}
		\end{itemize} }
	{ niks }
	{ administrator-rechten of scraper-rechten }

\function{Index.Create}
	{ een index toevoegen }
	{ en een struct met indexinformatie:
		\begin{itemize}
		\item{description}
		\end{itemize} }
	{ niks }
	{ administrator-rechten of scraper-rechten }

\function{Security.List}
	{ lijst met oppervlakkige informatie van effecten opvragen }
	{ een filter, met de volgende mogelijke sleutels:
		\begin{itemize}
		\item{}
		\end{itemize} }
	{ array van structs met informatie over het effect:
		\begin{itemize}
		\item{id: identificatie-symbool van het effect}
		\item{beurs: identificatie-symbool van de beurs waarop het effect zich bevindt}
		\item{index: identificatie-symbool van de index waarop het effect zich bevindt}
		\item{koers: de huidige koers}
		\item{flags: de gezette vlaggen}
		\end{itemize} }
	{ gebruikers-rechten }

\function{Security.Modify}
	{ informatie van een effect wijzigen }
	{ een filter, en een struct met informatie over het effect:
		\begin{itemize}
		\item{...}
		\end{itemize} }
	{ aantal aangepaste effecten }
	{ administrator-rechten of scraper-rechten }

\function{Security.Create}
	{ een effect toevoegen }
	{ een struct met informatie over het effect:
		\begin{itemize}
		\item{...}
		\end{itemize} }
	{ niks }
	{ administrator-rechten of scraper-rechten }
	
\function{Security.Delete}
{ een effect verwijderen }
{ een filter }
{ aantal verwijderde effecten }
{ administrator-rechten of scraper-rechten }

\function{Security.Details}
	{ defail-informatie van een effect opvragen }
	{ een filter }
	{ struct met defail-informatie over het effect:
		\begin{itemize}
		\item{salesvolume: omzet}
		\item{buyprice: aankoopprijs}
		\item{saleprice: verkoopprijs}
		\end{itemize} }
	{ gebruikers-rechten }

\function{Security.Flag}
	{ vlaggen van een effect wijzigen }
	{ het type vlag, de instelling, en een filter met de volgende mogelijke sleutels:
		\begin{itemize}
		\item{}
		\end{itemize} }
	{ niks }
	{ administrator-rechten }

\function{Security.Update}
	{ nieuwe quote aan een security toevoegen (of updaten) }
	{ een struct dat een quote voorstelt, met de volgende inhoud:
		\begin{itemize}
		\item{security: id van het security dat geüpdatet moet worden}
		\item{time: tijdstip waarop de quote gelde}
		\item{price: huidige prijs}
		\item{volume: omzet van het aandeel}
		\item{buy: aankoopprijs}
		\item{sell: verkoopsprijs}
		\item{low: laagterecord van het aandeel}
		\item{high: hoogterecord van het aandeel}
		\end{itemize} }
	{ niks }
	{ scraper-rechten }

De mogelijke vlaggen zijn:
\begin{itemize}
\item{1: kunnen gebruikers het effect zien}
\item{2: kunnen gebruikers het effect aankopen (impliceert flag 1)}
\end{itemize}


%
% Grafieken
%

\todo{details over de implementatie van Laurens zijn grafieken. Misschien ook alles van styling (template/masterpage) hier?}

\chapter{Dynamische grafieken}

Zoals reeds vermeld maken we gebruik van de JQuery en FLOT bibliotheken om interactieve grafieken te maken. Die zullen volgende functionaliteit bevatten:
\begin{itemize}
\item{zoomen in de grafiek doormiddel van het scrolwiel}
\item{zoomen in de grafiek doormiddel van een dubbelklik op een locatie}
\item{uitzoomen met een knop}
\item{de grafiek zal je kunnen vastnemen en over de x-as verslepen}
\item{je zal kunnen wisselen tussen een sleepmodus en een modus waarin je een kader kan selecteren waarin je wil inzoomen (enkel over de x-as)}
\item{er zal onderaan een tweede, gekoppelde grafiek te zien zijn waarin een bargrafiek te zien is waar de volumes van de effecten in te zien zijn}
\item{eventueel: je zal extra referentie lijnen kunnen toevoegen}
\item{je zal een beperkt aantal maal op je laatste handelingen kunnen terugkeren}
\item{de opgehaalde punten zullen gecached worden en enkel indien vereist zal via AJAX nieuwe punten opgehaald worden}
\end{itemize}


%
% Scraper
%

\chapter{Scraper}

De scraper is een elementair gedeelte van StockPlay, daar het hele project nood heeft aan een up-to-date dataset.

\section{Databronnen}

Vooraleer een scraper te ontwerpen, was het belangrijk de mogelijke bronnen te identificeren. Logischerwijs beperken we ons tot gratis beschikbare bronnen, maar zelfs dan komen er verschillende bronnen in aanmerking. Om uiteindelijk een zo uitgebreid mogelijke dataset te hebben, besloten we om het mogelijk te maken verschillende bronnen tegelijk te gebruiken.

\subsection{De Tijd}

Een eerste kandidaat om onze scraper op in te laten werken, was de website van De Tijd. Deze site voorziet in een uitgebreid aanbod aandelen en indexen van verschillende beurzen, en biedt sommige van die data zelfs real-time aan. Verder onderzoek van hoe die realtime updates gebeurden - en het reverse-engineering van de bijhorende HTTP-based API - leverde ons nogal snel een compact Perl-script op dat in staat was de aandelen en indexen van een bepaalde beurs eenmalig op te halen, om in een later stadium dan de effectieve koersen op te halen via de aangeboden API in kwestie.

Aangezien deze bron een hele waaier aan aandelen zichtbaar maakt, er veel data over publiceert, en bovendien die data beschikbaar maakt via een compacte en handige interface, hebben we deze als belangrijkste bron gebruikt bij het opvullen en updaten van onze database.

\subsection{Euronext}

Deze offici\"ele bron biedt via een webinterface heel gedetailleerde informatie over alle aandelen die beschikbaar zijn op de beurs. De site biedt ook een ruime blik in het verleden, door de gebruiker toe te laten om 2 jaar aan koersen op te halen. Hoewel deze data veel beter en hi\"erarchischer georganiseerd is, hebben we besloten deze bron niet te gebruiken en de site van De Tijd eerder te gebruiken. De reden hiervoor is dat de site van De Tijd, die eveneens alle aandelen op de Euronext beurs kan weergeven, realtime data weergeeft en hierbij ook een handigere en compactere interface voor gebruikt.

De site bleek echter wel nuttig te zijn bij het ontwerpen van een artifici\"ele intelligentie, waarbij niet zozeer realtime data maar vooral veel historische data benodigd is om het massief-lerende netwerk te trainen.

\subsection{Perl Finance modules}

Het Perl CPAN archief kent een uitgebreid aanbod aan modules die in staat zijn om koersen van aandelen op te halen van verschillende bronnen. Hoewel deze bron uitmunt in gemak van gebruik (eenvoudige maar doordachte API, hergebruik van code) hebben we toch besloten deze niet te gebruiken. De motivatie daarvoor bestaat uit een aantal punten: eerst en vooral is het de bedoeling dat we zelf een Perl scraper schrijven, eenvoudigweg een bestaande module gebruiken voldoet niet aan deze eis. Daarnaast kennen de modules voor Europese beurzen niet dezelfde kwaliteit en resolutie als een zelfgemaakt script die de site van De Tijd of die van Euronext raadpleegt, en een hoge resolutie is belangrijk om een kwalitatief en correct verloop van het spel te garanderen.

\section{Ontwerp}

Om op een flexibele wijze verschillende bronnen parallel te kunnen raadplegen, hebben we geopteerd voor een plugin-based design. Hierbij wordt elke bron geassocieerd met een enkele plugin, die een vaste interface implementeert om gestandaardiseerd data-opvragingen van buitenaf toegankelijk te maken. De plugins kunnen echter ook hi\"erarchisch uitgebreid worden: zo zal een plugin die data ophaalt niet rechtstreeks de \emph{Source} interface implementeren, maar gebruik maken van de \emph{Website} interface, die op zijn beurt de \emph{Source} interface uitbreidt met functionaliteit toegespitst tot websites.

De hoofdinterface Source beschrijft essentieel twee kenmerken: het beschikbaar zijn van een ``exchanges'' attribuut, en een ``getLatestQuotes'' methode. Het \textbf{exchanges} attribuut is een hi\"erarchische structuur van de beurzen die de plugin kan scrapen: de elementen zijn instanties zijn van de StockPlay::Exchange klasse, waarvan de objecten toegang geven tot de indexen (klasse StockPlay::Index) en aandelen (klasse StockPlay::Security) van de klasse. De verantwoordelijkheid voor het aanmaken van deze structuur ligt bij de plugin die het Exchange object aangemaakt heeft.
De Plugin interface dwingt ook een \textbf{getQuotes} functie af, die gebruikt wordt om eenmaal de initi\"ele opbouw aan Exchanges, Securities en Indexen voltooid is, de koersen van een lijst Securities op te halen. Deze functie geeft dan ook een lijst aan StockPlay::Quote objecten terug. Dergelijke objecten, 1 op 1 verbonden met een Security, bevatten de effectieve koersdata. Zo zijn verschillende velden beschikbaar zoals ``time'', ``volume'', of ``price''. Een ander belangrijk veld is het ``delay'' veld, dat omschrijft hoe lang er moet gewacht worden om een nieuwe koers op te halen.

Nu vastligt hoe de data in een uniforme en gestructureerde wijze opgeslagen wordt, is het duidelijk dat de data opgedeeld kan worden in twee categorie\"en: presistente en variabele data.
Onder de persistente data behoort de hi\"erarchie aan Exchange, Security en Index objecten die omschrijft welke beurzen en aandelen de plugin toegang tot heeft, en hoe die zich onderling relateren. Aangezien de objecten in kwestie steeds voorzien in een ``private data container'' (die plugins kunnen gebruiken om bronspecifieke maar verder niet relevante data op te slaan), volstaat het om die eenmaal op te bouwen. Ze kunnen daarna steeds opnieuw herbruikt worden om de variabele data op te halen. Die variabele data bestaat natuurlijk uit de uiteindelijke Quotes.
Door deze essenti\"ele splitsing is het mogelijk om de persistente data eenmaal op te halen en dan binnen een bepaald interval te herbruiken om variabele data op te halen. Dit maakt de scraper niet alleen sneller, het verlaagt ook de hoeveel data die steeds benodigd is om de aangevraagde data op halen. Hierbij reduceren we de kans dat de scraper teveel opvalt en eventueel zou verbannen worden van de bron in kwestie.

Functionaliteit zoals de private data containers, en andere huishoudelijke functionaliteit, zit verpakt in de StockPlay::Plugin klasse. Elke klasse die zich gedraagt als een plugin (zoals de Source interface hierboven beschreven), moeten deze klasse implementeren. Dit onderscheid is gemaakt zodat de plugin manager ook nog gebruikt zou worden voor andere scripts die gebruik maken van het Perl framework dat ontwikkeld is in functie van de scraper, zoals de artifici\"ele intelligentie.

\section{Communicatie}

Nu de scraper in staat is om gevraagde koersen op een effici\"ente manier op te halen, moet de methode van communicatie naar de backend nog ontworpen worden. Hierbij zijn er twee mogelijkheden: een PUSH of PULL model.

In geval van een \textbf{PULL} model is de Backend de initiator, en zal die de scraper vragen om een bepaalde selectie data. In een \textbf{PUSH} model daarentegen zal de scraper zelf het initiatief nemen. Hierbij is het vrij snel duidelijk dat het PUSH model de beste keuze is. Moesten we immers voor een PULL model gaan, dan zou de scraper nog een apart protocol moeten defini\"eren, of in geval van hergebruik van het XML-RPC protocol voorzien in een webserver en XML-RPC decoder. Als we echter voor een PUSH model kiezen, moet de scraper enkel voorzien in een XML-RPC client (wat veel minder werk is), en kan de reeds bestaande XML-RPC server code in de backend integraal herbruikt worden.
Om dit te realiseren zal de backend wel moeten voorzien in speciale functies toegespitst op de werking van de scraper, zoals het toevoegen van aandelen, of updaten van de bijhorende koersen. Ook is deze uitbreiding van het backend-protocol de druppel geweest die ons overtuigde om ten behoeve van autorisatie om te schakelen van een eenvoudige boolean-flag (isAdmin) naar een flexibeler model gebaseerd op rechten.

Om het PUSH model te realiseren moet de scraper voorzien in een algoritme dat continu alle aandelen overloopt, en selectief voor diegene die verlopen zijn nieuwe koersen ophaalt en die naar de backend stuurt. Deze functionaliteit wordt ge\"implementeerd door de StockPlay::Scraper klasse, dewelke bezit over een lijst van de geactiveerde plugins, en zo in de \emph{run()} functie van alle plugins de aandelen ophaalt en ze eventueel vernieuwt. Tijdens de refresh-run wordt tevens bepaald hoelang minimaal moet gewacht worden vooraleer er een nieuwe cyclus kan gestart worden. Om echter de belasting van de databron te minimaliseren, maakt de scraper intelligent gebruik van verschillende statistieken (zoals hoelang geleden het aandeel voor het laatst geüpdatet is) om de berekende wachttijd te transformeren naar een wachttijd passend bij het aandeel in kwestie.


%
% Artifici\"ele intelligentie
%

\chapter{Artifici\"ele intelligentie}

Als illustratie van de modulariteit van de backend, hebben we ook een artifici\"ele intelligentie aangemaakt die het spel meespeelt zoals een reguliere speler dat doet. Dit was echter geen evidente opdracht, en er was flink wat opzoekingswerk benodigd om dit tot een goed einde te brengen.


\section{Intelligentie}

Prioritair om aan de implementatie te kunnen beginnen, was het vinden van een geschikt algoritme die de intelligentie van de speler zou drijven. Hiervoor hebben we verschillende opties overwogen, gaande van expertsystemen zoals regelbanken, tot meer massief-lerende systemen zoals genetische algoritmes of neurale netwerken. Expertsystemen zijn vrij snel afgevoerd, ze werken misschien wel nauwkeurig maar hebben nood aan een uitgebreide set (moeilijk kwantificeerbare) regels die bovendien voor het opstellen een expertise vereisen die geen van ons hebben. Om aan deze eis van expertise te ontsnappen, zijn we gaan kijken naar massief lerende systemen die zichzelf kunnen verbeteren aan de hand van een ruime set aan testgegevens. Hierbij zijn genetische algoritmes misschien wel een interessante piste, maar ze vereisen een vrij specifieke implementatie: een library integreren zou veel werk kosten, laat staan het schrijven van een eigen implementatie. Zo zijn we uitgekomen op neurale netwerken, dewelke uiteindelijk ook eenvoudig integreerbaar bleken te zijn.

Een neuraal netwerk is een populaire methode om artifici\"eel gedrag te bekomen, en baseert zich daarbij op hoe een biologisch brein functioneert. Hierbij wordt dynamisch een gewogen pad gevormd tussen een vast aantal input en output neuronen, en zal via backpropagatie steeds ge\"evolueerd worden in de richting van een correctere oplossing. We hebben er voor gekozen om een aantal gegevens (zoals de koers van het aandeel, de koers van een overkoepelende index, de dag van de week, het volume, etc) als inputneuronen te registreren, en een enkel output neuron vast te leggen als de slotkoers van de volgende dag. Moesten we immers de slot van de volgende dag kunnen bepalen aan de hand van gegevens bepaald op de huidige dag, kunnen we steeds een positief-renderende portfolio samenstellen.

Bij het bepalen van de inputneuronen hebben we ons gebaseerd op bestaande kennis die te vinden was in de verschillende papers die we geraadpleegd hebben bij het kiezen van een optimale intelligentie-algoritme. In de uiteindelijke versie van het project worden de volgende inputs gebruikt:
\begin{itemize}
\item Slotkoers op dag $i$
\item Slotkoers van de meest prominente index op dezelfde beurs op dag $i$
\item Laagste koers van het aandeel op dag $i$
\item Hoogste koers van het aandeel op dag $i$
\item Verkoopvolume van het aandeel op dag $i$
\end{itemize}

Gebaseerd op dergelijke inputinformatie, zal het neuraal netwerk zich werken naar een set van outputdata die volgende informatie bevat:
\begin{itemize}
\item Slotkoers op dag $i+1$
\end{itemize}

Hoewel het ontwerpen van een eenvoudig neuraal netwerk misschien niet zoveel tijd kost, hebben we er voor gekozen om een bestaande bibliotheek te gebruiken. Verschillende belangrijke features die de performantie van het netwerk serieus verbeteren (zoals backpropagatie, shortcut-netwerken, of automatisch bepalen van een ideale activatiefunctie), zijn te moeilijk om binnen het opgegeven tijdsbestek zelf te implementeren. Bovendien is het enkel te bedoeling om de modulariteit van de backend te demonstreren, waaraan het implementeren van ingewikkelde algoritmen nauwelijks bijdraagt.


\section{Omgeving}

Bibliotheken om gebruik te maken van neurale netwerken, zijn wegens performantieredenen meestal geschreven in low-level talen zoals C of C++. Dit zou echter betekenen dat we opnieuw een datalaag zouden moeten opbouwen, en dit terwijl we reeds functionele datalagen hadden in zowel Java, Perl als C\#. Om toch niet in te moeten inboeten aan performantie, hebben we ervoor gekozen om gebruik te maken van een bestaande datalaag, en op zoek te gaan naar een bibliotheek geschreven in een performante taal, die bindings aanbiedt in de taal waarin onze datalaag geschreven is. Hierdoor is het performantieverlies minimaal, en kunnen we toch het schrijven van nog een datalaag vermijden.

Het in Perl geschreven \makeurl{AI::FANN}{http://search.cpan.org/~salva/AI-FANN-0.10/lib/AI/FANN.pm} pakket voldoet volledig aan deze specificatie: we kunnen gebruik maken van de bestaande datalaag die we reeds geschreven hebben voor de scraper, terwijl de echte performantie gehaald wordt door het interfacen met de in C geschreven bibliotheek \makeurl{FANN}{http://leenissen.dk/fann/}. Deze bibliotheek biedt tevens vele geavanceerde mogelijkheden, zoals het automatisch wijzigen van netwerkparameters via een genetisch algoritme, waarbij zowel activatiefuncties als neuronen dynamisch kunnen gewijzigd worden.


\section{Historische data}

Om een massief-lerend systeem te trainen, is er een voldoende hoeveelheid aan historische data nodig. Hierbij moet zowel de input als output waarden bekend zijn, zodat het netwerk via backpropagatie en andere technieken de gewichten van de paden in het netwerk kan bijstellen (of eventueel andere structurele veranderingen toevoegen). Die data hebben we kunnen halen bij de site van Euronext, een bron die we initi\"eel afgekeurd hadden wegens het gebrek aan realtime data, maar nu uitermate handig bleek te zijn door zijn eenvoudige interface tot data die 2 jaar in het verleden teruggaat.


\section{Modulariteit}

Om in een eventueel later stadium toe te laten om op een eenvoudige wijze gebruik te maken van alternatieve algoritmes, hebben we gebruik gemaakt van de plugin-infrastructuur ontwikkeld voor de scraper. Het is heel eenvoudig mogelijk om extra intelligentie-algoritmes toe te voegen, die dynamisch en automatisch zullen gebruikt worden bij het samenstellen van de ideale portfolio. Hierbij wordt elk algoritme steeds gebruikt, en diegene met de kleinste voorspelde fout uiteindelijk gebruikt om de slotkoers van de volgende dag te voorspellen.

Analoog aan hoe we de scraper plugin-infrastructuur gerealiseerd hebben, zal elke voorspellings-plugin  verplicht de interface StockPlay::AI::Forecaster moeten implementeren. Die klasse, die op zijn beurt StockPlay::Plugin implementeert, legt een interface vast die vervolgens door de AI manager kan gebruikt worden om het netwerk te trainen, koersen te voorspellen, foutmarges op te halen, etc. Zo worden uiteindelijk at-runtime alle voorspellers uitgeprobeerd, om uiteindelijk het nauwkeurigste resultaat te gebruiken bij selectie van een portfolio.


\section{Portfolio selectie}

Wanneer er tenslotte slotkoersen voor de volgende dag voorspeld zijn, moet er nog een optimale portfolio samengesteld worden. Dit aspect mag misschien klein lijken, het is helemaal niet evident om dit intelligent te realiseren! Een goed algoritme zou hierbij kijken naar de nauwkeurigheid van elke voorspelling, zich baseren op het uiteindelijke resultaat van vorige voorspellingen, en gebruik maken van vastgelegde parameters die bepalen hoe ``conservatief'' of net ``risicovol'' het zich moet gedragen. Maar dit valt opnieuw ruim buiten de scope van dit project, niet alleen is het geen exacte wetenschap, ook is het een relatief ongedocumenteerd onderwerp en zou er veel tijd kruipen in het ontwikkelen en benchmarken van een dergelijk algoritme. Daarom hebben we geopteerd om een eenvoudig algoritme te gebruiken, waarbij een evenredige selectie genomen wordt van de meest performante aandelen gebaseerd op een enkele voorspelling. Dit opnieuw omdat de artifici\"ele intelligentie vooral ontworpen is om de modulariteit en eenvoudige uitbreidbaarheid van de backend te demonstreren, het ontwikkelen van een complex portfolio-selectiealgoritme is hierbij irrelevant.


%
% Netwerkoptimalisaties
%

\chapter{Netwerkoptimalisaties}

Tijdens de implementatie van de applicatie, zijn we verschillende keren tegen een netwerk-gerelateerde bottleneck gelopen. Zo duurde het in eerste instantie bijvoorbeeld ettelijke seconden vooraleer de backend de laatste quotes van een bepaalde beurs opgehaald had. Om te vermijden dat de gebruiker effectief verschillende seconden moet wachten op het resultaat van een bewerking, hebben we verschillende optimalisaties toegepast.


\section{Datacompressie}

Zoals reeds vermeld bij de voordelen van het XML-RPC protocol, kunnen we eenvoudig gebruik maken van beschikbare technologie\"en. Zo hebben we om de duur van een request te verlagen, gebruik gemaakt van de compressie mogelijkheden die het HTTP-protocol reeds aanbiedt. De client kan zo in een request de \emph{Accept-Encoding} header meesturen, waarop de backend eventueel op gepaste wijze kan reageren door de reply te comprimeren. Hier zijn we echter tegen een beperking van de ws-xmlrpc library gelopen die we gebruiken binnen onze backend: enkel de \emph{gzip} compressietechniek wordt ondersteund. Dit had als gevolg dat we de Perl library \emph{RPC::XML} hebben moeten wijzigen, aangezien die enkel de \emph{deflate} compressietechniek ondersteunt (eigenlijk was dit niet nodig, bij gebrek aan ondersteuning werd compressie niet gebruikt, maar gezien de hoeveelheid data die de scraper dagelijks naar de backend stuurt was compressie toch gewenst).


\section{Caching}

Hoewel de bovenstaande netwerkoptimalisatie een grote impact heeft (grote datatransfers werden met factor 10 versneld), hielp dit niet bij de requests die de backend zelf stuurt in de richting van de database. Die requests zijn mogelijks reeds gecomprimeerd (aangezien ze verlopen via een binair protocol eigen aan de Oracle JDBC driver), maar zorgden toch voor een substanti\"ele vertraging. Zo bleek, na implementatie van datacompressie bij requests en replies, dat een groot deel van de originele delay zijn oorsprong kende bij het ophalen van data uit de database.

Daarom hebben we besloten om de datalaag in de backend uit te breiden met een flexibel caching systeem. Dit resulteerde echter in een aantal interessante problemen: wat moet er juist in de cache, wanneer worden entries geinvalideerd, en hoe bepalen we wat relevant is en wat te vluchtig is om in een cache op te slaan? Veel van deze problemen zijn niet eigen aan ons project, maar vallen binnen het domein van caches. Daarom hebben we besloten om een deel van deze problematiek te elimineren door gebruik te maken van een bestaande caching bibliotheek. De keuze is daarbij gevallen op \makeurl{cache4j}{http://cache4j.sourceforge.net/}: een compacte doch performante caching bibliotheek die net voldoende features biedt om aan onze eisen te voldoen.

De standaard opzet bij caches is om de data die de gebruiker terugkrijgt op te slaan, en wanneer de gebruiker iets opvraagt eerst de cache te controleren. Zo zouden we een cache moeten opzetten met beurzen, aandelen, koersen, etc. Wanneer de gebruiker dan een lijst van de beurzen opvraagt, kunnen we eenvoudigweg de inhoud van de cache teruggeven in plaats van de database te moeten contacteren.
De opzet van ons project bemoeilijkt dit concept echter. Aangezien we gebruik maken van een filter die nooit geïnterpreteerd wordt door de backend, maar louter gecompileerd wordt tot een string die de database dan kan verwerken, heeft de backend nooit weet van wat de effectieve selectiecriteria van de dataset waren. Daarom kunnen we het resultaat van de bevraging niet eenvoudig opslaan, laat staan in een later stadium de cache raadplegen om de juiste entries op te halen.
Om toch een cache te kunnen gebruiken, slaan we geen hiërarchie van datasets op, maar louter het letterlijke antwoord dat een gebruiker zou moeten krijgen bij een specifieke aanvraag. De sleutel die web gebruiken om de resultaatset op te slaan in de cache, is een gemodificeerde content-aware representatie van de aa,vraag, samengesteld uit de methodenaam en de argumentenlijst (meestal een filter). Zo zal een aa,vraag bij de backend, gericht op eenzelfde functie (vb ``Finance.Exchange.List''), waarbij de argumenten inhoudelijk identiek zijn aan de argumenten van een vorige aanroep, niet resulteren in een bevraging bij de database maar louter in het teruggeven van een waarde uit de cache.
Hoewel dit complex mag lijken, valt dit in Java relatief eenvoudig te implementeren. Eerst hebben we voorzien in een Proxy klasse, die elke aanvraag op object kan herleiden via een InvocationHandler (indien caching gewenst is natuurlijk). Die InvocationHandler zal in actie treden wanneer een willekeurige functie aangeroepen wordt op een object dat geregistreerd is bij de Proxy. In die methode kunnen we vervolgens de content-aware hashcode genereren, de cache controleren, eventueel een opgeslagen waarde teruggeven, of indien die ontbrak een aanroep naar het effectieve object uitvoeren om daarna het resultaat in de cache op te slaan.
Om dit proces vervolgens flexibel te maken, hebben we gebruikt van Java annotaties (\emph{@Cachable} en \emph{@Invalidates}), die de programmeur van de Data Access Objecten eenvoudig kan toevoegen aan een functiedeclaratie, om zo de interpretatie binnen een eventuele InvocationHandler te manipuleren.

Zoals vermeld hierboven hebben we ook voorzien in een \emph{@Invalidates} annotatie, die de InvocationHandler duidelijk maakt dat een bepaalde aanroep juist de bijhorende cache invalideert. Om de effici\"entie hierbij te verhogen, hebben we de cache gesplitst in een aantal deelcaches, elk geassocieerd met een specifiek Data Access Object. Zo zal een wijziging aan een beurs, enkel een cache invalidatie veroorzaken in de cache verbonden met het Data Access Object verantwoordelijk voor beurzen.

Hoewel we moeite gedaan hebben om cache invalidaties te beperken en zo specifiek mogelijk te maken, blijkt het effici\"entieverlies dat gepaard gaat met ons model, niet te verwaarlozen is. De beste optie zou natuurlijk zijn om het filtermodel te veranderen zodat de backend weet heeft van de selectiecriteria, om zo op een effici\"ente wijze resultaten op te slaan, terug te geven, en belangrijker: slechts selectief te invalideren. Het filtermodel ondersteunt deze uitbreiding, maar het zou binnen het tijdsbestek van het project niet realistisch zijn om een dergelijke interpreter te realiseren. Daarom hebben we gekozen om de performantie van de cache proberen te maximaliseren.

Om de performantie van de caches te verhogen, hebben we een cache monitor geïmplementeerd. Deze monitor haakt in op de InvocationManager, zodat het gedetailleerde informatie bevat over hoe vaak elke functieaanroep uitgevoerd wordt. Met deze informatie kan vervolgens een lijst opgesteld worden van de meest frequent uitgevoerde queries per cache. Vervolgens zal de cache monitor op geregelde tijdstippen (maar enkel wanneer de backend inactief is) de lijst met meest frequent uitgevoerde queries overlopen, kijken of het resultaat ervan in de cache zit, om indien dit niet het geval is de query uit te voeren en het resultaat ervan in de cache op te slaan.
Het lijkt misschien dubbel werk om deze informatie in een aparte entity op te slaan (deze informatie zou immers kunnen toegevoegd worden aan de caches zelf), maar daarvoor zijn de objecten benodigd om de data te kunnen ophalen (namelijk een \emph{java.lang.reflection.Method} object, naast eventuele argumenten) veel te groot en niet eenvoudig bruikbaar om als sleutel te dienen voor een cache. De hierboven vermelde inhouds-gevoelige hashcode is daarbij veel effici\"enter.


\section{Bulk methodes}

Na implementatie van bovenvermelde technieken bleek de performantie drastisch verhoogd te zijn. Toch bleek de belasting veroorzaakt door de scraper, substantieel te zijn. Combineer dit met het feit dat de scraper gemiddeld elke minuut een 50-tal quotes doorstuurt, en het wordt duidelijk dat extra optimalisaties benodigd zijn.

Vooraleerst hebben we het aantal queries geminimaliseerd door speciale \emph{bulk methodes} te implementeren. Hiermee kan de scraper n ``Finance.Security.Update'' methodes substitueren door een enkele ``Finance.Security.UpdateBulk'' aanroep. Hoewel dit de performantie duidelijk verbeterde, bleef de backend lang bezig om de quotes te versturen naar de database. Daarom hebben we de ``UpdateBulk'' methode een counterpart gegeven op database-niveau, door binnen deze gespecialiseerde routine gebruik te maken van JDBC batch statements. Hierdoor werd de belasting op de link tussen de backend en database verminderd, hoewel het verschil veel minder opvallend was dan over de XML-RPC link (dit vermoedelijk omdat het binaire protocol tussen JDBC en de Oracle database reeds weinig overhead kende, dit in tegenstelling tot een XML-RPC request die steeds opnieuw veel vaste velden, zoals de methodenaam, moest enumereren).



%
% Geavanceerde orders
%

\chapter{Geavanceerde orders}

In de order module zullen we een extra modulaire functionaliteit voorzien zodat we later extra types van geavanceerde orders kunnen toevoegen.
Deze modules accepteren twee parameters (limietkoersen), de huidige koers, het type order (aankoop of verkoop) en het tijdstip op wanneer het order aangemaakt is. Elke keer ze opgeroepen worden geven ze terug of aan de voorwaarde al dan niet voldaan is. Is de voorwaarde voldaan dan zal dit tot gevolg hebben dat het order omgezet zal worden in een effectieve aankoop van een effect.
Deze modules kunnen ook extra informatie aan de backend vragen die kunnen helpen bij de keuze. Bijvoorbeeld: ``Wat was de maximum koers van dit effect in periode x''?
Elke module krijgt bij het oproepen drie parameters:
\begin{itemize}
\item{effectcode}
\item{waarde een}
\item{waarde twee}
\end{itemize}
\todo{2 parameters? CSV? BLOB?}

Waarde een en twee zijn waardes die de gebruiker ingesteld heeft en hebben afhankelijk van de gekozen ordermodule een andere betekenis.
Hierna volgt een overzicht van de modules die we zullen implementeren.

\section{Onmiddellijk uitvoeren}
Deze module gebruikt de gegeven parameters niet en evalueert de voorwaarde steeds als waar. Het order wordt onmiddellijk uitgevoerd.

\section{Koerslimiet}
Deze module gebruikt de eerste parameter als limietwaarde. Bij een aankooporder gaat hij een positief antwoord geven als de koers lager is dan de limietwaarde. Deze koersinformatie komt hij via de backend te weten. Bij een verkooporder gebeurt het omgekeerde. 

\section{Trailing-stop}
Bij een verkoop order wordt de trigger waarde ingesteld op een vast aantal beurspunten onder de hoogste koers. Als de hoogste koers dus verhoogd in waarde, dan verhoogd ook de trigger waarde evenveel punten. Hetzelfde kan ook ingesteld worden bij een aankooporder. Deze trigger waarde krijgt de module via de parameters mee en het maximum sinds de periode van het plaatsen van de order en het huidige tijdstip vraagt deze aan de backend.

\section{Bracket-limiet}
Als de koers buiten een van de twee limietwaarden valt dan geeft de module een positief antwoord terug.

\section{Stop loss}
De module vraagt bij een aankooporder aan de backend wat de laagste prijs was tussen het plaatsen van het order en het huidige tijdstip. Is dit minimum lager dan de eerste parameter, dan geeft de module een positief antwoord terug. Bij een verkooporder gebeurt het omgekeerde.



%
% Filters
%

\chapter{Filters}

Een bijkomend probleem kwam naar boven bij de effectieve implementatie van het backend protocol. Daarbij hadden we immers voorzien in een flexibele filter-methodiek, vergelijkbaar met de WHERE clausule in een SQL statement.
Naar verloop van tijd hebben we echter een bijkomende genericiteit geïgeïntroduceerd in de backend, namelijk de Data Access Objects, waardoor het niet meer vastligt welke database effectief gebruikt zou worden. Hierdoor voldeed een filter opgemaakt in SQL-syntax niet meer, aangezien niet alle mogelijke data-backends SQL-filters ondersteunen (denk maar aan een Java in-memory data-backend, gebruik makende van datastructuren eigen aan Java zelf).

Daarom zijn we op zoek gegaan naar een methodiek om gemakkelijk filters door te geven aan de backend, mits mogelijkheden om eenzelfde filter te gebruiken in combinatie met verschillende data-backends. De JDBC backend die we nu gebruikten voor data toegang bleek daarvoor onvoldoende, aangezien die enerzijds enkel database-specifieke SQL queries aanvaardt, en anderzijds niet noodzakelijk gebruikt wordt in elke DAO implementatie. Ook bleken er geen bestaande pakketten beschikbaar te zijn om deze taak te realiseren, het werd dus snel duidelijk dat we onze eigen implementatie zouden moeten voorzien.


\section{Input formaat}

Vooraleer een implementatie te schrijven, was het imperatief om de semantiek vast te leggen: hoe geeft een gebruiker een filter door aan de backend. Dit moest voldoen aan een aantal eisen:
\begin{enumerate}
\item Compact: de data wordt over het netwerk verstuurd en mag dus niet te groot zijn.
\item Gebruiksvriendelijk: de filter wordt steeds opgesteld door een ontwikkelaar, en moet dus eenvoudig in gebruik zijn.
\item Flexibel: zowel eenvoudige als complexe filters moeten mogelijk zijn, zonder daarbij in een extreem moeilijke syntax terecht te komen.
\end{enumerate}

Een eerste methode zou zijn om de filter op te stellen aan de hand van een aantal Java objecten, waarbij aan elk object kinderen toegevoegd kunnen worden, en de root node dan instaat voor de evaluatie van de filter-boom. Hoewel deze methode makkelijk te implementeren is, is ze verre van gebruiksvriendelijk, en niet eenvoudig te versturen over het netwerk zonder gebruik te maken van serialisatieprotocollen (die onmiddellijk een vrij grote overhead introduceren). Daarom is deze optie vrij snel ge\"elimineerd.

Een volgende manier, gebaseerd op het \makeurl{Hibernate}{https://www.hibernate.org/} framework, stelt een filter op via een hi\"erarchie aan XML tags. Deze filters kunnen dan opgeslaan worden in aparte bestanden, at-runtime ingeladen worden, en aangevuld worden met de nodige parameters (vergelijkbaar per \emph{prepared statements}). Hoewel deze methode heel flexibel is, neigt de syntax snel tot complexe en onoverzichtelijke structuren. Bovendien is ze verre van compact: wat in een SQL filter een eenvoudig gelijkheidsteken is (voor het testen van gelijkheid), wordt hierbij al snel een lange reeks aan karakters met een vaste, vrij grote overhead (zoals de sluittags). Hoewel aantrekkelijker dan een filter opgebouwd met Java objecten, voldeed deze methode evenmin aan onze eisen.

Uiteindelijk zijn we meer gaan kijken naar bestaande en beproefde pakketten om onze filtersyntax te defini\"eren. Het uiteindelijke resultaat bestaat uit een SQL-achtige syntax, zoals we kunnen terug vinden in \makeurl{ASP.NET}{http://www.asp.net/}. Hierbij kan de gebruiker een filter opstellen en doorsturen als een reguliere tekenreeks, wat de eenvoud van het opstellen drastisch verbeterd. Ook blijkt de voorstellingswijze heel compact te zijn: overhead is nauwelijks aanwezig. Tenslotte kunnen we in deze syntax eenvoudig bijkomende hi\"erarchie\"en introduceren op een heel gebruikelijke manier: door het invoeren van extra haakjes. Dit maakt onze filter heel flexibel, zonder daarbij overdreven complex te worden.

\begin{code}
\begin{verbatim}
id == 42 && name EQUALS 'StockPlay'
\end{verbatim}
\caption{De uiteindelijke filter-syntax.}
\end{code}


\section{Transmissie en conversie}

Nu de semantiek van onze filter vaststond, konden we starten met een deel van de implementatie: de transmissie en conversie.
Zoals vermeld in bovenstaande sectie, stelt de gebruiker een filter-string op en zendt die door naar de backend over het XML-RPC protocol. Voor de backend is het echter moeilijk om een dergelijke string direct te verwerken, aangezien de filter veel ``menselijke opmaak'' bevat (zoals whitespace, of infix operatoren, meer hierover later). Vooraleer aan de verwerking van de filter te beginnen, is het dus belangrijk ze om te zetten naar een bruikbaar formaat.
Deze conversiemethoden zijn reeds bekend en goed gedocumenteerd in de vele naslagwerken over compilertechnieken, en we zullen ze dan ook maar kort aanraken.

\subsection{Tokenizer}

De eerste stap bestaat eruit om de ontvangen tekenreeks om te zetten naar een formaat waarbij de string opgedeeld is in syntactische blokken, die snel te identificeren en vergelijken vallen. Hiervoor wordt een lijst aan Token objecten opgebouwd, waarbij elk Token de start en eindpositie van de syntactische blok aanduidt, net als zijn type (een object van het TokenType enumeratietype, wat vergelijken eenvoudiger en sneller maakt) als eventueel een kleinere aanduiding van het syntactisch blok voor de effectieve inhoud.

Het omzetten van karakters naar dergelijke tokens gebeurt aan de hand van een set aan regels, reguliere expressies die telkens overeenkomen met een specifiek TokenType. Om mogelijke ambigu\"iteiten te voorkomen, wordt steeds de langste match effectief opgeslagen. Het detecteren van kortere blokken voor de effectieve inhoud (bijvoorbeeld: de effectieve tekst binnen een quoted string) worden de \emph{grouping} mogelijkheden van reguliere expressies gebruikt.

\begin{code}
\begin{verbatim}
Token [ 0,  2, WORD]: id
Token [ 2,  3, WHITESPACE]:  
Token [ 3,  9, WORD]: EQUALS
Token [ 9, 10, WHITESPACE]:  
Token [10, 12, INT]: 42
Token [12, 13, WHITESPACE]:  
Token [13, 16, WORD]: AND
Token [16, 17, WHITESPACE]:  
Token [17, 21, WORD]: name
Token [21, 22, WHITESPACE]:  
Token [22, 28, WORD]: EQUALS
Token [28, 29, WHITESPACE]:  
Token [29, 40, QUOTE]: StockPlay
\end{verbatim}
\caption{Infix-notatie van de filter-tekenreeks na verwerking door de Tokenizer.}
\end{code}

\subsection{Infix naar postfix omzetting}

Nu we een lijst hebben van tokens, moet een tweede typische karakteristiek van ``menselijke opmaak'' verwijderd worden: operator infix notatie. Bij deze notatie wordt een operator omring door zijn parameters, in tegenstelling tot de meer computer-gerichte notatie waarbij alle parameters voorafgegaan worden door hun operator. Het is mogelijk om deze stap over te slaan (iets wat we effectief gedaan hebben bij de eerste versies van deze filter), maar dat resulteert in veel complexere algoritmen voor het omzetten van de lijst aan tokens naar een AST.

We hebben deze stap gerealiseerd via het \emph{Shunting-Yard algoritme} van Edsger Dijkstra. Dit stack-based algoritme wordt vaak gebruikt bij het verwerken van infix-genoteerde mathematische uitdrukkingen, maar kan mits kleine wijzigingen perfect gebruikt worden voor het verwerken van onze filter. Voor de effectieve implementatie van dit algoritme verwijzen we naar het ruime aanbod van voorbeelden online.

Een beperking van dit algoritme is echter dat elke functie of operator steeds moet vastleggen hoeveel parameters die accepteert. Zonder dit kan het algoritme niet beslissen hoeveel parameters er van de stack moeten gehaald worden. In de huidige opzet van ons filterontwerp hebben we echter dergelijke functies niet nodig, en daarom is er ook geen voorziening geïmplementeerd om er mee te kunnen opgaan. Moest dit echter wel het geval zijn, is het mogelijk het shunting-yard aloritme \makeurl{uit te breiden}{http://www.kallisti.net.nz/blog/2008/02/extension-to-the-shunting-yard-algorithm-to-allow-variable-numbers-of-arguments-to-functions/} voor gebruik met een willekeurig aantal parameters, wat zeker handig is in de context van functies.

\begin{code}
\begin{verbatim}
Token [ 0,  2, WORD]: id
Token [10, 12, INT]: 42
Token [ 3,  9, WORD]: EQUALS
Token [17, 21, WORD]: name
Token [29, 40, QUOTE]: StockPlay
Token [22, 28, WORD]: EQUALS
Token [13, 16, WORD]: AND
\end{verbatim}
\caption{Postfix-notatie van de filter-tekenreeks na omzetting door het shunting-yard algoritme. Merk ook op dat hierbij nutteloze token (zoals whitespace tussen andere tokens) verwijderd is.}
\end{code}

\subsection{Bouwen van een AST}

Nu een lijst van tokens, opgebouwd in postfix notatie, beschikbaar is, kunnen we eenvoudig deze Reverse Polish Notation (RPN) evalueren naar een Abstract Syntax Tree (AST). Deze boomstructuur bevat tevens geen tokens meer, maar instanties van de relevante objecten.

Dit stadium bestaat dus essentieel uit twee onderdelen:
\begin{itemize}
\item Omzetting van tokens naar relevante objecten.
\item Conversie van RPN naar een boomstructuur.
\end{itemize}

Het eerste aspect hiervan is triviaal: gebruik makende van een secundaire set aan regels (opbouw identiek met de regels uit de tokenizer, echter komt een regel nu niet meer overeen met een TokenType maar met een Class) kunnen we op eenvoudige doch flexibele wijze tokens omzetten naar een lokaal object.

Voor het tweede gedeelte gebruiken we een relatief eenvoudig algoritme dat normaal gebruikt wordt voor het \makeurl{evalueren}{http://en.wikipedia.org/wiki/Reverse\_Polish\_notation\#The\_postfix\_algorithm} van postfix-expressies. We breiden het echter uit, zodat we een boom van niet-ge\"evalueerde structuren kunnen opstellen. Om daarbij geavanceerde syntactische expressies toe te laten, waarbij het (nog onbekende) uitvoeringsresultaat van een operator dient als input voor een andere operator, hebben we gekozen voor de volgens opzet: elk object dat in de boom kan optreden implementeert de Convertable interface, die voorziet in een \emph{compile()} instructie. Die instructie zal voor data-objecten louter zijn inhoud teruggeven, maar zal voor complexere sub-bomen alle nodes evalueren en het eindresultaat van die evaluatie teruggeven. Zo kan waar een operator een data-object verwacht, eveneens een schakering aan operatoren doorgegeven worden aangezien die eveneens in dezelfde \emph{compile()} instructie voorzien.

\begin{figure}[h!]
	\centering
		\includegraphics[width=\textwidth]{images/realisatie/AST}
	\caption{Abstract Syntax Tree van een voorbeeldfilter.}
\end{figure}


\section{Verwerking}

Nu de semantiek en conversie van onze filter vastligt, konden we starten met de effectieve implementatie ervan: het omzetten naar of toepassen van de filter op een variabele data-backend. Hierbij zijn de twee grote pistes direct duidelijk: toepassing, of omzetting.

\subsection{Lokale toepassing}

Bij deze opzet wordt de filter lokaal toegepast op een binnengehaalde dataset. Het idee hierbij was om alle records op te halen, en de filter dan lokaal een selectie te laten maken. Die selectie, die voldoet aan de eisen die door de gebruiker zijn opgelegd, kan dan teruggestuurd worden.

Hoewel dit idee perfect toepasbaar is bij systemen waar alle data lokaal aanwezig is (zoals een lokale Java data-backend), treden er problemen op als de data enkel remote aanwezig is en lokaal niet gecachet wordt. In dit geval wordt de overhead van het ophalen van alle records veel te groot, zeker toegepast op StockPlay waarbij de Securities-tabel gemakkelijk meer dan duizend records kan bevatten. Hoewel deze evaluatietechniek dus eenvoudiger te implementeren is (het data-backend-specifieke gedeelte is niet meer aanwezig), hebben we door de grote overhead besloten een alternatief te zoeken.

\subsection{Omzetting}

Het alternatief voor een lokale toepassing, is een omzetting van de filter-boom naar een formaat dat geschikt is om op afstand verwerkt te worden. Hierbij is er geen overhead meer aanwezig, maar wordt het geheel een pak complexer daar het moet kunnen omgezet worden naar een specifieke syntax, afhankelijk van de data-backend.

Om dit te kunnen realiseren, hebben we eerst voorzien in een pseudo-abstracte klasse (Convertable, wel instantieerbaar, maar de compile() functie is ``verboden'') die voorziet in een interface en gemeenschappelijke functionaliteit, onder meer voor het verwerken van eventuele parameters. Filter sub-objecten (Condities, Relaties, en Data objecten) breiden vervolgens deze abstracte klasse gepast uit, door niet zozeer functionaliteit toe te voegen, maar louter voorziet in een hi\"erarchie (zo kunnen we later bijvoorbeeld zeggen dat een Condition enkel Data-objecten als argumenten kan aanvaarden).
Het volgend niveau in het Filter model is dat van de effectieve Condities, Relaties, en Datatypes. Een dergelijke klasse (vb ConditionEquals) breidt opnieuw zijn ouder uit, waarbij de toegevoegde functionaliteit bestaat uit het opbouwen van een graaf-subtree (meer hierover later), het specificeren van een functiesignatuur, en het vereisen van een \emph{compile()} functie voor alle implenterende klassen.
Het finale niveau in ons model voorziet in een implementatie van bovenstaande objecten. Zo zal elke Convertable een implementatie moeten hebben voor elke data-backend (bijvoorbeeld sql.ConditionEquals). Via introspectie zal een Convertable at-runtime een correcte implementatie instantieren, wiens \emph{process()} functie dan instaat voor het genereren van een object dat bruikbaar is voor een data-backend (in geval van SQL zal dit een String).

Deze opzet (pseudo-abstracte Convertable objecten die opnieuw een Convertable instanti\"eren die nu wel over een compile() functie beschikt) mag misschien complex lijken, maar het is elementair in het generiek maken van de filters. Aangezien de pseudo-abstracte Convertable objecten rechtstreeks instantieerbaar zijn, kan de Parser een boomstructuur aan dergelijke objecten opstellen zonder weet te hebben van hoe die boom uiteindelijk zal gecompileerd worden. Het is slechts wanneer die aanroep effectief gebeurt, dat elk Convertable object zal opvragen hoe hij gecompileerd moet worden, om zo de juiste Converter op te roepen. Die Converters voorzien ook niet in een \emph{compile()} instructie, omdat die instructie geen parameters vastlegt en dus manueel de private dataleden zou moeten raadplegen om parameters op te halen. Om dit te vermijden hebben we gekozen voor een extra instructie, de \emph{process()} instructie, die w\'el voorziet in een specifieke signatuur om zo verkeerde implementaties van de Convertables te vermijden.

\begin{code}
\begin{verbatim}
id = 42 AND name = "StockPlay"
\end{verbatim}
\caption{Finaal resultaat na omzetting door de SQL-converters.}
\end{code}


\section{Syntaxreferentie}

Zoals reeds beschreven bestaat een filter essenti\"eel uit drie verschillende mogelijke objecten: condities, relaties, en data-objecten. In deze sectie documenteren we die types, en de exacte grammatica die gebruikt moet worden.

\subsection{Operatoren}

\begin{itemize}
\item{Ronde haakjes: wordt gebruikt om conditities en relaties te groeperen om precedentie in te voeren.}
\item{Komma: wordt gebruikt om de argumenten die aan een functie doorgegeven worden, te scheiden (momenteel ongebruikt).}
\end{itemize}

Ronde haakjes zijn nodig wanneer precedentie onduidelijk is. Aangezien er geen niveaus van precendentie gedefini\"eerd zijn tussen verschillende relaties, zullen er altijd haken moeten gebruikt worden wanneer er zich verschillende relaties in eenzelfde filter bevinden. Bij gelijke relaties wordt er echter impliciet linkse-precedentie toegepast, en zijn haken niet vereist.

\begin{code}
\begin{verbatim}
-- door de impliciete linkse-precedentie wordt eerst de
-- id-expressie ge\"evalueerd, vervolgens de age-expressie en
-- tenslotte de name-expressie.
id >= 5 && age >= 10 && name != 'Jan's

\end{verbatim}
\caption{Demonstratie van de impliciete linkse-precedentie.}
\end{code}

\subsection{Datatypes}

De verschillende datatypes kunnen op verschillende manieren opgesteld worden: rechtstreeks, of indirect. De directe manier is echter enkel beschikbaar voor primitieve dataobjecten (integers, floating-point getallen, strings, en sleutels). Hierbij moet de gebruiker louter de waarde van het getal invoegen in de filter, en zal de parser detecteren om welk formaat het gaat.
Indirecte contructie is steeds mogelijk, en is some de enige optie (bij complexere dataobjecten). Hierbij geeft de gebruiker de data in als een \emph{quoted string} (die voldoet aan een vaste grammatica), en zal een karakter na de laatste quote (de \emph{modifier}) indiceren om welk datatype het gaat.

\subsubsection{Sleutels}

Dit speciaal datatype wordt gebruikt om bij een conditie een specifieke kolom te selecteren. Het mag enkel bestaan uit alfabetische karakters (zowel lower- als uppercase), en een \_ symbool. Het wordt niet omringd door aanhalingstekens, indien dit wel het geval zou zijn wordt het gewoon ge\"interpreteerd als een reguliere string.

\begin{code}
\begin{verbatim}
-- id is een sleutel: verwijst naar een specifieke kolom om de
-- gelijkheids-conditie op in te laten werken
id EQUALS 5
\end{verbatim}
\caption{Illustratief gebruik van een sleutel.}
\end{code}

\subsubsection{Tekenreeksen}

Dit datatype stelt een ordinaire string voor, en kan alle karakters omvatten op voorwaarde dat het omsloten wordt door enkele quotes (en mag er dus geen bevatten, \emph{escaping} wordt niet ondersteund. Zoals wellicht zal opvallen lijkt deze notatie sterk op de indirecte constructiemethode die kan gebruikt worden voor andere datatypes, en eigenlijk is het ook zo geimplementeerd: bij afwezigheid van een modifier na de laatste quote wordt die impliciet verondersteld als zijnde een 's', de modifier voor strings.

\begin{code}
\begin{verbatim}
name EQUALS 'Foo'
name EQUALS 'Foo's
\end{verbatim}
\caption{Illustratief gebruik van een tekenreeks.}
\end{code}

\subsubsection{Natuurlijke getallen -- int}

Dit primitief datatype wordt gebruikt om natuurlijk getallen te detecteren, en mag enkel getallen bevatten (eventueel voorafgegaan door een minteken). Het datatype kan rechtstreeks geconstrueerd worden, en bij indirecte constructie wordt het voorgesteld door de modifier 'i'.

\begin{code}
\begin{verbatim}
id EQUALS 5
id EQUALS '5'i
\end{verbatim}
\caption{Illustratief gebruik van een natuurlijk getal.}
\end{code}

\subsubsection{Gehele getallen -- float}

Dit is een uitbreiding van de natuurlijke getallen, en mag zo ook een decimaal punt bevatten om kommagetallen aan te duiden. Het datatype kan rechtstreeks geconstrueerd worden, en bij indirecte constructie wordt het voorgesteld door de modifier 'f'.

\begin{code}
\begin{verbatim}
cash EQUALS 123.45
cash EQUALS '123.45'f
\end{verbatim}
\caption{Illustratief gebruik van een geheel getal.}
\end{code}

\subsubsection{Datums -- date}

Dit is een voorbeeld van een complex type dat enkel kan geconstrueerd worden via de indirecte methode, gebruik makend van modifier 'd'. Aangezien datums op verschillende manieren kunnen voorgesteld worden, accepteren we enkel volgende methoden (gestandaardiseerd in ISO 8601):
\begin{itemize}
\item{YYYY-MM-DD: 2010-04-07}
\item{YYYY-MM-DD'T'HH:MM'Z': 2010-04-07T05:56Z}
\end{itemize}

Uren zijn steeds in 24-uurs formaat, gespecificeerd in de UTC-tijdzone. Bij afwezigheid van het uur (vb. in YYYY-MM-DD), wordt dit impliciet ingesteld op '00:00' (middernacht), en zal ook zo in de database-backend verwerkt worden. De converters hebben dus geen weet van de manier waarop de datum ingegeven is (al dan niet met gespecificeerd uur).

\begin{code}
\begin{verbatim}
date EQUALS '2010-04-07'd
date EQUALS '2010-04-07T16:01Z'd
\end{verbatim}
\caption{Illustratief gebruik van een datum.}
\end{code}

\subsubsection{Reguliere expressies -- regex}

Dit complex datatype wordt steeds gebruikt in combinatie met de LIKE operator, en kan enkel geconstrueerd worden via de indirecte methode, gebruik makend van de modifier 'r'. Dit datatype ondersteund als enige ook extra modifiers, die het gedrag van de reguliere expressie verfijnen. De volgende extra modifiers worden ondersteund:
\begin{itemize}
\item 'i' modifier: zorgt dat de reguliere expressie case-insensitive werkt
\end{itemize}

\begin{code}
\begin{verbatim}
name LIKE '^j*n$'ri
\end{verbatim}
\caption{Illustratief gebruik van een reguliere expressie.}
\end{code}

\subsection{Condities}

Een conditie wordt gebruikt om een specifieke vergelijkingsoperatie toe te passen op een kolom, zodat er een selectie gebeurd op basis van deze conditie. De mogelije condities (met tussen haken de signatuur) zijn:

\begin{itemize}
\item == of EQUALS: vereist dat twee velden identiek zijn [Key, *]
\item != of NOTEQUALS: vereist dat twee velden verschillend zijn [Key, *]
\item > of GREATHERTHAN: vereist dat het ene veld groter is dan het andere [Key, *]
\item < of LESSTHAN: vereist dat het ene veld kleiner is dan het andere [Key, *]
\item >= of GREATHERTHANOREQUAL: vereist dat het ene veld strikt groter is dan het andere [Key, *]
\item <= of LESSTHANOREQUAL: vereist dat het ene veld strikt kleiner is dan het andere [Key, *]
\item =~ of LIKE: vereist dat het ene veld voldoet aan de string (met wildcards) in het tweede veld [Key, String]
\item !~ of NOTLIKE: vereist dat het ene veld verschilt van de string (met wildcards) in het tweede veld [Key, String]
\end{itemize}

\subsection{Relaties}

Relaties dienen om verschillende condities aan elkaar te schakelen, en het resultaat op een specifieke wijze te interpreteren.

\begin{itemize}
\item \&\& of AND
\item || of OR
\end{itemize}

\subsection{Functies}

Momenteel zijn er nog geen functies gedefini\"eerd in de parser, maar de infrastructuur is er op voorzien. Mits een eenvoudige uitbreiding van de tokenizer kan er zelf gebruik gemaakt worden van functies met een variabel aantal argumenten.


